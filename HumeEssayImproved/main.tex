%%%%%%%%%%%%%%%%%%%%%%%%%%%%%%%%%%%%%%%%%
% Diaz Essay
% LaTeX Template
% Version 2.0 (13/1/19)
%
% This template originates from:
% http://www.LaTeXTemplates.com
%
% Authors:
% Vel (vel@LaTeXTemplates.com)
% Nicolas Diaz (nsdiaz@uc.cl)
%
% License:
% CC BY-NC-SA 3.0 (http://creativecommons.org/licenses/by-nc-sa/3.0/)
%
%%%%%%%%%%%%%%%%%%%%%%%%%%%%%%%%%%%%%%%%%

%----------------------------------------------------------------------------------------
%	PACKAGES AND OTHER DOCUMENT CONFIGURATIONS
%----------------------------------------------------------------------------------------
\PassOptionsToPackage{ngerman}{babel}
\documentclass[ngerman,12pt]{diazessay} % Font size (can be 10pt, 11pt or 12pt)
\setcounter{tocdepth}{4}
\usepackage{setspace}
  \usepackage[
backend=biber,
style=trad-abbrv,
]{biblatex}
\addbibresource{sample.bib}

%----------------------------------------------------------------------------------------
%	TITLE SECTION
%----------------------------------------------------------------------------------------

\title{\textbf{David Humes Argumentation gegen die Fähigkeit der Vernunft, das sittlich Gute vom sittlich Bösen zu unterscheiden} \\
\vspace{3em}
{\Large\itshape Eine Rekonstruktion \vspace{4em}}} % Title and subtitle

\author{
FAU Erlangen \\
\textit{Philosophische Fakultät, Grundkurs Praktische Philosophie} \\
Dozent Prof. Dr. Erasmus Mayr, Tutorin Velia Fischer \\
\textbf{Simon Dorr, Matrikel-Nr. 23132534} \\
simon.dorr@fau.de \\
1. Fachsemester Philosophie/Politikwissenschaft \\
WS22/23
} % Author and institution
 % Date, use \date{} for no date

%----------------------------------------------------------------------------------------

\begin{document}
\begin{doublespace}
\selectlanguage{ngerman}
\maketitle % Print the title section
\newpage
%----------------------------------------------------------------------------------------
%	ABSTRACT AND KEYWORDS
%----------------------------------------------------------------------------------------

%\renewcommand{\abstractname}{Summary} % Uncomment to change the name of the abstract to something else

\tableofcontents
\newpage
%----------------------------------------------------------------------------------------
%	ESSAY BODY
%----------------------------------------------------------------------------------------

\section{Einleitung}

Der schottische Philosoph David Hume stellt im Abschnitt: \textit{Von den Motiven des Willens} seines \textit{Traktats über die menschliche Natur} zu seinem Bedauern fest, dass der Affekt als Motiv unserer Handlungen einen üblen Ruf besitzt. Seiner Meinung nach besteht sogar eine Forderung der Allgemeinheit, nach der jedes vernünftige Wesen seine Handlungen ausschließlich nach seiner Vernunft einrichten soll. Er selbst hält diese Forderung für unsinnig. Und er macht es sich zum Ziel, diese Sinnlosigkeit zu beweisen. Also stellt er die These auf, dass die Vernunft allein niemals einen Willen erzeugen oder unterdrücken kann \cite[siehe S.484]{Hume.2013}. Die Erkenntnisse aus der Behandlung dieser These verwendet Hume anschließend dazu, seinen Gedanken vom allgemeinen Handeln auf das sittliche Handeln zu übertragen. Die daraus entstehende zweite These besagt, dass: „[D]as sittlich Gute und das sittlich Böse allein durch die Vernunft zu unterscheiden [nicht möglich ist]“ \cite[S.433]{Hume.2013}. In diesem Essay ist es mein Ziel, in schlüssiger und verständlicher Form die Argumentation für die zweite These darzulegen, weswegen natürlich eine ebensolche Argumentation für seine erste These notwendig ist.
\par\bigskip   
Des Weiteren werde ich die Frage behandeln, in welchem Verhältnis Hume die Vernunft und die Affekte zueinander stehen sieht.

%------------------------------------------------

\section{Das Verhältnis von Affekt, Wille, Handlung und Vernunft}

Um die nachfolgenden Erläuterungen verständlicher zu machen, beginne ich mit Humes Untersuchung des Wesens jedes Affekts. Er stellt dabei fest, dass ein subjektiv von mir erfahrener Affekt keine Repräsentation von etwas ist, das außerhalb meiner Erfahrung liegt. Der Affekt ist kein Abbild eines sich in der Welt befindlichen Gegenstands, sondern ein „originales Etwas“, das tatsächlich in mir ist \cite[siehe S.486]{Hume.2013}. Es gibt also nichts, was der Affekt repräsentiert. Deshalb gibt es auch nichts, was mit dem Affekt verglichen werden kann, um eine Übereinstimmung bzw. Nicht-Übereinstimmung zu erkennen. In dieser Erkenntnis aber liegt das Fundament von Wahrheit und Irrtum \cite[siehe S.534]{Hume.2013}. So zeigt Hume, dass ein Affekt keinerlei Beziehung zur Wahrheit haben und damit auch nicht der Vernunft wider-, oder entsprechen kann. Im folgenden Rekonstruiere ich Humes Darlegung des umgekehrten Schlusses, nämlich, dass die Vernunft allein niemals einen Willen erzeugen oder unterdrücken kann.

\subsection{Einteilung der Vernunft}

Um den Beweis zu erbringen, dass die Vernunft allein niemals einen Willen erzeugen oder unterdrücken kann, teilt Hume die Tätigkeit der Vernunft in zwei Kategorien ein. „[E]s gibt keine dritte Tätigkeit“ \cite[S.540]{Hume.2013}. Die Kategorien sind folgende:
\begin{enumerate}
	\item Das Urteilen über die Beziehungen von inneren Vorstellungen
	\item Das Urteilen über die Beziehungen von wahrnehmbaren Gegenständen
\end{enumerate}
\cite[vgl. S.540]{Hume.2013} \\
Da nun alle Beziehungen von inneren Vorstellungen unabhängig von der Realität sind, können sie uns nur über Ursachen und Wirkungen aufklären, aber niemals direkt den Willen beeinflussen \cite[siehe S.484]{Hume.2013}. Gleichzeitig aber hat jeder Wille eine Verknüpfung zu einem Gegenstand der realen Welt - denn nur ein solcher kann Ab-, oder Zuneigung in uns hervorrufen und damit unseren Willen beeinflussen. Dadurch ist bewiesen, dass das Urteilen über die Beziehungen von inneren Vorstellungen nie der Ursprung eines Willens sein kann. Weiterhin kann sich durch das Erkennen von Ursache/Wirkung Zusammenhängen dieses Gegenstands mit anderen Gegenständen eine Wirkung auf den Willen zwar übertragen, aber diese besteht nicht aus der Leistung der Vernunft, sondern aus unserem Gefühl gegenüber dem ersten Gegenstand \cite[siehe S.485]{Hume.2013}. Eine Einwirkung des Verstandes auf den Willen ist also gänzlich unmöglich, auch nicht durch Unterdrückung, da ein Impuls, der in der Lage ist einen Willen zu unterdrücken, wenn er für sich allein wirken würde, auch einen Willen erzeugen müsste. Da die Vernunft aber nie der Ursprung eines Willens sein kann, kann sie niemals so einen Impuls geben. Deshalb kann sie auch nie einen Impuls geben, der einem Willen entgegen wirkt. Es gilt: Die Vernunft kann keinen Willen erzeugen oder unterdrücken \cite[S.486]{Hume.2013}. 
\par\bigskip 
Außerdem wird somit auch gezeigt, dass jede Handlung immer auf einen Affekt zurückzuführen ist, weil sie nur durch einen Willen hervorgerufen werden kann \cite[siehe S.489]{Hume.2013}.\footnote{Diese Aussage ist im Sinne der Beweisführung vereinfacht. In der Wirklichkeit ist es immer eine Vielzahl von Affekten, die um ihren Einfluss auf unseren Willen konkurrieren. Hume sagt außerdem, dass sich die Gewichtung der Affekte gegeneinander mit der Stimmung des Menschen ändert.} Und da für einen Affekt gilt, dass er nicht der Vernunft wider-, oder entsprechen kann, und nur ein Affekt - nicht die Vernunft - einen Willen erzeugen kann, muss also dasselbe für jede Handlung gelten.

\subsection{Übertragen des Gesagten auf die Sittlichkeit}
Den Begriff der Sittlichkeit bindet Hume an den Prozess der Beurteilung einer Handlung oder eines Charakters als sittlich Gut oder sittlich Böse. Mit der Untersuchung dessen, auf welche Weise der Mensch diese Beurteilung vornimmt, untersuchen wir somit auch die menschliche Sittlichkeit \cite[siehe S.532]{Hume.2013}.
\par\bigskip   
Hume setzt a priori voraus, dass die Sittlichkeit Affekt, Willen und Handlungen beeinflusst. Er begründet diese Voraussetzung nicht logisch, sondern appelliert an die „allgemeine Erfahrung“, von der er sagt: „diese lehrt uns, daß Menschen [...] von Handlungen zurückgehalten
werden, weil sie dieselben für unrecht [sittlich Böse] ansehen, und das Gefühl
der Verpflichtung [gegenüber dem sittlich Guten] sie zu anderen Handlungen antreibt.“ \cite[S.533]{Hume.2013}. Stimmt man mit Hume überein, dass diese Voraussetzung gilt, so muss man auch mit folgendem Schluss übereinstimmen: Die Vernunft kann nicht fähig sein, das sittlich Gute vom sittlich Bösen zu unterscheiden. Denn wäre sie dazu in der Lage, würde ihr Urteil in der Lage dazu sein, einen Willen in uns zu erzeugen. Und der Beweis, dass die Vernunft nie der Ursprung eines Willens sein kann, wurde bereits erbracht, da weder das Urteilen über die Beziehungen von inneren Vorstellungen, noch das Urteilen über die Beziehungen von wahrnehmbaren Gegenständen diese Macht besitzt. Eine andere Art es auszudrücken ist diese: Die Erkenntnis des sittlich Guten oder sittlich Bösen hat die Macht, Einfluss auf unsere Handlungen auszuüben. Da die Vernunft aber nie einer Handlung wider-, oder entsprechen kann, kann sie auch nicht diese geforderte Macht besitzen \cite[siehe S.534]{Hume.2013}.

\subsection{Das Verhältnis von Vernunft und Affekt}

Das Verhältnis von Vernunft und Affekt ist also unbedingt nur als ein indirektes zu verstehen. Der fälschliche Eindruck des direkten Verhältnisses entsteht dadurch, dass die Änderung eines Affekts unter einem bestimmten Umstand sofort einer Tätigkeit der Vernunft folgen kann \cite[siehe S.488]{Hume.2013}. Dieser bestimmte, notwendige Umstand ist das Bestehen eines Irrtums. Und das Verhältnis von Vernunft und Affekt besteht in der Fähigkeit der Vernunft, mich über diesen Irrtum aufzuklären \cite[siehe S.536]{Hume.2013}. Zwei Arten von Irrtümern sind möglich: Der erste Irrtum kann über die Existenz eines Gegenstands bestehen. In diesem Fall gibt es eine Differenz zwischen der Wirklichkeit des Gegenstands und meiner Vorstellung von ihm. Die zweite Art des Irrtums kann mir bei meinem Urteil über einen Ursache/Wirkung Zusammenhang unterlaufen \cite[siehe S.536]{Hume.2013}.
\par\bigskip   
Konsequent folgert Hume aus seinen bisherigen Schlüssen folgendes: Sobald wir uns der Wahrheit über einen solchen Irrtum bewusst werden, ändern sich gleichzeitig mit der Auflösung des Irrtums auch unsere Affekte \cite[siehe S.488]{Hume.2013}.

%------------------------------------------------

\section{Schluss}

Dieses nun zuletzt besprochene Verhältnis von Vernunft und Affekt ist natürlich kein unbedeutendes, hat aber keine Bedeutung für die spezielle Frage nach dem eigentlichen Ursprung unserer Affekte. Denn haben wir ein klares, korrektes Bild von einer Sachlage, dann können Vernunft und Affekt in keinerlei Verhältnis zueinander stehen. Und dementsprechend kann dann auch niemals unvernünftig genannt werden, was immer unser Wille im Kontext dieser Sachlage ist.
Diese Schlussfolgerung spitzt Hume treffend zu:
\begin{quote}
„Es läuft der Vernunft nicht zuwider,
wenn ich lieber die Zerstörung der ganzen Welt will als einen
Ritz an meinem Finger.“ \cite[siehe S.487]{Hume.2013}
\end{quote}
Ich hoffe dieser Essay konnte nachvollziehbar darlegen, warum Hume mit dieser Aussage recht hat. Keine Handlung und kein Wille können je durch die Vernunft entstehen. Und weil mich das sittlich Gute doch zum Handeln drängt, so muss jedes sittliche Urteil am Ende von nichts anderem als einem mir eigenem Gefühl geleitet sein.
%----------------------------------------------------------------------------------------
%	BIBLIOGRAPHY
%----------------------------------------------------------------------------------------
\newpage
\printbibliography


%----------------------------------------------------------------------------------------
\end{doublespace}
\end{document}
