% universal settings
\PassOptionsToPackage{ngerman}{babel}
\documentclass[ngerman,smalldemyvopaper,11pt,oneside,onecolumn,openright,extrafontsizes]{memoir}
\usepackage[utf8x]{inputenc}
\usepackage[T1]{fontenc}
\usepackage[osf]{Alegreya,AlegreyaSans}
% PACKAGE DEFINITION
% typographical packages
\usepackage{microtype} % for micro-typographical adjustments
\usepackage{setspace} % for line spacing
\usepackage{lettrine} % for drop caps and awesome chapter beginnings
\usepackage{titlesec} % for manipulation of chapter titles
% for placeholder text
\usepackage{lipsum} % to generate Lorem Ipsum
% other
\usepackage{calc}
\usepackage{hologo}
\usepackage[hidelinks]{hyperref}
%\usepackage{showframe}
% PHYSICAL DOCUMENT SETUP
% media settings
\setstocksize{8.5in}{5.675in}
\settrimmedsize{8.5in}{5.5in}{*}
\setbinding{0.175in}
\setlrmarginsandblock{0.611in}{1.222in}{*}
\setulmarginsandblock{0.722in}{1.545in}{*}
% defining the title and the author
%\title{\LaTeX{} ePub Template}
%\title{\textsc{How I Started to Love {\fontfamily{cmr}\selectfont\LaTeX{}}}}
\title{Der Schwimmer}
\author{Simon Dorr}
\newcommand{\ISBN}{0-000-00000-2}
\newcommand{\press}{My very own studio}
% custom second title page
\makeatletter
\newcommand*\halftitlepage{\begingroup % Misericords, T H p 153
\setlength\drop{0.1\textheight}
\begin{center}
\vspace*{\drop}
\rule{\textwidth}{0in}\par
{\Large\textsc\thetitle\par}
\rule{\textwidth}{0in}\par
\vfill
\end{center}
\endgroup}
\makeatother
% custom title page
\thispagestyle{empty}
\makeatletter
\newlength\drop
\newcommand*\titleM{\begingroup % Misericords, T H p 153
\setlength\drop{0.15\textheight}
\begin{center}
\vspace*{\drop}
\rule{\textwidth}{0in}\par
{\HUGE\textsc\thetitle\par}
\rule{\textwidth}{0in}\par
{\Large\textit\theauthor\par}
\vfill
{\Large\scshape\press}
\end{center}
\endgroup}
\makeatother
% chapter title manipulation
% padding with zero
\renewcommand*\thechapter{\ifnum\value{chapter}<10 0\fi\arabic{chapter}}
% chapter title display
\titleformat
{\chapter}
[display]
{\normalfont\scshape\huge}
{\HUGE\thechapter\centering}
{0pt}
{\vspace{18pt}\centering}[\vspace{42pt}]
% typographical settings for the body text
\setlength{\parskip}{0em}
\linespread{1.09}
% HEADER AND FOOTER MANIPULATION
% for normal pages
\nouppercaseheads
\headsep = 0.16in
\makepagestyle{mystyle} 
\setlength{\headwidth}{\dimexpr\textwidth+\marginparsep+\marginparwidth\relax}
\makerunningwidth{mystyle}{\headwidth}
\makeevenhead{mystyle}{}{\textsf{\scriptsize\scshape\thetitle}}{}
\makeoddhead{mystyle}{}{\textsf{\scriptsize\scshape\leftmark}}{}
\makeevenfoot{mystyle}{}{\textsf{\scriptsize\thepage}}{}
\makeoddfoot{mystyle}{}{\textsf{\scriptsize\thepage}}{}
\makeatletter
\makepsmarks{mystyle}{%
\createmark{chapter}{left}{nonumber}{\@chapapp\ }{.\ }}
\makeatother
% for pages where chapters begin
\makepagestyle{plain}
\makerunningwidth{plain}{\headwidth}
\makeevenfoot{plain}{}{}{}
\makeoddfoot{plain}{}{}{}
\pagestyle{mystyle}
% END HEADER AND FOOTER MANIPULATION
% table of contents customisation
\renewcommand\contentsname{\normalfont\scshape Contents}
\renewcommand\cftchapterfont{\normalfont}
\renewcommand{\cftchapterpagefont}{\normalfont}
\renewcommand{\printtoctitle}{\centering\Huge}
% layout check and fix
\checkandfixthelayout
\fixpdflayout
% BEGIN THE DOCUMENT
\begin{document}
\emergencystretch=7pt
\pagestyle{empty}
% the half title page
\halftitlepage
\cleardoublepage
% the title page
\titleM
\clearpage
% copyright page
\noindent{\small{This novel is entirely a work of fiction. The names, characters and incidents portrayed in it are the product of the author's imagination. Any resemblance to actual persons, living or dead, or events or localities is entirely coincidental.\par\vfill\noindent Paperback Edition\space\today\\ISBN\space\ISBN\\\copyright\space\theauthor. All rights reserved.\par\vfill\noindent\theauthor\space asserts the moral right to be identified as the author of this work. All rights reserved in all media. No part of this publication may be reproduced, stored in a retrieval system, or transmitted, in any form, or by any means, electronic, mechanical, photocopying, recording or otherwise, without the prior written permission of the author and/or the publisher.\par}}
\clearpage
% dedication
\begin{center}
\itshape{\noindent{Dedicated shamelessly to myself}}
\end{center}
% begin front matter
\frontmatter
\pagestyle{mystyle}
% table of contents
\clearpage
\tableofcontents*
% begin main matter
\mainmatter
\chapter{Spazierengehen}
Die am Wegrand wachsenden Bäume erinnerten Leno an etwas. Leider konnte er sich nicht richtig bewusst machen, was es war. Er war nun schon einige Zeit unterwegs, war den langgezogenen Weg hinter dem Wohnblock entlang geschlendert, und stand nun seit einigen Momenten wie angewachsen still. Seine Hände stemmten sich in die Träger seines Rucksacks, die sich durch den Druck tief in seine sehnigen Schultern pressten. Er verharrte in einer sehr geraden, aufrechten Position mit einem hüftbreiten Stand. Seine Gedanken drehten sich um die Zukunft während sein Blick auf den Blättern des längsten Asts des größten Baums in seinem Sichtfeld ruhte, der etwas Abseits stehend die ansonsten eher jungen Pflanzen deutlich überragte. Der Baum war, das sah man ihm an, alt, stand schon sehr lange hier, und hatte bereits viele Tage kommen und gehen sehen. Der Wind, der durch die Zweige und Blätter blies, wog den Ast mit einer großen Kraft hin und her. Und Leno nahm den Anblick zum Anlass, sich ein wenig Optimismus zu erlauben. Seinen Blick ließ er mit dem Schaukeln des Asts wandern, und dabei ging er alle Eventualitäten durch und meisterte sie in Gedanken elegant. Er war voller Vorfreude und Zuversicht, voller Glaube an sich selbst und voller Vertrauen in die Güte des Universums, das es bestimmt gut mit ihm meinen würde.
Dann unterbrach er sein Tagträumen und wandte den Blick von den Blättern ab hin zu seinen Füßen, wo sich ein leichter Schmerz um seine Aufmerksamkeit bemühte. Er fokussierte sich auf das Gefühl im linken Fuß und stellte fest, dass der Ursprung des Schmerzes ein kleiner Stein war, welcher sich anscheinend irgendwie zwischen Fußsohle und Sandale verfangen hatte. Da löste er sich aus der scheinbaren Versteinerung, in welcher er bis eben verweilt hatte, hob den betroffenen Fuß an, blieb einen Augenblick auf einem Bein stehen, zog dann seine Zehen nach oben und schwenkte den verbleibenden, in der Luft hängenden Fuß mitsamt Sandale gefühlvoll hin und her, sodass der Stein nach kurzer Zeit zurück zu den anderen auf den Weg fiel. Und dann ließ er seine Augen noch ein wenig auf eben diesem Stein ruhen. Er fand es seltsam, wie genau er ihn von den anderen Steinen unterscheiden konnte, obgleich er diesen fast gänzlich glich. Sein Stein hatte neben den anderen weißen, groben, kantigen Kieseln kein Merkmal, das einen Kontrast zu ihnen bilden könnte, und dennoch schien er sich auf eine ganz bestimmte Art vom Rest der Steine abzuheben. Irgendetwas war anders an ihm, und wenn nicht schon immer, was denkbar unwahrscheinlich war, dann zumindest seit gerade eben, seit Leno ihn vorhin aus der Sandale geschüttelt hatte. Und wenn er, der Stein, seither also nicht mehr der selbe war, dachte Leno, dann war es doch keinesfalls unwahrscheinlich, dass auch er selbst nicht unbeeinflusst geblieben war.
Er blickte wieder auf und kontrollierte den Weg. Um ihn herum hatte sich nichts verändert. Dann nahm er den Rucksack ab, legte ihn auf den Boden, ging hinunter in die Hocke, blickte wieder nach unten und sah sich den Kiesel nochmal genau an. Betrachtete die Konturen seiner Form, erfasste den exakten Ton seiner Weißfärbung und die Beschaffenheit seiner Oberfläche. Und er dachte daran, wie die Zeit den Kiesel genau so gestaltet hatte. Und wie er jetzt, gerade eben, wieder umgestaltet worden war, vielleicht etwas im Allgemeinen, in jedem Fall aber auch im ganz Besonderen für ihn.
„Kein Ding. Ich weiß es war keine Absicht. Ist schon vergessen.“, sagte er zum Kiesel, und spürte, wie sich gemächlich ein Grinsen in seine Gesichtszüge schnitt. Er hatte das tatsächlich laut ausgesprochen, und war dabei sehr zufrieden mit sich. Ein großer Spaß, die paar Worte an diesen Kiesel zu richten. Ein bisschen entgleist zu sein, sich ein bisschen neben der Norm bewegt, ein Tabu gebrochen zu haben. Und damit ließ er den Höhepunkt der transzendenten Episode dann erreicht und überschritten sein und der Griff der Konzentration konnte sich wieder lockern und ihm Raum geben für neue, diesseitigere Eindrücke.
\vspace{0.5em} \\
Die dominanteste der sogleich neu erwachenden Empfindungen war überraschenderweise eine sanfte Erschöpfung, die Leno vor der ungeplanten Pause niemals in den Sinn gekommen wäre, seine Glieder jetzt aber schwer werden ließ. Er entschied, dem Gefühl nachzugeben. Gleich ein paar Schritte weiter befand sich ein kleiner Hügel, der sauber und trocken genug aussah, um sich ein wenig auf ihm auszuruhen. Und er ging zu ihm rüber und setzte sich. Tatsächlich ließen ihn seine Füße, Waden und Oberschenkel jetzt spüren, dass er schon einige Zeit unterwegs war, und riefen ihm dabei gleichzeitig in Erinnerung, dass er heute noch nicht alles erledigt hatte, was zu erledigen war. Er stütze sich nach hinten mit den Händen ab. Ein kühler Luftzug drückte das T-Shirt an seinen Rücken, der wegen des Rucksacks komplett nassgeschwitzt war.\\
Wieder fühlte er, diesmal gründlich, in seinen Körper hinein und grübelte dabei darüber nach, ob der Fußmarsch das heutige Training beeinflussen würde, kam aber schnell zu dem Schluss, dass er sich deswegen keine Sorgen zu machen brauchte. Beim Schwimmen wurde vor allem der Oberkörper belastet und der befand sich in einem vollkommen erholten Zustand, darauf hatte er geachtet. Er entspannte sich. Ließ den Nacken nach vorne unten absinken und zog dabei langsame Kreise mit den Schultergelenken. Heute würde er mit vollem Einsatz schwimmen, ohne Rücksicht auf mögliche Überbelastung oder Muskelkater. Denn morgen würde er den ganzen Tag verreist sein, da gab es keine Gelegenheit zum Training. Den ganze Tag lang würde der Körper sich Regenerieren und Aufbauen können, und zu diesem Zweck wollte Leno ihm unbedingt einige Baustellen mitgeben. Er umschloss das Handgelenk des rechten Arms mit der linken Hand und zog den Arm weit zur linken Seite, mit dem Ziel die Muskulatur der Schulter in eine Dehnung zu bringen. In seinem Kopf konnte er sich bereits im Schwimmbecken sehen, sein Herzschlag im oberen Pulsbereich, er kurz vor der Wende. Zur rechten Seite tief einatmend schob sich sein Körper flach an der Oberfläche entlang, rollte anschließend widerstandslos durchs Wasser und stieß sich dann kräftig von der Beckenwand ab. Die Augen hatte er mittlerweile geschlossen. Er löste die Arme nun aus der Dehnung und führte sie über den Kopf, stellte sich vor, wie er nach der Rollwende zügig durchs Wasser glitt. Er merkte, wie unordentlich er die Arme über den Kopf hielt, und korrigierte die Haltung hastig. Die Hände legte er so aufeinander, dass sie sich genau an der Körperachse spiegelten, und streckte sie in die Länge, bis die Oberarme an den Ohren auflagen. Für einen Moment öffnete er die Augen wieder und hielt die Luft an. Dann nahm er die Arme herunter und griff nach seinem linken Handgelenk, um auch die andere Seite zu dehnen. Ein Windzug fuhr ihm über die Haare. Etwas fühlte sich beim Ziehen verklemmt an in der linken Schulter. Kein richtiger Schmerz, aber ganz richtig kam es ihm nicht vor. Er verstärkte langsam die Kraft mit der er zog und die Verklemmung schien sich zu lösen. In Gedanken schwamm er erneut auf die Beckenwand zu und vollführte die Rollwende ein weiteres Mal. Sah sich selbst aus verschiedensten Perspektiven, wie er die Schwimmer auf den anderen Bahnen noch unter Wasser überholte, und er, Leno, als Führender auftauchte. Er spürte wie sich ein Gefühl von Bestimmtheit, von Zielsicherheit in ihm ausbreitete, ja, vielleicht sogar von Euphorie. Und dann stand er auf, atmete tief durch und versuchte seine Einbildungen zumindest ein klein wenig einzuschränken. Er hatte sich von der Vorstellung allzu weit forttragen lassen. Es war ihm fast peinlich. Er war schließlich kein Profi.\\
Einmal noch hielt er kurz den Atem an, danach sammelte er alle notwendige Willenskraft um der Erschöpfung zu widerstehen und wuchtete seinen Körper nach oben. Die rechte Hand griff nach dem Träger des Rucksacks und schwang ihn mit einer fließenden Bewegung auf den Rücken. Dann setzte er seinen Weg fort. In seinem Rucksack wippten das Handtuch, eine Badehose, seine Schwimmbrille und eine Flasche Wasser leicht hin und her. Beim Gehen zog er sein Mobiltelefon aus der Hosentasche und drückte die Taste an dessen Stirnseite, wodurch das Display aufleuchtete. Dann las er die Uhrzeit vom Bildschirm ab. Es war zwar noch nicht so spät, dass er sich beeilen musste, aber Zeit lassen durfte er sich auch nicht mehr.
\chapter{Schwimmtraining}
Als Leno beim Schwimmbad ankam, war das vergitterte Tor vor dem Gelände bereits geschlossen. Langsam ging er die paar Meter rüber zum Kassiererhaus und klopfte am Fenster. Durch die Scheiben hindurch konnte er die Konturen eines Gesichts erkennen. Ein rundes Gesicht mit einer Brille auf der Nase und einem skeptischen Augenpaar dahinter, das ihn flüchtig musterte. Nachdem die Augen genug gesehen hatten und wieder verschwanden, drehte Leno sich um und ging den Weg zum Tor zurück. Und dann wartete er einige Momente lang und es passierte nichts, bis Nacar herauskam, wortlos den Schlüssel im Schloss umdrehte, ihn hereinließ und das Tor hinter ihm wieder schloss. Klick. Leno war froh, als die Interaktion damit beendet war. Er bedankte sich bei dem ihm bereits den Rücken zukehrenden grummeligen Kassier und machte ein paar schnelle Schritte auf dem gepflasterten Gehweg Richtung Kinderschaukel, von da an nahm er die Abkürzung über die Wiese zum Schwimmerbecken. Und dort setzte sich auf die Bank und ließ seinen Rucksack zwischen die Beine auf den Boden gleiten. Er gab sich etwas Zeit zum ankommen, tat nichts anderes als ein bisschen zuzusehen. Drei Schwimmer zogen ihre Bahnen im Wasser. Einer davon war Michael, der fast jeden Tag zu dieser Zeit hier schwamm, die anderen beiden waren Mike und Malte. Leno wartete, bis alle drei einmal auf Höhe seiner Bank vorbeigeschwommen waren. Dann nahm er das Wasser aus seinem Rucksack und trank einige Schlücke. Er schüttelte Arme und Schultern gut durch, wobei er aufmerksam auf seine rechte Schulter achtete. Sie hatte ihm in letzter Zeit manchmal Probleme gemacht, fühlte sich heute aber ganz gut an. Danach stand er auf, ging zu den Duschen und duschte sich warm ab. Während er zum Becken zurücklief, sah er einen vierten Schwimmer ins Wasser steigen, Magnus. Er musste kurz nach Leno angekommen sein, sodass sie sich vorher nicht begegnet waren. Magnus war häufig zur ungefähr gleichen Zeit wie Leno hier und legte dann meist mehrere Meilen während seines Trainings zurück. Immer auf der selben Bahn, stoisch, ruhig, immer mit der selben Geschwindigkeit. Früher, als sich Leno noch nach einer Bahn Kraulschwimmen keuchend am Beckenrand festkrallen musste, hatte er Magnus' Ausdauer auf geradezu abgöttische Weise bewundert. Wie er sich am Wasser entlang schob, nie zögernd, nie ermüdend, nie kraftlos. Heute, da er selbst die Technik gelernt hatte und sich zum gemütlichen Schwimmen auch nicht mehr allzu sehr anstrengen musste, bewunderte er viel mehr seine felsenfeste Beständigkeit, die hartnäckige Konstanz, die Magnus der Unberechenbarkeit des Lebens entgegenwarf. Er wirkte vollkommen unbeeinflussbar von seiner Umwelt, und gleichzeitig schien er sich bewusst dazu zu entscheiden, auch seine Innenwelt keinen Einfluss auf ihn nehmen zu lassen. Als wäre er der einzige Fixpunkt, während sich alles andere um ihn herum bewegte, unscharf war und verschwommen, sich immer wieder auflöste, umverteilte und neu gruppierte, und einzig er beständig im selben Zustand verharrte.
Mike dagegen, der gerade wieder an Lenos Bank vorbeizog, keuchend, strampelnd, wurde eindeutig durch alle möglichen inneren und äußeren Faktoren beeinflusst. Das war offensichtlich. Gerade eben war er gemächlich auf dem Rücken liegend an Leno vorbeigeschwommen, jetzt schoss er mit größter Anstrengung und Geschwindigkeit in die andere Richtung, dabei Wasser verspritzend wie ein defekter Rasensprenger. Obwohl es Leno eigentlich egal sein konnte, fragte er sich oft, ob Mike auf diese Art wirklich Freude am Schwimmen finden konnte. Während des Trainings stützte er sich oft lange Zeit nur mit den Unterarmen am Beckenrand ab und ruhte sich schwer atmend aus. Manchmal sah man ihn auch wochenlang überhaupt nicht im Bad, aber irgendwie schien er doch immer wieder zurückzufinden.
\vspace{0.5em} \\
Mittlerweile hatte Leno das Becken erreicht und ließ sich ins Wasser gleiten. Kalt war es. Es breitete sich eine Gänsehaut über seinen Armen und Schultern aus. Mit den Händen griff er den Beckenrand und zog sich an diesem entlang zu einer freien Bahn. Dort angekommen tauchte er seinen Kopf unter Wasser, hielt den Atem an und drückte seinen Körper für ein paar Sekunden unter die Wasseroberfläche. Sein Gesicht wurde kalt und nass und die Geräusche von außerhalb einen Moment gedämpft. Er war ganz still. Das Wasser stellte die Zeit vorübergehend ab. Er hörte seinem Herz ein paar Schläge beim Pumpen zu. Dann tauchte er langsam ausatmend wieder auf. Zwischenzeitlich schlug Mike rechts neben ihm am Rand an, hielt sich fest und schnappte nach Luft. Leno erkannte aus den Augenwinkeln, dass er zu ihm rüber sah, drehte sich aber nicht um. Er blickte weiter geradeaus die Bahn entlang, die er bald durchschwimmen würde. Es war eindeutig klar, dass er sich jetzt auf sein Training konzentrieren musste. Und er hätte auch nicht gewusst, was er mit Mike reden sollte. Nach ein paar Sätzen hätten sie beide sowieso weiter schwimmen müssen. Nein. Leno blickte weiter geradeaus nach vorn seine Bahn entlang und zeigte damit, dass er voll umfassend und allein auf sein Training konzentriert war. Das entsprach schließlich auch der Wahrheit. Und dann holte er Luft und hielt sie an, schob sich ein Stück nach oben, ließ sich von der Erdanziehungskraft unter die Wasseroberfläche drücken und drehte seinen Körper dabei in eine nach unten ausgerichtete, waagrechte Position. Einige Momente verharrte er schwerelos, die Beine zum Bauch angezogen, und wartete, bis sich das Wasser um ihn herum beruhigt hatte. Dann drückte er seine Füße mit einem explosionsartigen Krafteinsatz gegen die Wand des Beckens und katapultierte seinen Körper dadurch nach vorn. Sich von den Fingerspitzen bis zu den Zehen streckend spürte er das Wasser an sich vorbeiziehen. Der Strom rauschte in seinen Ohren. Er ließ sich Unterwasser gleiten, hielt die Augen geschlossen. Er glitt so lange, bis der Auftrieb zu stark wurde, er die Oberfläche durchstieß und mit dem Kraulen begann. Kontrolliert und kraftvoll führte er seine Arme durchs Wasser, stieß sich an ihm ab und schob sich nach vorn. Schwamm die Bahn hinunter. Atmete regelmäßig aus und ein. Gewissenhaft bemüht, die verbleibende Distanz zur anderen Seite korrekt einzuschätzen. Dann, als er sich im passenden Abstand zum Beckenrand wähnte, zog er noch einmal den linken Arm am Körper entlang zur Hüfte, wartete einen Augenblick, schob schließlich auch den Rechten nach hinten und drückte nun seinen Kopf nach unten, schwang seine Beine über seinen Körper nach vorne und rollte eine halbe Umdrehung vertikal durchs Wasser. Die Beine waren jetzt wieder zu seinem Bauch angezogen. Er brachte sie einen kurzen Moment in Vorspannung und stieß sie dann erneut mit ganzer Kraft in die Wand, um wieder zu der Seite zurück zu gleiten, von der er gekommen war. Sein Körper war von den Fingern bis zu den Zehenspitzen gestreckt. Das Wasser strömte an ihm vorbei. Er tauchte auf, zog seine Arme an seinem Körper entlang und atmete tief ein. Schwamm dorthin zurück, wo er gestartet war. Dann drehte er sich wieder, um erneut die Richtung zu wechseln. Er fügte sich ein in eine Routine, die fest in ihn einprogrammiert war. Er zählte die Bahnen und achtete dabei akribisch auf seine Technik. Er versuchte alles um ihn herum und in ihm auszublenden und nur zu Schwimmen. Das hatte er schon immer als schwierigste Aufgabe empfunden. Die Gedanken festzubinden an jeden Armzug und jeden Beinschlag und sie nicht irgendwohin abgleiten zu lassen. Denn wenn er nicht acht gab, würde er bald wieder an Gestern und Morgen denken und dabei geistlos im Wasser strampeln. Manchmal zählte er seine Armzüge. Gab seinem Geist damit eine Aufgabe, die ihn beschäftigt hielt. Manchmal sang er innerlich immer wieder die gleiche Liedstrophe während er schwamm. Während er seine Bahnen zog. Während er immer und immer wieder das selbe Wasser durchquerte. An dem er sich immer wieder abstieß und das ihn im selben Atemzug bremste. Es ist gleichzeitig Widerstand und Antrieb. Und es interessiert sich nicht für deine Stimmung, bleibt immer gleich. Du fühlst dich mal besser, mal schlechter, mal erschöpft und mal kräftig und durchquerst immer das selbe Wasser. Du schwimmst schnell oder langsam, auf dem Rücken oder auf der Brust, schwimmst lange Strecken oder kurze. Das Wasser, das dich umströmt, ist immer das selbe. An einem Tag ist es das selbe Wasser, und am nächsten Tag ist es wieder das selbe Wasser. Egal wie sehr du dich jemals anstrengen wirst, wie viel du investierst und wer du bist, es wird immer das selbe Wasser bleiben. Dachte Leno. Dann führte er seinen Körper zur Wende nach unten. Stieß sich von der Wand ab. Tauchte auf. Der erste Armzug auf der linken Seite. Der zweite Armzug auf der Rechten. Drei. Vier. Nach 27 weiteren Armzügen rollte er erneut durchs Wasser. Nach weiteren 32 befand er sich wieder am Ausgangspunkt.
\chapter{Der nächste Morgen}
Der nächste Morgen kündigte sich an, indem er unvermittelt und unendlich hell durch Lenos Fenster schien. Nach dem Schwimmen hatte Leno es gerade noch so geschafft seinen Körper bis nach Hause zu schleppen und war dort wie ein Stein in sein Bett gefallen. Jetzt, aus einem tiefen Schlaf gerissen, schlug er träge die Augen auf. Jeder Muskel seines Körpers schmerzte. Gut hatte er gestern trainiert. Jeden einzelnen Armzug hatte er kraftvoll und kontrolliert ausgeführt und heute durfte er das Resultat dieser Leistung in seinen Gliedern spüren. Dabei war ihm die Anstrengung nie unangenehm geworden. Es war ihm leicht gefallen. Er hatte es genossen sich anzutreiben und an seine Grenze zu gehen. Und jetzt genoss er es, das Ergebnis dieser Bemühungen fühlen zu können. Sein eben aufgewachter Körper lag entkräftet auf der Matratze. Sanft wurde er von der Erdanziehungskraft auf sie gedrückt. Ein paar Sonnenstrahlen fielen aus der Stratosphäre durch den Himmel hindurch, auf die Erde und durch sein Fenster, auf sein Gesicht und seinen Hals. Er blieb noch etwas liegen, drehte sich zur Seite, schob die Füße über den Rand des Betts und richtete sich dann unter großer Anstrengung auf. Seine Arme und Beine waren vom festen Schlaf noch ganz unbeweglich. Und er rieb sich einige Male mit den Händen über das Gesicht. Drehte die Fingerknöchel gegen die Augenlieder. Dann raffte er sich auf und tapste in die Küche, befüllte den Wasserkocher mit Wasser und schaltete ihn an. Er antwortete mit einem leisen Summen. Seit einigen Wochen gab der Wasserkocher diesen Ton von sich, während er das Wasser erwärmte, ein gleichmäßiges, fast befangenes Summen. Ansonsten funktionierte er aber noch gut und es war auch nicht weiter störend. Leno trat ans Fenster. Draußen, auf der Straße vor dem Haus, hupte ein Auto. Und auch sonst war es laut und geschäftig. Einige Vögeln zwitscherten eifrig, begrüßten den anbrechenden Tag, und zusammen mit ein paar Kindern, die schon so früh am Morgen auf ein Garagentor Fußball spielten, ergab sich eine durchaus ansehnliche Geräuschkulisse. Beim Einschlag das Balls im Metall krachte es ohrenbetäubend. Der Klang der Hupe, das blecherne Scheppern und das Summen des Wasserkochers vermischten sich wie eine krumme Symphonie mit dem Vogelgesang. Und dann kam dazu noch abgedämpft Musik aus der angrenzenden Wohnung in die Küche gedrungen. Wie ein willkürliches Ensemble aus Streichern, Bläsern, Vokalisten und Trommlern, ohne Dirigent, zufällig und ungeleitet. Und Leno nahm einen Teelöffel aus einer der Schubladen, schraubte die Dose mit Kaffeepulver auf und füllte einen Löffel in eine seiner Tassen. Er stützte die Hände auf die Arbeitsplatte und hörte zu. Dem Vögeln beim Singen, dem Rufen der Kinder, dem Wasserkocher beim Summen. Und während er wartete, nickte er mit dem Kopf im Takt der Musik mit, versuchte sich zu konzentrieren und seinen Tag vor sich auszubreiten. Nach dem Kaffee würde er sich gleich etwas kochen und später war er mit Anthony verabredet. Sie hatten ausgemacht, sich am Bahnhof zu treffen, gemeinsam mit dem Zug in die Stadt zu fahren und dort vielleicht ein paar Klamotten zu kaufen. Zu Fuß lag der Bahnhof eine gute halbe Stunde von Lenos Wohnung entfernt. Er würde also aufpassen müssen, nicht zu spät loszugehen, damit sie ihn nicht verpassten.
\vspace{0.5em} \\
Schließlich nahm er die Hände wieder von der Arbeitsplatte und hörte auf mit dem Nachdenken, drehte sich um und holte sein auf dem Bett liegendes Mobiltelefon. Ein leichter Druck auf die kleine Taste an der Stirnseite lässt das Display aufleuchten. Keine neuen Nachrichten. Das Summen des Wasserkochers war inzwischen zu einer beachtlichen Lautstärke angeschwollen. Und Leno zog den einzigen Stuhl des kleinen Küchentischs unter diesem hervor, setzte sich parallel zur Tischplatte, genau so, dass er den linken Unterarm auf dieser ablegen konnte, und wischte mit seinem Daumen über die Oberfläche des Mobilgeräts. Ein interessanter Artikel leuchtete bläulich auf dem Display auf. Und Lenos Blick blieb an den Bildschirm geheftet, bis der Wasserkocher laut klickte, das Wasser eifrig blubberte und das Summen aufhörte. Dann erst stand er auf, stemmte sich mit Armen und Beinen hoch. Er goss das heiße Wasser in die Tasse. Rührte ein paar mal um. Beim Aufgießen bildete sich eine kleine Schicht Schaum darauf. Natürlich war er noch viel zu heiß zum Trinken, aber trotzdem konnte er es nicht lassen, immer wieder kleine Schlücke davon zu nehmen. Genau so, dass er sich gerade nicht, oder gerade eben doch, die Lippen verbrannte. Bis die Flüssigkeit kühler und kühler wurde und die Schlücke gieriger werden konnten. Währenddessen konsumierte er fleißig die Anzeige des Mobiltelefons, und als der Kaffee leer war, las er noch die letzten Zeilen vom glimmenden Bildschirm ab, stand dann auf und ließ heißes Wasser ins Spülbecken ein. Gleichzeitig setzte er den Topf für die Nudeln auf, die es als verfrühtes Mittagessen geben sollte. Er war zwar niemand, der jede einzelne Kalorie zählte, aber dass er nach einer anstrengenden Trainingseinheit viel Energie zu sich nehmen sollte, das wusste er. Und da aß er gerne auch schon Vormittags unverhältnismäßige Mengen an Nudeln. Mittlerweile bewegte er sich beim Abspülen lässig zu der nach wie vor von irgendwoher kommenden Musik und gefiel sich dabei. Sie schien lauter gedreht worden zu sein. Als würde ihn irgendjemand geradezu dazu einladen wollen, den Tag lieb zu gewinnen. Und er war gerne bereit, diese Einladung anzunehmen. Immer wieder einen Blick auf den Topf werfend schrubbte er Pfannen und Teller, Messer, Gabeln und Löffel. Und bis er fertig war hatten sich bereits kleine Bläschen am Boden des Nudelwassers gebildet. Er salzte es, setzte er sich nochmal an den Tisch, suchte ein Video auf YouTube, blickte nach unten, nach unbestimmter Zeit wieder auf, riss die Verpackung auf und schüttete die Nudeln ins Wasser. Dann senkte er seinen Blick wieder, ließ die Nudeln ein klein wenig verkochen, servierte sie sich mit fertig gekaufter Tomatensoße und schloss während des Essens das Video, da er volle Konzentration benötigte, um auch noch die letzte Nudel in seinen eigentlich schon komplett vollen Magen zu drücken. Als er das geschafft war, kontrollierte er nochmal die Zeit. Eine gute Stunde hatte er noch, bis er sich auf den Weg machen müsste.
\chapter{Anthony}
Er ging viel zu früh los. Leno ging gern zu Fuß und legte alle Strecken, bei denen es möglich war, auch so zurück. Und nach seinem Maßstab war es bei vielen Strecken möglich, sie zu Fuß zu gehen. Zum Schwimmbad ging er immer zu Fuß. Wenn er in die Stadt wollte, fuhr er nicht mit dem Bus, sondern lief. Sogar die Strecke zu seiner Arbeit legte er zu Fuß zurück. Es war ihm einfach irgendwann aufgefallen, dass er nach Spaziergängen immer recht gut gelaunt war, und seit eben jener Erkenntnis war er nicht mehr dazu bereit, sich diese gute Laune nehmen zu lassen. Dabei war er schon immer ein sehr langsamer Fußgänger gewesen, was dazu führte, dass er ständig von anderen überholt wurde. Er hatte das noch nie verstehen können. Seiner Ansicht nach machte er nichts anders als die übrigen Gehenden. Er setzte einen Fuß vor den zweiten. Abwechselnd den Linken und Rechten. Nacheinander. Er hatte keine besonders kurzen Beine und könnte durchaus als sportlich beschrieben werden, aber sobald er versuchte sein Gehtempo zu steigern, fühlte es sich unnatürlich an, sogar komisch. So komisch, dass er nicht verstehen konnte, wie andere in diesem Laufschritt tagein, tagaus umhereilen konnten. Nicht, dass er nicht an die Möglichkeit glaubte, gerade diese Rennerei könnte sich für jemand anderes natürlich anfühlen, nein. Das konnte er schon glauben. Verstehen aber, konnte er es nicht.
Daran ändern konnte er aber auch nichts, und er hatte sich längst damit abgefunden ein langsamer Fußgänger zu sein. Mit der Zeit hatte er sich immer besser darauf verstanden, aus dieser Eigenart eine Tugend zu machen. Er meinte, durch seinen langsamen Gang ruhig zu wirken und dadurch einen Gegenpol zur allgemeinen Gehetztheit zu bilden. Und er glaubte, so ein bisschen bewusster zu leben, ein bisschen besser. Die Wahrheit aber war und blieb, endgültig und unumstößlich, dass er einfach nur sehr langsam seine Füße voreinander setzte. In jedem Fall lag Leno viel daran, sein Schritttempo nicht unnötig erhöhen zu müssen.
\vspace{0.5em} \\
Darüber hatte er nachgedacht, während er den Weg bis zur Kreuzung gegangen, dort rechts abgebogen, dann an der Kirchenmauer entlang geschlendert war, anschließend um die Ecke schritt, die langgezogene Straße zum Bahnhof hinunter trabte und eine viertel Stunde zu früh am Gleis eintraf. Aber das machte nichts und jetzt saß er auf einer der Bänke und überbrückte die zu wartende Zeit mit surfen auf seinem Mobiltelefon. Das bunte Leuchten des Displays sog ihn so tief in seine digitale Welt ein, dass er seine Umgebung erst wieder wahrnahm, als ihn eine Hand an der Backe streichelte, und er schreckte auf und drehte sich überrascht um.
\vspace{0.5em} \\
„Hallo, du wunderschönes Geschöpf!“, grüßte Anthony.
\vspace{0.5em} \\
„Hi, Anthony.“, grüßte Leno zurück.
\vspace{0.5em} \\
Anthony grinste, umkurvte die Sitzbank und nahm neben Leno Platz. Seine knochigen Finger griffen die Rückenlehne und stabilisierten den schmalen Körper. Er trug eine enge schwarze Hose, rote Sportschuhe und ein viel zu großes weißes T-Shirt. Er setzte sich, drehte er den hageren Körper leicht nach links Richtung Leno und griff in eine seiner hinteren Hosentaschen, um eine Schachtel Zigaretten und ein Feuerzeug hervorzukramen. Beides legte er neben sich auf der Bank ab bevor er sich Leno zuwandte:
\vspace{0.5em} \\
„Und Bruder? Wie geht's dir? Alles klar? hm, sag?“
\vspace{0.5em} \\
Leno kannte Anthony seit der Schulzeit. Aus zufälligen Sitznachbarn waren vor Jahren über viele ereignislose Mathestunden und turbulente Pausen hinweg Freunde geworden. Sie hatte unzählige Nachmittage, Abende und Nächte miteinander verbracht, hatten gemeinsam die ersten und damit einzig bedeutsamen Erfahrungen mit Alkohol und anderen Drogen gemacht. Hatten nebeneinander schreitend jeden zu begehenden Fehler begangen. Hatten ihre jugendlichen Probleme und Dummheiten miteinander geteilt. Sie wussten um jede zurückliegende Blamage des anderen. Sie kannten sich aus Zeiten, in denen sie ihrem Naturell ausgesetzt gewesen waren und das Fundament ihrer Freundschaft bestand nicht aus gegenseitiger Zuneigung, es war ein Konstrukt aus zurückliegenden Ereignissen und Erfahrungen. Die beiden mussten sich nichts mehr recht machen oder beweisen. Es war eine Tatsache, dass sie den Rest des Lebens in irgend einer Form nebeneinander beschreiten würden, und so schritten sie, weil es für beide von Zeit zu Zeit angenehm war, auch miteinander. So hatten beide immer noch einen Fuß in der Tür der Vergangenheit. Als könnte man, wenn man unzufrieden wäre mit dem Verlauf der Dinge, einfach diese Tür wieder aufstoßen. Und zurückkehren zu einem Zeitpunkt, an dem man noch unbeschriftet war und neu und voller Potential, zu einem Zeitpunkt an dem man noch formbar war. Unbelastet und unverbraucht.
\vspace{0.5em} \\
„Bei mir ist alles klar. Freut mich voll, dass es heute geklappt hat. Und bei dir?“
\vspace{0.5em} \\
„Ja mir gehts prächtig!“
\vspace{0.5em} \\
Anthony griff jetzt wieder nach der Packung Zigaretten und spielte einen Moment mit ihr herum, dann öffnete er sie und bot Leno eine an:
„Willst du eine? Sag?“
\vspace{0.5em} \\
„Ne, danke.“
\vspace{0.5em} \\
„Besser so.“, lachte Anthony, „Für mich wird's glaub ich auch bald Zeit, aufzuhören. Ich rauche eh schon viel weniger. In der Arbeit komm ich gar nicht dazu. Viel zu viel zu tun. Gestern war wieder auf einmal Mittagspause, und dann hab ich erst gemerkt, dass ich den ganzen Vormittag gar nicht geraucht hab.“
\vspace{0.5em} \\
„Ist doch gut, wenn das von alleine weniger wird. Weißt du, wer gerade auch aufhören will?“, sagte Leno.
\vspace{0.5em} \\
„Ne, hm, sag.“
\vspace{0.5em} \\
„John“
\vspace{0.5em} \\
Anthony lachte laut auf: „John? Mit dem Rauchen komplett oder will er weiter kiffen? Hm? Sag? Oder will er nur mit dem Kiffen aufhören?“
\vspace{0.5em} \\
„Ich schätz mal komplett alles“, antwortete Leno.
\vspace{0.5em} \\
„John.", Anthony lachte wieder: "Der hat immer gesagt er scheißt drauf. Immer schon morgens zwei, drei Kippen reingezogen und gehustet dabei und dann gemeint, dass man sich von allem dem Tod holen kann und die Raucherei im Endeffekt scheißegal ist.“
\vspace{0.5em} \\
Leno schaute kurz zu Boden und scharrte mit seinem Schuh etwas auf dem Pflaster.
„Ja mein Gott sowas sagt man halt“, entgegnete er dann trocken.
\vspace{0.5em} \\
Und dann fuhr die Bahn am Gleis ein, und die beiden erhoben sich von der Bank und suchten sich einen Platz in einem der Abteile. Die Unterhaltung floss weiter dahin. Sie erreichten wichtige Themen nicht weniger oder mehr energisch als belanglose, und als sie angekommen waren nahmen sich die Freunde beim Bahnhof etwas zu essen mit und fuhren mit dem Bus in die Innenstadt. Dort steuerte Anthony zielstrebig auf eines der vielen Modegeschäfte zu. Leno folgte ihm leidenschaftslos nach. 
\chapter{Einkaufen}
An der Stange sah jedes Kleidungsstück sehr schick aus. An Leno eher wie ein schlecht transplantierter Fremdkörper. Er stand vor einem Spiegel und musterte seine Reflexion darin skeptisch. Die Kombination aus Hemd und Cardigan, deren Abbildung auf dem Werbeplakat ihm gut gefallen hatte, weigerte sich jetzt, hier, in der Umkleidekabine, ihm zu passen. Er zupfte am Cardigan herum, versuchte, ein paar der Falten zu glätten, den Stoff besser auszurichten und ihn eleganter an seinen Körper zu legen, aber die Bemühungen blieben erfolglos. Egal wie er sich drehte, aus welchem Winkel er sich auch betrachtete, die Klamotten wollten an ihm einfach nicht funktionieren. Schließlich gab er auf, streifte Hemd und Cardigan von seinem Körper ab und zog sich sein altes T-Shirt wieder an, das sich sofort fügsam an die Form seines Oberkörpers schmiegte. Dann verließ er die Umkleidekabine, und beim Durchschreiten des Vorhangs warf ihm sein Spiegelbild einen letzten urteilenden Blick nach. Und dann ging er zu dem Regal zurück, von dem er die Sachen genommen hatte, faltete sie so gut er konnte zusammen und legte sie weg. Das Model, das auf dem Plakat über ihm eben dieses Hemd und diesen Cardigan präsentierte, sah ihn dabei aus stechend blauen Augen so durchdringend lasziv an, als würde es ihn jeden Moment gegen die Wand drücken und ihm die Zunge in den Mund rammen. Aber es war nur ein Plakat. Und er stand in den geschäftigen Räumen des Modehauses herum, ohne Plan und Auftrag, verloren zwischen Preisschildern und Kleiderbügeln. Unentschlossen wanderte sein Blick hin und her, rutschte an allem ab, fand nirgends Halt.
\vspace{0.5em} \\
Die Verkäufer schienen ausnahmslos in Klamotten aus dem Bestand gekleidet zu sein. Und sie sahen gut darin aus. Sie waren modern frisiert, hatten scharf geschnittene Gesichter, und überhaupt war ganz offensichtlich davon auszugehen, dass sie sehr allgemein etwas besser waren als Leno. Unschlüssig schob er seine Füße auf dem Boden umher, verlagerte sein Gewicht vom rechten auf das linke Bein. So als könnte er die nächste handvoll Sekunden einfach unter seinen Schuhen zerdrücken. Dann zog er sein Mobiltelefon aus der Hosentasche um nachzusehen, wie spät es war. Als er das Gerät in der Hand hielt, setzte ein Automatismus ein, und seine Finger verbanden einige der auf dem Display angezeigten Punkte zu einem Netz. Es entsperrte, der Bildschirm leuchtete auf und Leno öffnete ein Video auf YouTube. Augenblicklich wurde ihm bewusst, wie losgelöst von der Situation sein Handeln war, und er schaltete das Gerät wieder aus. Das Display erlosch, wurde dunkel, und verschwand in seiner Tasche. Und die Uhrzeit war vergessen. Als er es erneut hervorholte mahnte er sich zur Aufmerksamkeit. Es war 4 Uhr. Leider würde es noch etwas dauern, bis es Zeit war zu Abend zu essen.
Eine ganze Weile war nun schon vergangen, während er immer noch an der selben Stelle vor dem Plakat herumstand. Zu lange eigentlich für einen Einkaufenden, fand Leno. Es wurde ihm unangenehm, und so fing er an in irgendeine Richtung loszugehen.
Endlich verfing sich sein Blick in einer Ecke, in der verschiedene Hosen auslagen. Eine neue Hose konnte Leno brauchen. Die beiden Jeans, die er am liebsten hatte und zu jeder Gelegenheit anzog, waren schon arg abgetragen. Er sah an sich hinunter. An seinen Hosentaschen zeichneten sich hell die Durchdrücke von seinem Schlüssel und seinem Mobiltelefon ab. Genau an den Stellen, an denen er sie seit Jahren verstaute und wo sie tagtäglich bei jeder Bewegung am Stoff rieben. Die Enden der Hosenbeine hatten ebenfalls schon zu oft und lange über den Boden reiben müssen und fransten mittlerweile stark aus. Leno setzte sich also in Bewegung, ging vorbei an Damenwäsche und Winterjacken, hin zur gegenüberliegenden Seite des Geschäfts in Richtung einer neuen Hose. Dabei warf er immer, wenn er einen Gang passierte einen Blick zwischen die Regale, in der Hoffnung, Anthony könnte irgendwo inmitten des Gewusels auftauchen. Gang für Gang für Gang. Und jeder Gang in den er blickte, schien wie eine eigene kleine Welt zu sein. Jeder hatte seine eigene Dynamik, sein eigenes Innenleben. Wie ein in dünne Scheiben geschnittenes Universum. In einem Gang sah er einen älterer Herr mit einem angestrengt neutral dreinblickendem Gesicht, neben ihm sauber aufgereiht eine Bataillon schwarzer Schuhe. Seine Frau wuselte gerade mit einem neuen Paar auf ihn zu und der Mann hielt den langen, hölzernen Schuhlöffel so fest am Griff, dass Leno nicht überrascht gewesen wäre, wenn er ihr im nächsten Moment die Schuhe damit aus der Hand schlagen würde. Aber da war er schon weiter. Im nächsten Gang saß, fast direkt vor Leno, nur etwas weiter als eine Armlänge entfernt, saß ein kleines Kind auf einem Stuhl. Vor ihm kniete eine junge Frau. Der Kopf des Kindes war weggedreht, als würde es irgendeinen entfernten Punkt an der Decke auf der anderen Seite des Geschäfts fixieren. Dabei hatte es die Knie ganz durchgestreckt und reckte der Frau energisch die Füße entgegen. Diese versuchte mit einigem Krafteinsatz, dem Kind einen Schuh über den rechten Fuß zu stülpen. Und genau in dem Moment, in dem Leno in eben diesen einen Gang stierte, drehte sich das Gesicht des Kindes in seine Richtung und strahlte ihn an. Es war ihm die pure Glückseligkeit ins Gesicht geschrieben, während seine Mutter konzentriert an seinen Füßen herumwerkelte. Ihre feucht benetzte Zungenspitze lugte dabei ein kleines Stück aus ihrem Mundwinkel.
\vspace{0.5em} \\
Und tatsächlich erblickte Leno irgendwann ein paar Gänge weiter Anthony vor einem Spiegel stehen. Auch er probierte Schuhe an, um ihn herum lagen mehrere geöffnete Kartons. Ein Verkäufer stand bei ihm und hielt zwei weitere Verpackungen in der Hand. Anthony drehte gerade die weißen Sneakers an seinen Füßen vor einem Spiegel auf verschiedene Seiten um sie zu betrachten. Dann erkannte er Leno im Spiegel, grinste, formte ein V mit seinem Zeige-, und Mittelfinger und grüßte ihn damit geräuschlos. Leno seinerseits grüßte gleichsam grinsend zurück.
Und dann schob sich schon wieder das nächste Regal in Lenos Blickfeld. Er war während seiner Beobachtungen zwar sehr langsam geworden, aber doch immer weiter gelaufen. Mittlerweile musste er seinen Körper weit zurücklehnen und seinen Kopf in einem unnatürlichen Winkel drehen, um zwischen den Regalen einen letzten Blick auf Anthony zu erhaschen. Plötzlich spürte er einen Kontakt an seinem rechten Knie. Er blieb sofort stehen. Sein Kopf schnellte ruckartig zurück und lenkte sich in Richtung der Berührung. Ein Gefühl des Schuldbewusstseins schoss ihm in die Magengrube. Er hatte in seiner Unachtsamkeit einen kleinen Jungen gestreift. Entschuldigend hob er die Hand:
„Sorry!“.
Aber der Junge erwiderte nichts, sah ihn nur verstört an und drehte sich dann wieder weg. Glücklicherweise hatte er ihn nur leicht berührt. Er schaute dem Jungen nach, wie er zielstrebig wegstapfte. An der Mütze auf seinem Kopf hing noch das Etikett. Dann wandte er sich wieder seinem Ziel zu, der Ecke mit den Hosen, und schlenderte betont gelassen weiter. Noch zweimal musste er unterwegs in einer kleinen Kurve anderen Kunden ausweichen. In der Tat war Leno der Ansicht, dass das Geschäft regelrecht überfüllt war. Und wie er feststellen musste, waren alle Besucher des Geschäfts genauso stilsicher gekleidet wie dessen Angestellte. Jeder Kunde schien entweder Model oder Modedesigner zu sein und hier sein aktuelles Outfit zu präsentieren. Er fragte sich, zu welchem Zweck diese Leute überhaupt einkaufen waren, wo sie doch offensichtlich schon genug schöne Klamotten hatten. Und dann erreichte er sein Ziel und stand vor einem Regal mit mehreren Türmen aus verschiedenen Hosen. Sofort begann er, seine Finger an einem Stapel Chinos entlanglaufen zu lassen, ihre Etiketten herauszufummeln und ihre Größen zu kontrollieren.
\vspace{0.5em} \\
„Entschuldigung, kann ich ihnen helfen?“. Unbemerkt hatte sich ein Verkäufer angeschlichen.
\vspace{0.5em} \\
„Nein, Danke, ich schau mich nur um.“, erwiderte Leno, und drehte im selben Moment das Etikett einer Hose so, dass er ihre Größe lesen konnte. 36. Er sprang eine Hose tiefer. Der Verkäufer lächelte ihn an, bog im rechten Winkel ab und entfernte sich wieder von ihm. Lenos Finger zogen derweil schon am Bändchen des nächsten Etiketts. Es hatte sich irgendwie verhakt und wollte sich nicht vom Stoff der Hose lösen. Er musste die zweite Hand zu Hilfe nehmen, hob den Stapel an der richtigen Stelle an, entklemmte dadurch die Hose und gab das Etikett frei. Es schnalzte heraus. Größe 36. Es ärgerte ihn, dass er sich nicht schicker angezogen hatte. Nicht, das irgendeine Kombination der Sachen aus seinem Kleiderschrank mit denen der anderen hätte mithalten können. Aber er gab einfach gern sein Bestes. Und dieses Ziel hatte er heute deutlich verfehlt. Seine Finger waren jetzt fast am Ende des Turms angekommen. Dann, endlich, war eines der Etiketten mit der richtigen Größe beschriftet. Er packte die dazugehörige Hose, verschwand in der nächsten Umkleidekabine und schloss den Vorhang. Mühsam legte er seine alte Jeans ab. Dann stieg er in die ausgewählte Hose, zog den Hosenbund nach oben und hob den Kopf. Es präsentierte sich ihm sein Spiegelbild, mit dem bekannten T-Shirt, den bekannten Haaren, dem bekannten Gesicht und der neuen Chino. Und wieder schien es einfach nicht zu passen, kein rundes Bild zu ergeben. Und Leno wusste nicht mehr, ob er der Hose die Schuld dafür geben konnte. Anscheinend war er derjenige, der hierher transplantiert worden war. Er war der Fremdkörper an den Kleidungsstücken. Nicht andersrum. Er schlüpfte zurück in seine alte Jeans, die ihm passte, in der er natürlich aussah. In der er aussah, wie er selbst.
\vspace{0.5em} \\
Und genau das ärgerte ihn. Er wollte nicht aussehen wie er selbst. Er wollte nicht festgelegt sein auf ein bestimmtes Äußeres, auf eine definitive Erscheinung und auf einen expliziten Eindruck. Er war eben dafür hier, diese Dinge zu ändern. Und allein diese Tatsache musste schon jede Festgelegtheit widerlegen. Er konnte jede Hose tragen, vorausgesetzt sie saß gut und gefiel ihm. Also zwar nicht diese Hose, in jedem Fall aber eine andere. Es war schließlich nur das. Eine Hose. Er entledigte sich also der Chino und zog sich seine alte Jeans wieder an. Dann verließ er die Umkleidekabine, um eine andere Hose auszusuchen. Plötzlich kam Anthony um die Ecke, er hatte einen Schuhkarton unter den Arm geklemmt. Als er Leno sah, weiteten sich seine Augen überrascht.
\vspace{0.5em} \\
„Ah, da bist du ja. Hab dich nicht gefunden und schonmal bezahlt. Hast du Lust dann schon mal was Essen zu gehen? Sag? Ich hab ziemlich Hunger.“
\vspace{0.5em} \\
Und Leno freute sich sehr und meinte, dass er irgendwie heute sowieso nichts finden würde. Sie könnten gerne schon Essen gehen, er müsse nur vorher noch kurz auf die Toilette. Anthony ging in der Zwischenzeit bereits nach oben und setzte sich ins Café. Leno folgte kurze Zeit später nach.\\

\chapter{Café am Dach}
Auf dem Dach angekommen fand Leno Anthony bereits mit einem Glas Wein, einer Flasche Wasser und einer Tasse Kaffee am Tisch sitzen. Er kam dazu und machte die witzigste Anmerkung, die ihm auf dem Weg zu seinem Freund über dessen Getränkeauswahl eingefallen war. Anthony belohnte sein Bemühen mit einem authentischen Lachen. Sein Blick hing gerade über der Speisekarte.
\vspace{0.5em} \\
„Willst du was Essen? hm? Sag?“, fragte er Leno.
\vspace{0.5em} \\
Leno hatte nur halb zugehört, weil er im Moment den Hauptteil seiner Aufmerksamkeit dafür brauchte, mit der Hand einen Kellner an ihren Tisch zu winken. Als sein Gestikulieren mit einem Kopfnicken beantwortet wurde, blickte er wieder zu seinem Freund. Dieser schaute ihm immer noch mitten in die Augen und wiederholte:
„Sag, Leno, willst du hier was Essen?“
\vspace{0.5em} \\
„Lass uns doch erstmal in Ruhe was Trinken.“, entgegnete Leno, dem aufrichtiges Überlegen im Moment noch zu anstrengend war.
\vspace{0.5em} \\
„Es gibt Sandwiches mit Feige.“ meinte Anthony, und die Freude über diese Besonderheit auf der Karte spiegelte sich deutlich in seiner Stimme und seinem Gesicht wieder. Dabei konnte Leno sich immer noch nicht ganz auf seinen Freund konzentrieren. Während Anthony seinen Überzeugungsversuch vorgebracht hatte, war der Kellner bereits an ihren Tisch getreten und stand nun erwartungsvoll neben Leno. Er deutete über den Tisch hinweg zu den Getränken Anthonys:
„Ich bekomme bitte einmal das Selbe.“
\vspace{0.5em} \\
„Einmal Rotwein, einmal Kaffee, einmal Wasser.“, zitierte der Kellner, während er die Bestellung notierte. Dabei brach ihm die Bleistiftspitze ab. Er wischte sie ohne darüber nachzudenken mit dem Handrücken von seinem Notizblock und schrieb unbeirrt mit der Kante der abgebrochenen Spitze weiter. Währenddessen nahm Anthony einen großen Schluck aus seinem Kaffee.
Leno beobachtete ihn dabei mit einem Schmunzeln. Er genoss es, sich in Unbefangenheit Anthonys treiben zu lassen. Anthony hatte sich schon immer für genau das entschieden, was er im Moment wollte. Ohne zu zögern und zu zweifeln. Und jetzt, als er hier mit ihm saß und die Sonne des Nachmittags die am Tisch stehenden Gläser durchstrahlte, als ein angenehm kräftiger Wind über das Dachcafé zog und die Haare der Gäste sowie die Blätter der Topfpflanzen tanzen ließ, als das durch die Gläser scheinende Licht Muster auf das Holz des Tisches warf. Da konnte es keine andere Möglichkeit geben, als ihm in genau dieser Art des Handelns Recht zu geben.
\vspace{0.5em} \\
„Ich kann das jetzt nicht nicht bestellen, Leno. Hm. Da würde ich heute nicht mehr glücklich werden, wenn ich jetzt nicht das Feigensandwich probieren würde.“, lachte Anthony und nahm einen großen Schluck aus seinem Weinglas. „Du kannst ja auch nichts Essen, aber ich muss das jetzt haben.“, und er hob die Hand und Leno war es etwas unangenehm, den Kellner schon wieder zu ihrem Tisch zu bitten, wo er doch gerade erst hier gewesen war. Er kannte sich selbst allerdings auch gut genug um zu wissen, dass er keine andere Wahl hatte, als ebenfalls ein Sandwich zu bestellen, um zu verhindern, später auf seinen Freund neidisch zu sein. Und es war auch notwendig, dass Leno sich so schnell entschied, denn der Kellner hatte Anthony schon bemerkt und schob nur zügig eine Tasse Kaffee auf ein Tablett, bevor er zu ihrem Tisch kam.
\vspace{0.5em} \\
Er sagte: „Hier schon einmal der Kaffee, Wasser und Rotwein kommen gleich. Darf's sonst noch etwas sein?“
\vspace{0.5em} \\
„Ja, wir nehmen noch zweimal das Sandwich Fico bitte.“, kam Leno seinem Freund zuvor.
\vspace{0.5em} \\
„Und von dem Rotwein, da nehmen wir noch eine Flasche.“, fügte Anthony hinzu, der sein Glas anscheinend seit dem ersten Schluck nicht mehr aus der Hand gegeben hatte, und nun nur noch einen kleinen Rest darin umherschwenkte.
\vspace{0.5em} \\
„Dann anstatt dem zweiten Glas eine Flasche.“, wiederholte der Kellner, und strich mit einem perfekt angespitzten Bleistift über seinen Notizblock.
\vspace{0.5em} \\
„Nein, bitte schon das Glas und zusätzlich noch eine Flasche.“, korrigierte Anthony den Kellner sachlich. Leno hob in der Zwischenzeit seinen Kaffee vor den Mund und pustete ein bisschen. Als der Kellner weg war, fragte er Anthony halb scherzend, halb ernsthaft vorwurfsvoll, wer denn den Wein seiner Vorstellung nach trinken sollte. Aber obwohl Anthony ihm versicherte, er würde den Wein auch alleine trinken und dass ihn zu teilen nur ein unverbindliches Angebot war, hatte Leno sich eigentlich schon lange mit seinem Schicksal angefreundet. Er hatte sich die letzten Wochen sehr auf den Sport konzentriert und war lange nicht mehr richtig ungezwungen gewesen. Das Training morgen einmal ausfallen zu lassen, würde kein Beinbruch sein. Und als der Kellner das Glas und die Flasche Wein brachte, da trank er genau so gierig wie sein Freund. Die Sandwiches brauchten sehr lange, aber es war egal und da die Flasche schon fast leer war bestellten sie gleich noch eine. Und weil sie durch den Wein sehr hungrig wurden und das Essen sehr gut war bestellten sie nochmal jeder ein Sandwich und dazu noch eine Falsche Wein. Und schon bald war es fest beschlossen, dass der Abend nicht auf dieser Terrasse enden konnte und sie noch weiter in die Innenstadt ziehen würden. Sie bezahlten, gaben gutes Trinkgeld, waren laut und spaßten ein wenig mit dem Kellner und machten sich dann auf in die Nacht. Anthony kannte sich gut aus im Nachtleben - oder war zumindest überzeugt davon - und das ersparte den beiden eine Diskussion darüber, wo ihr nächstes Ziel sich befand. Leno lief einfach seinem Freund nach.
\chapter{Club}
Als Leno und Anthony beim Club ankamen, erstreckte sich eine lange Warteschlange vor dessen Eingang. Wortlos fädelten sich die beiden ein, und sie entschleunigten dabei abrupt. Gerade hatten sie sich noch zielstrebig durch die Stadt geschoben, jetzt bildeten sie still ein neues Glied in der Kette der Wartenden, die in kleinen Gruppen sauber hintereinander aufgereiht vor dem Türsteher standen und sich geschäftig anschwiegen. Aus dem Inneren des Clubs drang ein dumpfer Bass, der die Szene untermalte. Etwas Abseits drückten ein paar Rauchende die Überreste ihrer Zigaretten mit den Fußspitzen in das Kopfsteinpflaster. Sie benahmen sich laut und derb, und einer von ihnen, der mindestens zwei Köpfe größer zu sein schien als alle Anderen, erzählte etwas in den aus ihren Körpern gebildeten Kreis hinein und gestikulierte dabei aufgeregt mit seinen Gliedern. Und auch wenn niemand es offensichtlich machen wollte, war eindeutig klar, dass die Aufmerksamkeit Aller auf ihn gerichtet war. Wie Peitschen schnalzte er seine Arme von oben nach unten, von der einen zur anderen Seite seines Körpers und ließ synchron dazu seine dröhnende Stimme die Straße beschallen. Seine Zuschauer stärkten ihm dabei durch Zurufe den Rücken, und bei jeder Pause lachten sie ebenso laut auf, wie er sie unterhielt. Und wenn das Weiße in ihren Augen und die brennenden Enden ihrer Zigaretten im Dunkeln nicht geleuchtet hätten, sie wären mit ihren finsteren Klamotten vom Hintergrund der Nacht gänzlich verschluckt worden. Auch Anthony und Leno lachten ein paar mal mit und wechselten nur wenige Worte untereinander, während sie ganz hinten in der Schlange warteten und noch etwas unsicher waren.
\vspace{0.5em} \\
Dann war irgendwann seine Geschichte zu Ende erzählt und der hagere Große verstummte, die letzte Glut war zertreten, die Gruppe Raucher begann sich aufzulösen, und sie schlenderten nacheinander an den beiden Freunden vorbei zurück ins Innere des Clubs. Leno schaute ihnen nach. Jedes Mal, wenn einer von ihnen die Tür öffnete, schwappten einige Schläge des Basses aus dem Innenraum nach draußen und die Musik verlor für einen Augenblick die Dumpfheit und wurde ganz klar. Bei diesen Gelegenheiten drangen dann auch einige Lichtstrahlen durch die Öffnung der Tür, die erst im Nebel in der Luft sichtbar wurden und danach in einem schmalen Streifen farbig an der Wand aufleuchteten. Dort erhellten sie für einen kurzen Augenblick den Vorraum. Erhellten den Kassier und den Türsteher und die gerade Bezahlenden, deren Arme und Schultern und Gesichter. Dann war kurz Licht dorthin gebracht worden, wo es eigentlich nicht hin sollte. Und dann schloss sich die Tür, und versteckte Licht und Musik wieder hinter sich. Und niemand konnte sich dann noch sicher sein, was genau sich hinter ihr befand. Es blieb nur noch eine unsichere Vorahnung. Und an dem jetzt wieder abgedämpften Klopfen des Basses vorbei traten die Stimmen der vor dem Club Wartenden wieder in den Vordergrund. Sie unterhielten sich jetzt genauso lautstark wie gerade eben noch der rauchende dünne Hüne, und es schien ihnen nicht nur egal zu sein, wer ihnen zuhörte, auch der Inhalt ihrer Sätze war egal, solange er nur in der Lage dazu war die Stille zu füllen. Sie alle wussten, dass die Nacht ihr Gerede schon bald und selbstverständlich vergessen machen würde. Und Anthony und Leno addierten sich einfach zu der Ansammlung an Gerede dazu, warfen sich belanglose Sätze zu. Hauptsache zu laut dafür, wie nahe sie an all den Fremden standen, und rutschten währenddessen weiter Richtung Ende der Schlange. Und dann, irgendwann, endlich, waren sie an der Reihe, gaben ihre Ausweise zur Kontrolle und zahlten Eintritt. Der Türsteher, eine gedrungene Gestalt mit langen Haaren und einer plattgedrückten Nase, wünschte ihnen einen schönen Abend, und Leno nahm sich seinen Ausweis zurück, schritt an ihm vorbei, griff zur Türklinge und stemmte seinen Körper nach vorne gegen das Gewicht der Tür, um sie selbstbewusst und kraftvoll aufzustoßen. Dann ließ er Anthony an sich vorbeigehen und folgte seinem Freund schließlich selbst nach.
\vspace{0.5em} \\
Hinter der Tür führte eine langgezogene Treppe nach unten, die gespickt war mit festgetretenem Kaugummi und sich vom Tanz ausruhender Mädchen. Die Musik brauste nun immer lauter auf Leno zu und mit jeder abgestiegenen Stufe wurde das Zurückgelassene ein wenig stärker abgeschwächt. Ein dichter Nebel begann ihn zu umspülen. Er und Anthony näherten sich einer eigenen Welt, einer abgetrennten Blase fern der Normalität und außerhalb jeder Mäßigung. Sie tänzelten mittlerweile zügig, ja, flogen fast die Treppe hinunter, und je tiefer sie kamen, umso schwerer wurde die Luft und umso bissiger der Geruch nach Schweiß, Parfüm und süßlichem Alkohol. Und ganz unten angekommen hatten die beiden schon die Verbindung verloren zu allem, was normal war. Hatten sich losgelöst von den Erinnerungen an den Tag, die es bis hierhin noch geschafft hatten, nicht vergessen worden zu sein, und auch ein wenig von sich selbst. Und dann stiegen sie über die letzte Schwelle und tauchten sich vollständig in den Nachtclub ein.
Dicht gedrängt liefen sie hintereinander, gerade so, wie es der spärlich verfügbare Platz noch zuließ. Leno stieß seinen Atem in flachen Schüben in Anthonys Nacken und sich selbst und ihn dabei nach vorn. Fremde Körper drückten sich gegen sie, sich weg von ihnen, Brust stoß an Schulter, Ellenbogen an Bauch, Hüfte an Arsch. Nachdem sie so wenige Schritte weit zentimeterweise vorangekommen waren, drehte Anthony sich zu Leno um. Er führte eine Hand an seinen Mund, spreizte seinen Daumen und kleinen Finger ab und hob die Hand wie ein Glas, um Leno unmissverständlich klar zu machen, dass sein erstes Ziel die Bar und etwas zu Trinken sein würde. Und Leno lief ihm hinterher, bildete sich nicht erst selbst eine Meinung, sondern nahm die seines Freundes bereitwillig an und folgte ihm nach. Widerstandslos, denn sein Körper wurde sowieso fast ohne sein Zutun dem des Anderen hinterhergeschwemmt. Dabei sah er sich um. Viele Menschen standen die Köpfe zusammensteckend an den Rändern des Innenraums. Viele sahen genauso umher, wie er es tat. Einige, die einfach wahllos im Panorama umherglotzten, manche, bei denen es so schien, als würden sie den Raum präzise absuchen. Ein paar Mal traf sich Lenos Blick mit einem anderen. Dann schaute er kurz in ein Gesicht und ein fremder Blick in Seines. Und dann, einen Moment später, war einer der beiden vergangen. Welcher, das war meist schwer auszumachen.
\vspace{0.5em} \\
Irgendwann stand er verdächtig still. Anthony war mit ihm an der Bar angekommen und drängelte jetzt in eine der Lücken, die sich von Zeit zu Zeit zwischen den Durstigen auftaten. Sein Oberkörper lehnte zum Bestellen bereits weit über den Tresen. Leno drehte sich von ihm weg. Er entdeckte einen abgesenkten Bereich hinter der Bar, auf dem getanzt wurde. Ein Gewühl aus Körpern, in dem kein Einzelner mehr wirklich auszumachen war, zappelte auf der dafür vorgesehenen Tanzfläche. Das Spektakel beobachtend fing er ein bisschen von der Ausgelassenheit auf und hier und da von einem Lächeln, das aus der Masse herausstrahlte. Das Gemenge schlug Wellen, bäumte sich auf und wieder ab, wurde getragen vom Beat, und egal welches Lied spielte, es wurde ein Jauchzen angestimmt, sodass es schien, als wäre auf genau dieses eine gewartet worden.
Wie gebannt war er auf das Schauspiel fixiert. Dann tippte jemand auf seine Schulter und ein Vodka Lemon wurde ihm unter die Nase gehalten. Er nahm ihn und drehte sich um und sah, dass Anthony mit zwei Mädchen anstieß. Er stieß sein Glas ebenfalls dazu und wurde den beiden vorgestellt. Und da Anthony sich bereits mit der Einen unterhielt und er und die Andere nur verlegen danebenstanden, versuchte er mit bestem Vorsatz wenigstens ein kleines Gespräch zustande kommen zu lassen. Aber es ging nicht. Also wendete er sich halb ab, starrte wieder auf die Tanzfläche. Noch einmal schaute er zu Anthony, sah diesen nah am Ohr des Mädchens sprechen und sie anschließend kichern. Dann machte er einen Schritt in ihre Richtung, um sich zu seinem Freund vorzulehnen: „Ich geh mal nach unten zum Tanzen.“
\vspace{0.5em} \\
„Ok, wir finden uns dann später schon, oder? sag?“, sagte Anthony.
\vspace{0.5em} \\
„Ja, klar.“, nickte Leno selbstverständlich. Und er taumelte rüber zur Tanzfläche, hielt sich am Geländer fest und stützte sich ab, während er die hohe Stufe hinunterstieg. Er konzentrierte sich, die Füße gut zu setzen, nicht auszurutschen und hinzufallen, den Kopf dabei oben zu lassen und sein Getränk nicht zu verschütten. Etwas wacklig, durchaus aber noch souverän sank er ab. Dann stand er unten bei den Tanzenden, unbeholfen und noch vollkommen unsynchron. Zuerst bewegte er Kopf und Schultergürtel im Rhythmus, dann folgten die Knie nach, mit denen er zum Takt wippte. Dann stellte er sein Bein einen halben Schritt weit aus, drehte seine Hüfte und verschmolz mit der sich vergnügenden Menge. Er trieb über die Tanzfläche, tanzte bald mit diesem und jenem Mädchen. Ließ sich vom Beat in den Moment drücken. Seine Beine schoben ihn rhythmisch vorwärts, seine Arme warf er zum Takt der Musik wild durch die Luft. Der Nebel umhüllte ihn und die anderen, die Bar, die Wände, Anthony. Bunte Lichter erscheinen mal am linken, mal am rechten Rand seines Sichtfelds, blitzen auf und zerschmolzen wieder zurück in den Nebel. Die Luft tanzte, er tanzte mit und tanzte mit allen und tanzte im Raum herum, bis der Raum auch um ihn zu tanzen schien. Er fand sich gegenüber einer Frau und liebte es wie sie ihre langen Haare durch die Luft wirbeln ließ. Er tanzte mit einer Anderen und genoss es wie sie ihn kurz durchdringend ansah, nur um im nächsten Moment durch ihn hindurch zu blicken wie durch Glas. Er tanzte mit einem in Hemd und Krawatte gekleideten Mann und liebte die Ausgelassenheit in dessen Gesicht. Und seine Beine trugen ihn aus der Mitte der Tanzfläche zum Rand und von dort wieder zurück ins Zentrum, wo Blickfeld und Horizont beschränkt waren auf die ihm nächsten handvoll Menschen. Und Leno nahm sie an, fragte nicht nach, lud sie ein zu seinem Tanz und liebte sie. Er ließ die Nacht geschehen, tanzte darin und nahm sie mit. Er wusste nichts über die Menschen, mit denen er tanzte, und wusste doch, dass sie sich gleichzeitig vollkommen von ihm unterschieden und ihm gänzlich glichen. Und dafür liebte er sie alle gleich, er liebte sie alle gleich und die Mädchen doch besonders. Und er wünschte sich von allen gleich, dass sie seine Liebe zulassen würden, und von den Mädchen doch besonders. So tanzte Leno immer wilder, ließ seinen Körper sich aufrichten und seine Augen mit Entschlossenheit blitzen. Er gab das Versprechen von Kraft und Schönheit, von Sicherheit und Willen, von Vertrauen und Treue und warb damit. Er hatte den Wunsch begehrt zu werden, bewundert. Er wünschte sich Möglichkeiten. Er warb um die Geheimnisse und Eigenheiten seiner Mittänzer und um die Erlaubnis, diese ergründen zu dürfen. Seine Beine flogen über den Boden, sein Hemd war durchnässt vom Schweiß und Vodka Lemon und sein Atem ging schnell und flach. Er wollte der Größte sein, der Stärkste und Schönste und die Macht haben, die Menschen dazu zu bringen für ihn ihre Hüllen abzulegen, ihn sie erfahren und lieben zu lassen.
\vspace{0.5em} \\
So tanzte Leno eine Ewigkeit lang, oder nur einen Augenblick, oder ganz außerhalb der Zeit. Dann veränderte sich seine Umgebung. Er sah seinen Freund Anthony auf ihn zukommen. Und wo sich gerade noch das große Wissen befand, dass Leno keine Sicherheit haben konnte über die ihn Umgebenden, was sie alle für ihn gleich und gleich unterschiedlich machte, da saß jetzt ein kleines Wissen. Sein kleines Wissen über seinen Freund tauchte wie ein Fels in einem Meer des Verhüllten auf und Leno trieb von einer ihm unbekannten, seinem Freund entspringenden Kraft gezogen, auf diesen zu. Er erreichte und griff nach ihm und hielt sich fest. Und sein Körper stabilisierte sich, wurde nicht mehr von den Wellen aus Musik und Tanz umhergeworfen und war nun nur noch ihm bekannt. Den ihn umspülenden Bewegungen und Körpern war er alsdann fremd geworden. Er folgte seinem Freund nach draußen und ließ sich auf der Tanzfläche stehen. Aber er nahm auch etwas mit. In seiner Brust gewachsen war etwas Tiefes und Weites, das er weder in Worten noch in Gedanken fassen konnte und von dem er nur einen Teil überhaupt erfahren konnte. Er sehnte sich, und die Sehnsucht brannte in ihm, während Anthony ihn Richtung Ausgang zog.
\chapter{Nach dem Club}
Leno und Anthony tauchten gemeinsam aus dem Nebel des Nachtclubs auf. Anthony ging voraus, die Kippenschachtel in der einen Hand, zwei Wasserflaschen in der anderen. Er näherte sich ein paar Sitzsteinen bei einem Brunnen und ließ sich nieder. Um den Brunnen herum, der um diese Jahreszeit trocken war, standen mehrere lebensgroße Steinstatuen. Leno setzte sich zu Anthony.
\vspace{0.5em} \\
Sein Freund hielt ihm eine der beiden Wasserflaschen hin: „Ich habe mir gedacht, du willst bestimmt auch eine.“
\vspace{0.5em} \\
Leno griff dankbar nach der Flasche. „Ja, Danke, dass du an mich gedacht hast.“, entgegnete er, schraubte das Wasser auf und nahm einen großen Schluck.
\vspace{0.5em} \\
Anthony hatte sich in der Zwischenzeit eine Zigarette angezündet. Fragend nickte er in Lenos Richtung: „Willst du auch eine? Hm?“
\vspace{0.5em} \\
Leno griff nun ebenfalls nach der geöffneten Schachtel und nahm sich eine Zigarette heraus. Anthony gab ihm das Feuer und er zündete sie an. Er nahm einen tiefen Zug und merkte schnell, dass er zu betrunken war, um die Wirkung des Rauchs in seinen Lungen zu spüren. Anthony wandte sich zu ihm und sprach energisch.
„Hast du die mit den langen Haaren gesehen? Ich meine, hast du sie dir angeschaut?", er schüttelte ungläubig den Kopf, "Das ist doch nicht fair, oder, sag? Mit einem solchen Hintern und den Brüsten auch noch so hübsch zu sein.“ Er schaute kurz auf, schüttelte nochmals den Kopf und öffnete die Hände in einer entrüsteten Geste. Dann führte er die Zigarette wieder an seinen Mund. Leno zog nochmal, aschte neben sich auf den Boden.
\vspace{0.5em} \\
Er antwortete: „Ja, ihre Haare waren schön. Und tanzen konnte sie.“
\vspace{0.5em} \\
Anthony schraubte nun auch seine Wasserflasche auf und trank. Leno nahm dies zum Anlass, sein Wasser ebenfalls anzusetzen und hinunterzustürzen. Er war seiner Besoffenheit überdrüssig. Das Wasser fühlte sich frisch an, neu, rein und klar. Es schmeckte ein bisschen danach, als könnte es seinen Rauschzustand lindern. Das täuschte natürlich nur. Sein Kopf fühlte sich noch immer genauso beschwert und vernebelt an wie zuvor, so wie er sich die nächsten Stunden eben anfühlen wird. Anthony meinte zu Leno, dass er ihn geholt habe, weil sie nun los müssten. Sonst würden sie den letzten Zug verpassen. Kurz nachdem sie sich auf den Weg machten bat er Leno, eine Navigation zum Bahnhof zu starten. Er tippte angestrengt die dafür notwendigen Eingaben in sein Mobiltelefon und sie begannen, der blau in der Karte eingezeichneten Linie zu folgen. So liefen die betrunkenen Freunde nebeneinander her durch die Nacht, vom Club in die Stadt gespuckt. Dabei kontrollierte Leno in winzigen Abständen das Display, auf dem sich der Pfeil beständig an der blauen Linie entlang in Richtung Hauptbahnhof schob.
Als sie angekommen waren, setzten sie sich in den bereits am Gleis stehenden Zug. Anthony setzte sich ans Fenster, Leno zum Gang. Er fragte Anthony, ob er dessen noch halb volles Wasser austrinken durfte. Anthony bejahte. Als der Zug dann losfuhr, war er schon eingeschlafen. Leno, der sich tief in den Sitz drücke, hatte seinen Blick auf das Fenster der gegenüberliegenden Sitzreihe gerichtet und mit der Nacht noch nicht abgeschlossen.
\chapter{Heimweg}
Als der Zug ihre Haltestelle erreichte, weckte Leno seinen Freund auf. Die Erschöpfung des langen Tages und der Nacht steckte beiden jetzt tief in den Knochen und die kurze Strecke von ihrem Sitzplatz durch das Abteil bis auf den Bahnsteig fiel ihnen unendlich schwer. Und nachdem sie diese mit größter Anstrengung geschafft hatten, mussten sich die Freunde zunächst auf der Sitzbank vor der Haltestelle von ihr erholen. Anthony schlug vor, sich gemeinsam in einem Taxi nach Hause fahren zu lassen, aber Leno wollte nicht. Er hatte schon während der Zugfahrt den Plan gefasst, zu Fuß zu gehen. Trotzdem zog Anthony eine Visitenkarte aus dem Geldbeutel, um einen Taxidienst anzurufen. Es war ihm genau anzusehen, dass er jetzt und hier, direkt auf der Sitzbank, in der selben Sekunde wieder einschlafen könnte, in der er es sich erlauben würde. Aber obwohl kein Wort mehr geredet wurde, bis das Taxi vorfuhr, hielt er sich eisern wach. Dann versicherte er sich nochmals, ob Leno nicht doch mitkommen möchte, was er natürlich verneinte. Und dann fuhr das Taxi fort und Leno begann, nach Hause zu laufen. \\
Er fragte sich, ob die Tänzerinnen noch an ihn dachten. Er selbst dachte noch an sie. An die, die ihn so durchdringend fordernd, und gleichzeitig versprechend angesehen hatte und an die, mit den wunderschönen Haaren. Er hatte in diesem Moment nicht nachgedacht, war völlig in ihm befestigt gewesen. Er formte vor seinem inneren Auge die Gesichter der Tanzenden, ließ sie einzelne Szenen nachspielen, durchlebte einige Gefühle ein zweites Mal. So in der Erinnerung schwelgend schritt er von der Haltestelle weg. Der schicke Mann, mit dem er ebenfalls getanzt hatte, kam ihm in den Sinn. Er fragte sich, ob die Frauen auch an diesen Mann dachten. Leno selbst dachte an alle Frauen gleich und hielt sie gleichermaßen in Ehren, aber er musste sich eingestehen, dass er sich von ihnen etwas anderes wünschte. Er wollte nicht neben einem anderen stehen müssen. Er wollte sich keine Liebe teilen müssen. Er bog links ab. Während er an der Kirchenmauer entlanglief, ließ er seinen Handrücken an ihr schleifen.\\
Und dann merkte Leno, dass es egal war, ob diese Liebe für ihn allein war. Er merkte, dass es nie auch nur teilweise seine Liebe gewesen war. Die Frauen konnten nicht an ihn denken. Sie könnten ihn nicht lieben. Denn die Wahrheit war, dass sie ihn nicht kannten. Schließlich dachte auch er, Leno, wenn er sein Gehirn ein Bild von ihnen zeichnen ließ, nie an sie, wie sie gerade mit dem Taxi heimfuhren. Nie an einen Menschen mit schönen Haaren, der vielleicht noch immer im Nachtclub tanzt, oder an einen mit klarem Blick, der sich vielleicht gerade selbst damit im Spiegel betrachtet. Für Leno waren sie Tänzerinnen gewesen und würden es immer sein, eingefangen in einem vergangenen Moment. Ihr Wesen war ihm verborgen geblieben hinter der strahlenden Wahrheit seiner Wahrnehmung. Und so konnte auch er für sie nur ein Licht der Vergangenheit sein. Egal wie sehr der Tänzer geliebt wurde, Leno, der Mensch, würde unerkannt und ungeliebt sein. Und während er die letzten Gedanken fasste, war sein Gang immer langsamer geworden und kam schließlich ganz zum stehen. Er verharrte kurz still. Dann wandte er sich nach links, fasste mit seinen Händen auf die Oberseite der Kirchenmauer, zog sich an ihr hoch und setzte sich. Er fühlte in sich hinein und saß zufrieden. Er wusste, seine Einsicht bedeutete nicht, dass er einsam war. Sie erlaubte ihm loszulassen von dem Wunsch, als Mensch geliebt zu werden. Denn er hatte niemals einen solchen Anspruch an eine der Tänzerinnen gestellt. Es lag eben darin die Schönheit dieser Liebe. An ihrer Anspruchslosigkeit. Von Leno war nichts weiter verlangt worden, als zu sein. Für diesen Moment. Und genau so hatte auch er nicht mehr von den anderen verlangt. So war es richtig gewesen und schön, und er erinnerte sich voller Glück an den zeitlosen Tanz, und an das Einverständnis des Moments mit sich und ihnen und ihm. Aber so schön es gewesen war, es war ihm nicht genug. Er wollte nicht nur mehr für sich, er brauchte nicht nur mehr für sich. Er wollte alles. Und er wusste, dass er diesen Willen nach allem brauchte. Und Leno saß auf der Kirchenmauer und fühlte erneut ein Brennen in seiner Brust. Ein Brennen ähnlich dem, das er schon früher in der Nacht gespürt hatte, das aber dennoch ganz anders war.\\
Er atmete tief ein, stützte sich ab, schob seinen Körper nach vorn und ließ sich langsam auf die Erde gleiten. Genug ausgeruht. Er machte sich wieder auf den Weg, schließlich hatte er ein Ziel und musste nach Hause kommen. Und sein gemächlicher Gang brachte ihn diesem Schritt für Schritt näher. Irgendwann kam er an, schloss die Tür hinter sich, fiel aufs Bett und schlief sofort ein.
\chapter{Kater}
Nach drei Stunden unruhigem Schlaf schlug Leno die Augen auf. Sein ganzer Körper schmerzte. Er drehte sich zur Seite und hustete. Er lag etwas. Er hustete nochmal. Er lag wieder etwas. Sein Mund war trocken und sein Kopf innen ganz schwammig. Die Gedanken entglitten ihm unfolgsam. Mit größter Anstrengung befahl er seinem Arm einen Halbkreis beschreibend über der Matratze entlangzufahren. Der Stoff schliff weich an seinem Unterarm. Er schob den Arm immer weiter vorwärts, so lange, bis er endlich an einen kalten, flachen Gegenstand stieß. Als er den Kontakt spürte, griff er den Gegenstand mit der Hand, führte ihn am Bettlaken entlang an seinem Körper vorbei bis neben seinen Kopf und drückte den Anschalter. Freudig antwortete das Display und feuerte ein gleißendes Leuchten in Richtung seines Gesichts. Er wurde unerwartet stark geblendet, und als Antwort darauf kippte er das Telefon ruckartig weg und drückte es mit dem Bildschirm nach unten in den Bettüberzug, wie als würde er es ersticken wollen. Das Leuchten reduzierte sich dadurch zu einem rechteckigen Schimmer an den Rändern des Geräts. Jetzt blickten Lenos halboffene Augen nur noch in das graue Nirgendwo seines Zimmers. Und nach einer kurzen Weile schloss er sie wieder und schob den Kasten von sich weg.\\
Es ging ihm wirklich nicht gut. Der ganzen Länge nach fühlte er in seinen Körper hinein, befahl den steifen Gliedern sich zu strecken und lag dann wieder still. Er hatte Kopfschmerzen. Und ihm war unwohl. Am ganzen Körper war ihm unwohl, von den äußersten Stellen, den Finger- und Zehenspitzen bis ins mittlerste Zentrum seines Selbst. Und das Unwohlsein ließ sich nicht abschütteln, wie ein zäher Schleim, der überall an ihm haftete und den er nicht wegwischen konnte. Wie zäher, glibbriger, ekliger Schleim. Und er würde abwarten müssen, bis dieser langsam an ihm heruntergelaufen war, um dann Tropfen für Tropfen abzuperlen, bis er sich endlich wieder gut fühlen durfte. Ihm war schlecht.\\
Nachdem er noch eine Weile so gelegen war, wagte er einen erneuten Versuch, die Augen zu öffnen, schaltete das Mobiltelefon wieder an und zwang sich das Brennen seines Leuchtens so lange zu ertragen, bis er die Helligkeit des Bildschirms herunterregeln konnte. Es gelang ihm. Auf der nun erträglich strahlenden Oberfläche wurde eine Nachricht Anthonys angezeigt:\\
„Mir geht's heute gar nicht mal so gut haha\textasciicircum\textasciicircum“\\
Leno legte das Mobiltelefon wieder weg. Gleich würde er von seinem Bett aufstehen, Wasser trinken und seinen Körper wieder stärken. Nur ein bisschen musste er noch liegen bleiben und sich auf diese nun vor ihm liegende Aufgabe vorbereiten. Er streckte sich nochmal. Dann versuchte er sich wenigstens teilweise aufzurichten, beugte seinen Oberkörper weit zur linken Seite und zog den Vorhang von seinem Fenster. Postwendend stieß ihn das eintretenden Licht der Sonne brutal in sein Bett zurück. Drückte ihn erbarmungslos auf und in die Polster. Drückte ihn mit solcher Kraft in seine Matratze, dass ihm nichts anderes blieb als zu liegen, sich dann auf die andere Seite zu drehen und liegen zu bleiben. Er musste husten. Der Brechreiz war etwas abgeklungen, aber es fühlte sich nicht danach an, als würde es ihm besser gehen. Die Übelkeit schien sich vielmehr aus dem Magen in irgendeine andere Region des Körpers verzogen zu haben. In eine, aus der sie nicht im selben Maß ins Bewusstsein dringen konnte. Irgendwo hinter die Milz, oder zwischen Leber und Gallenblase. Im Mund hing ihm ein Geschmack, als wäre gestern eine Ratte darin gestorben.\\
Er wollte sich zum Aufstehen zwingen, und scheiterte. Er griff nochmal nach seinem Mobiltelefon. Er sollte sich die Zähne putzen. Mit den Daumen tippte er auf das Glas des Bildschirms und konsumierte mit den Augen dessen Schimmern. Mittlerweile wusste er nicht nur, dass er besser Wasser trinken sollte, der Durst wurde übermächtig. Er stellte es sich vor aufzustehen, und das Aufstehen fühlte sich unsagbar schwer an. Er ging in Gedanken zum Wasserhahn und die Bewegungen fühlten sich unendlich mühselig an. Er fantasierte davon das Wasser mit tiefen Zügen zu trinken und blieb derweil liegen und drückte den Morgen in sein Kissen. Er war mit sich unzufrieden. Er wusste, dass das Wasser ihm gut tun würde. Dass sein Körper es brauchte. Für seine Gesundheit und sein Wohlbefinden. Wenn er wirklich liegen bleiben würde. Wenn er nicht in der Lage dazu wäre, das zu tun, was am besten für ihn war, dann konnte er auch einfach nie wieder aufstehen. Konnte für immer liegen bleiben. Konnte sich suhlen im Schleim, der ihn einschmierte. Einfach nie mehr aufstehen und immer nur das tun, was den geringsten Widerstand entgegenbringt. Also nichts. Gar nichts. Nur liegen. Wenn er sich nicht für etwas entscheiden und es dann umsetzten konnte, dann brauchte er auch nicht mehr nachdenken. Einfach liegen und nicht nachdenken. Nicht vergleichen, nicht entscheiden. Jeden Gedanken willenlos abgleiten lassen. Nur liegen. So lange liegen, bis er tot war. Ausgeliefert und unbeeinflusst liegen. So lange, bis er an die Bettdecke angewachsen war und mit dem Schmierschleim verschmolzen. Bis alles, die Decke, die Matratze und sein Körper eine undefinierbare Masse war und er tot. Wenn er nicht so handelte, wie er wusste, dass es gut für ihn war, dann konnte er es auch sein lassen. Also alles. Handeln überhaupt. Das ganze Aufstehen, Machen, Einschlafen, Aufstehen. Das ganze Getreibe, Gejage und Getue. Alles sein lassen und jetzt, so wie er war, hier liegen bleiben. Bis er sich erst Einpissen und Einscheißen würde und schließlich verdurstete und starb. Weil sich Leno aber sicher war, dass er gern lebte und nicht sterben wollte, stand er auf, ging in die Küche, goss sich ein großes Glas Wasser ein und zog es sich mit riesigen Schlücken in den Rachen. Gleich als nächstes goss er sich ein zweites Glas ein und stürzte auch dieses herunter. Und tatsächlich ging es ihm danach besser. Das Wasser schmeckte rein. Unbelastet. Es fühlte sich so an als könnte es sich mit Lenos versäuertem Körper vermischen, den Zustand irgendwie verdünnen und damit erträglicher machen. Und es tat es auch. Er putzte noch die Zähne, machte das Fenster auf und legte sich völlig geschafft wieder ins Bett. Schloss die Augen und döste noch eine Weile vor sich her. Dann durchsuchte er die für sein Mobiltelefon angebotenen Anwendungen nach einem Spiel und schoss für ein paar Stunden kleine Kugeln in farbiger Zusammengehörigkeit an eine virtuelle Decke. Zwischendurch antwortete er Anthony:
\vspace{0.5em} \\
„mir auch :D“
\vspace{0.5em} \\
Das neu heruntergeladene Spiel war nicht unkompliziert. Die Reihenfolge der abzuschießenden Farben war nicht immer sofort ersichtlich, und manchmal musste die Farbe der als nächstes zu verschießenden Kugel ausgetauscht werden. Dabei wurden Punkte gezählt. Leno strengte sich so gut es ging an, die Kugeln zu schießen, viele Punkte zu machen, und versuchte dann - in der nächsten Runde - noch ein bisschen besser zu sein. So konnte er alles Weitere ein bisschen verdrängen - seinen Kopf, seinen Magen. Und irgendwann war es Abend und der Tag war fast totgespielt und ihm ging es zumindest ein bisschen besser. Dabei hatte er die Uhr immer gut im Blick. Er bemerkte sehr genau, wann es Zeit gewesen wäre, die Sachen zu packen und zum Schwimmbad zu gehen. Und er ließ die Zeit verstreichen und blieb liegen. Drei Minuten, fünf Minuten. Dann, mit einem lauten Stöhnen, stand er doch auf, packte robotisch ein Handtuch und eine Badehose und warf beides in den Rucksack. Nahm die Wasserflasche heraus und kontrollierte, ob die Schwimmbrille noch eingepackt war. War sie. Ging in die Küche, füllte die Wasserflasche auf, warf sie zu den anderen Sachen zurück und zog den Reißverschluss zu. Dann hob er den Rucksack vom Boden auf und warf ihn sich über den Rücken. Und dann stand er noch eine Zeit lang im Raum, die Augen suchend nach einem Gegenstand, den er vergessen haben könnte, oder einem hinreichenden Grund, der es ihm erlauben würde zuhause zu bleiben. Flüchtig sprangen seine Gedanken durch das Zimmer. Und nachdem die Suche erfolglos geblieben war und er nichts davon finden konnte, trat er zur Tür. Endlich legte er die Hand auf die Türklinke, drückte und zog, öffnete damit die Tür, entkorkte die Welt wie einen billigen Wein und musste hilflos zusehen, wie sie sich in seinen Hausflur ergoss. Dann schritt er nach draußen, schloss die Tür hinter sich und entfernte sich schweren Schritts von dem Ort, an den er von der Nacht wie ein zerkauter Kaugummi in den Raum geklebt worden war. Beim Verlassen des Orts zog der Kaugummi lange Fäden, die ihn ähnlich einer physikalischen Anziehungskraft zurückzogen. Er schritt voran und ließ die ihm anheftenden Stränge sich in die Länge ziehen, bis sie hauchdünn wurden, zerfielen und kraftlos zu Boden sanken. Dann setzte er seinen Gang fort, leichter und federnder, für den äußeren Betrachter aber immer noch etwas langsam anmutend.
\chapter{Training}
Mehrmals hatte Leno auf dem Weg der Versuchung widerstanden, sich einfach auf den Boden zu legen anstatt den nächsten Schritt zu gehen. Obwohl er wusste, dass er früher oder später ankommen würde, war ihm sein Ziel bis zur letzten Abzweigung wie in einer undefiniert weiten Ferne erschienen. Tatsächlich war es nun aber geschehen und er war im Schwimmbad angekommen, zog sich um und ließ seinen Rucksack neben die Sitzbank gleiten. Er ging sich abduschen, nahm sich dabei viel Zeit, gewöhnte seinen Körper gleichmäßig an das Wasser. Er arbeitete die Bewegungen ab und dachte keinen Gedanken. Seine Hand griff zum Wasserhahn, drehte ihn noch ein Stück weiter auf und er beugte langsam seinen Kopf nach vorn. Er schloss Mund und Augen, hielt sein Gesicht unter den Strahl. Das Wasser traf Stirn und Wangen, prallte ab und tropfte kühl auf seinen Oberkörper.\\
Das Gehen hatte ihm gut getan. Hatte das körperliche Unwohlsein zumindest teilweise von ihm abgleiten lassen. Und unweigerlich hatte es seinen Körper unter diese Dusche geführt, wo nun unerbittlich die Tropfen auf ihn niederschlugen. Und es schien sich so zu Verhalten, dass dieses Einschlagen der Wassertropfen auf seiner Haut zu bedeuten hatte, dass er in wenigen Augenblicken Bahnen im Schwimmbecken ziehen würde. Wie er schon viele Male Schwimmen gegangen war, nachdem er sich in dieser Dusche gesäubert hatte. Und trotzdem war es kein Ritual. Er befand sich in keinem Zustand der Vorfreude oder Erwartung. Er war zu nichts bereit oder entschlossen. Er machte sich einfach nass. Nicht mehr. Das war Alles. Und irgendwann nach einem nicht genau zu bestimmenden Zeitabstand hatte ihn das Wasser vom Kopf bis zu den Zehenspitzen überall genug berührt und er ging zum Schwimmbecken, setzte sich an dessen Rand und hielt die Füße hinein. Eine Weile verblieb er schweigend, sitzend, beobachtete das Wasser mit den anderen Schwimmern, die leise gleitend an ihm vorbeizogen. Dann drückte er die Hände nach unten auf den Boden am Rand des Beckens, hob seinen Körper, und schob seine Hüfte nach vorn. Ließ ein paar ereignislose Momente verstreichen, hörte schließlich auf, durch seine Schultern, Ober-, und Unterarme Druck auf seine Handballen auszuüben, entspannte seinen Trizeps und ergab sich der Schwerkraft, die ihn ohne Zögern in und kurz darauf unter das Wasser zog. Anschließend tat er nichts. Schwebte nur unter der Oberfläche. In zufälliger, vorbestimmter Bewegung. In keinem größeren Unterschied zu den Molekülen der Flüssigkeit, die ihn umgaben. Dann übernahm er wieder die Kontrolle über sich, tauchte auf und griff nach dem Beckenrand, um sich an ihm entlang zu einer freien Bahn zu ziehen. Mutwillig schwenkte er seinen Körper und richtete sich parallel zur Längsseite des Schwimmbeckens aus. Brachte sich in die optimale Position, um genau geradeaus loszuschwimmen. Und dann wollte er sich mit den Beinen von der Wand abstoßen und alles was seine Muskulatur hergab war ein schwächliches Hineinschieben seines Körpers in die Bahn, das seinen Körper zu nicht mehr als einem zögerlichen Treiben ermutigte. Der gute Vorsatz ließ ihn dabei zumindest die Arme gestreckt nach vorne halten, an deren Enden das Wasser träge durch seine Finger floss. Bald zog er den linken Arm unter sich nach hinten durch, während der Rechte weiter gestreckt blieb. Bis er den Linken auf Hüfthöhe aus dem Wasser nahm, ihn in der Luft nach vorn bewegte und direkt neben seinem Ohr wieder hineinstieß. Dann zog er den anderen Arm, den Rechten, unter sich entlang, drückte sich ab und schob sich geradeaus. Und das Wasser strömte an ihm vorbei wie an jedem anderen Tag. Er schwamm nicht schnell, er schwamm keinem Plan folgend und nicht konzentriert, er fühlte sich weder gut noch stark, aber er war im Wasser. Er schwamm. Die Armzüge, die gezogen werden mussten, zog er. Die Beinschläge, die geschlagen werden mussten, schlug er. Wenn ihm eine Rollwende gut gelang, schwamm er weiter. Wenn ihm eine Rollwende nicht gut gelang, schwamm er weiter. Und er konnte dabei fühlen, wie er sich am Wasser entlang zu seinen Zielen ziehen konnte. Genauso wie er fühlte, wie sie vom fließenden Wasser zu ihm getragen wurden. Wie sie irgendwo in der Ferne auf ihn warteten und näher rückten. Er konnte sie als festen Bestandteil des Wegs wahrnehmen. Seine bloße Anwesenheit in diesem Strom war ausreichend und er würde sie erreichen. Gleichzeitig zog er mit aller ihm gegebener Kraft, stieß, schlug, rang und biss. Und er ergab sich dem Ziehen und Schlagen, dachte nicht nach, hinterfragte nicht. Er schwamm von einer Seite des Beckens zur anderen, wendete und durchquerte das selbe Wasser wieder zur anderen Seite, nur um dort erneut zu wenden, um wieder das selbe Wasser zu durchqueren. Und doch schwamm er dabei immer vorwärts, nie zurück, drehte sich nicht im Kreis, sondern ging einen Weg dessen Sinn und Zweck er kannte. Und ihm wurde während dieser Schwimmzüge bewusst, wie sehr er eben jenen Sinn brauchte. Wünsche waren nicht genug für ihn. Er war ihnen überdrüssig, den ewig entfernten Versprechen. Er musste los, er musste weiter, er musste sich bewegen. Vorwärts.
\chapter{Mehr Training}
Von da an war Leno fast jeden Tag im Schwimmbad. Tagsüber arbeitete er und Abends stellte er seinen Rucksack neben der Sitzbank ab, duschte sich, und ging ins Wasser. Wenn er sich an einem Tag zu müde zum Schwimmen fühlte, ging er ins Schwimmbad und schwamm. Wenn er keine Motivation fand zu schwimmen, ging er ins Schwimmbad und schwamm. Wenn es einen eindeutigen, logischen Grund gab, nicht zu schwimmen, ging er ins Schwimmbad und schwamm. Es war kein Wille notwendig. Er musste sich nicht dazu aufraffen. Es war eine Frage, die sich schlicht nicht stellte. Es gehörte zu ihm wie Atmen und Schlafen, wie der Leberfleck auf seinem Nacken. Er packte jeden Tag seine Sachen, lief ins Schwimmbad und stieg ins Wasser mit der selben Beharrlichkeit mit der die Sonne täglich auf-, und unterging. Und es war nicht so, dass er sich dabei auf irgendeine Weise besonders glücklich fühlte. Es fühlte sich für ihn nicht so an, als würde er besonderen Ehrgeiz oder besondere Ambition zeigen. Nicht so, also würde er irgendein ihm gegebenes Potential ausschöpfen. Nicht danach, dass er irgendetwas gut oder richtig machen würde. Es grenzte lediglich die Summe der Entscheidungen ein, denen er sich wieder und wieder entgegenstellte und gab ihm den Frieden, sein Handeln nicht ununterbrochen bewerten und hinterfragen zu müssen. Das war das deutlichste Indiz dafür, dass diese Entwicklung eine Gute war. Eine latente Sicherheit, die seine Gedanken eingrenzte. Die seinen Geist davor bewahrte, ständig in jede mögliche Richtung auszuschwärmen. Sie beschränkte seine Möglichkeiten auf eine übersichtliche Fläche und gab ihm mehr Raum für das Handeln und Sein selbst. Wenn er schwamm, dann schwamm er. Nachdem sein Arbeitstag vorbei war, ging er kurz nach Hause, füllte seinen Rucksack mit den paar benötigten Dingen, und ging zum Schwimmbad. Und mehr und mehr beheimatete er sich in dieser Routine.\\
Nachdem sein Training regelmäßiger wurde, wurde es auch seine Leistung. Mit jedem neuen Armzug stabilisierte sich Leno's Bezug zum Wasser, zu seinem Körper und zum Verhältnis der beiden zueinander. Immer weitgreifender konnte er sich die Kontraktionen seiner Muskeln und die Winkel seiner Gelenke bewusst machen, prüfen und korrigieren. Er wurde ungemein feinfühlig bei der Taktung seiner Atmung. Erlange die Fähigkeit, die Zeitpunkte der Atemzüge so präzise auszurichten wie ein Metronom. Dabei nicht etwa orientiert an Abständen gemessen in hundertstel oder tausendstel Sekunden, sondern vielmehr am ureigenen Rythmus seines Körpers. Genauso richtete er jede seiner Bewegungen immer näher an den ihm eigenen Gegebenheiten aus. Ließ sie mehr geschehen, als dass er sie bewirkte. Lernte besser zu erkennen, was er wann von seinem Körper erwarten konnte. Wo seine wahren Grenzen lagen. Und allmählich fing er an Gefallen daran zu finden sich anzutreiben. Immer häufiger wurde sein Blick zur Uhr vor und nach einer Strecke. Er wollte den Umriss seiner Leistungsfähigkeit benennen, mit dem Finger darauf deuten. Er fing an, sich seine Zeiten für kurze Strecken zu merken, dann für längere Strecken. Verglich die Zeiten des einen Tages mit denen des nächsten und denen der vorhergehenden Woche. Er begann in Intervallen zu trainieren. Verausgabte sich vollends, stieg aus dem Wasser, setzte sich neben seinen Rucksack auf die Sitzbank und rastete. Nachdem er sich später wieder erholt hatte, schwamm er die selbe Strecke erneut mit voller Kraft. Er experimentierte mit seiner Atmung, atmete mal nach nur zwei Armzügen, mal nach acht. Passte die Frequenz seines Atems der Strecke an, oder konzentrierte sich ausschließlich auf die Atmung, ohne Berücksichtigung von Strecke und Zeit. Er modulierte die Feinheiten und Arten seiner Technik, die Gestalten und Verhältnisse seiner Bewegungsabläufe, erweiterte seine Auffassung des Schwimmens an sich. Und mit jedem Muskel, den er besser unter seine Kontrolle brachte, schienen sich neue Kapazitäten freizusetzen. Geschwindigkeiten, die ihm vor wenigen Tagen noch alles abverlangt hatten, konnte er bald über weite Strecken halten. Seine Arme glitten fast mühelos durchs Wasser. Gleichzeitig beförderten sie ihn mit mehr Kraft nach vorne als jemals zuvor. So als ob das Wasser nur darauf gewartet hätte, ihn anschieben anstatt abbremsen zu dürfen. Am auffallendsten war sein Fortschritt bei der Wende. Schon immer war er beim Auftauchen nach der Rolle unter dem Wasser aus dem Rhythmus gebracht, geschwächt und außer Atem gewesen. Das war jetzt nicht mehr so. Jetzt freute er sich beim Zuschwimmen auf den Beckenrand schon darauf, abzutauchen, sich zu drehen, kräftig von der Wand abzudrücken und ein weites Stück unter Wasser zu gleiten. Und mit jedem Mal, welches er die Wende ausführte, verstand er sie noch ein bisschen besser. Lernte, Unterschiede des Abstands zum Rand auszugleichen. Verinnerlichte, wie er die Technik an seine Geschwindigkeit anpassen musste. Oft führte er die Rolle in Gedanken auch bei der Arbeit, oder wenn er mit Freunden unterwegs war aus. Und er freute sich dann schon aufs nächste Training, wenn er weiter an seiner Ausführung feilen konnte. Es hatte etwas Schönes. Einfach, weil es eine komplizierte und ästhetische Abfolge von Bewegungen war.\\
Eines Tages kaufte Leno sich schließlich ein Notizbuch und fing an, Strecken, Zeiten und Pausen zu notieren. Von da an saß er oft zuhause über seinen Notizen am Küchentisch und verglich die Leistungen seiner letzten Einheiten, notierte seine Steigerungen gesondert und stellte Prognosen an, wie sie sich voraussichtlich in Zukunft entwickeln würden. Nicht jedes Training war gut, nicht jeder Tag war leicht, aber jedes Training und jeder Tag war ein Schritt den er ging. Und Leno konnte die Schritte abgehen, aufzeigen und in seinem Buch niederschreiben. Dann richtete er mehr Aufmerksamkeit auf seine Ernährung. Sammelte Wissen über den Stoffwechsel und die Energiebereitstellung an, las Artikel und Studien. Er sah Videos über die Diäten der besten Schwimmer der Welt. Er fing an Mahlzeiten genauer zu planen, las Inhaltsangaben und Nährwerttabellen. Er stimmte seine Nahrung auf sein Training ab, ließ es zur Gewohnheit werden, jedes Essen auf seine Nährstoffe zu schätzen und zu bewerten. Er entschied darüber, zu welcher Zeit er seinem Körper welche Menge Energie zuführen wollte, achtete auf seinen Mineralstoff- und Vitaminhaushalt. Und so ging Leno immer mehr Schritte auch außerhalb des Wassers, außerhalb des Trainings. Er schob Erledigungen auf, verzichtete darauf, den Verabredungen seiner Freunde zuzusagen. Auch John und Anthony sah er seltener. Er wurde sich der Opfer, die er brachte, bewusst. Es wurde ihm bewusst, dass sie genau den gleichen Anteil an seinen Schritten hatten, wie die körperliche Arbeit, die er im Schwimmbecken verrichtete. Und damit verloren seine Schritte ihren Rhythmus, ihre Struktur, und sein Vorwärtskommen wurde zu einem andauernden Gleiten, für das ein andauerndes Stoßen, Schlagen, Ringen und Beißen notwendig wurde. Leno genoss diese Anstrengungen, genoss deren Sinnhaftigkeit. Er arbeitete auf ein Ziel hin, tat das Richtige. Und Abends legte er sich mit einer inneren Ruhe schlafen. Er lebte seine Tage in dem Gefühl, das Beste für sich selbst zu tun. Und das fühlte sich an wie alles, was er je gewollt hatte. Sein Schlaf war tief und erholsam und nie wälzte er sich über einem Gedanken Nachts hin und her.
\chapter{Essen}
An einem milden Sonntagmorgen wurde Leno von der Sonne geweckt, weil sein Vorhang nicht vor dem Fenster hing. Er verzichtete immer häufiger darauf, ihn Abends zuzuziehen, damit er nicht so lange schlief. Ein paar Mal blinzelte er der Sonne entgegen, dann setzte er sich noch etwas benommen im Bett auf und nahm kurz sein Mobiltelefon in die Hand. Er tippte ein bisschen auf dessen Bildschirm umher und ließ sich über ein paar Dinge benachrichtigen, bis er schließlich aufstand und in die Küche ging. Dort befüllte er den Wasserkocher und stellte eine Schüssel und eine Tasse auf seinen Küchentisch. Er gab Kaffeepulver in die Tasse und Haferflocken in die Schüssel, goß den Kaffee mit heißem Wasser auf, verteilte Milch zwischen Schüssel und Tasse. Aus der Gefriertruhe nahm er sich einen Beutel Eis und fixierte ihn mit einem Handtuch an seiner rechten Schulter. Sie machte ihm in letzter Zeit verstärkt Probleme. Zunächst hatte er noch darauf gewartet, dass der Schmerz von selbst vergehen würde, nachdem das aber seit einer ganzen Woche nicht passiert war, versuchte er jetzt so gut es ging sie zu kühlen und zu schonen. Er musste mehrmals am Knoten des Handtuchs herumkorrigieren, bis das Eis einigermaßen fest saß. Dann, als er mit seiner Selbstbehandlung zufrieden war, holte er sein Mobiltelefon vom Bett, legte die Beine auf dem Küchentisch ab und stellte die Schüssel in seinen Schoß. Er trank den Kaffee aus. Er aß die Haferflocken auf. Und er blieb sitzen, lange als er schon zu Ende gefrühstückt hatte, saß mit hochgelegten Beinen am Tisch und surfte auf seinem Mobiltelefon. Nach einiger Zeit schob sich eine neue Nachricht Anthonys von oben auf das Display: „Na Leno, wie geht's so? Lust heute mal wieder was zu machen?“. Er las die Nachricht und nahm die Beine vom Tisch. Er freute sich sehr darüber, dass Anthony sich gemeldet hatte. Sie tauschten einige Nachrichten aus und verabredeten sich zum Essen. Nachdem er diese Aktivität in seinen Tag geplant hatte, verharrte Leno noch eine Weile auf dem Stuhl und wog ab, wo er sein Training in den Tag einordnen sollte, dann stand er auf und packte seinen Rucksack. Das Restaurant, in das sie gehen wollten, lag ein gutes Stück entfernt und er würde bald loslaufen müssen, um es mit absoluter Sicherheit vermeiden zu können spät dran zu sein. Also entschied Leno, dass es sich nicht lohnte noch 10 Minuten zuhause totzuschlagen, und machte sich sofort auf den Weg. Während er die Treppen nach unten trabte zog er seinen Rucksack von seinem Rücken nach vorne auf seinen Bauch, öffnete ihn einen Spalt weit und kramte darin nach seinen Kopfhörern. Sie waren arg verknotet und er brauchte einen guten Teil der Strecke, bis er sie benutzen konnte. Es fühlte sich fast wie eine Belohnung an, als er sie in die Ohren steckte und einen Podcast spielen ließ.
\vspace{0.5em} \\
Als Leno am Restaurant ankam, war es noch etwas zu früh und Anthony nicht da, also suchte er sich einen abseits gelegenen Tisch aus, setzte sich, bestellte eine Cola und las den Artikel zu Ende, mit dem er Vormittags begonnen hatte. Manchmal ließ er sich dabei von einer Benachrichtigung ablenken und wischte durch ein paar kurze Videos. Irgendwann später tauchte der schmale Anthony im Türrahmen auf, setzte sich zu ihm, und es dauerte nicht lange bis Leno sich nicht mehr zurückhalten konnte und das Thema der Unterhaltung auf das einzige lenken musste, das ihn gerade beschäftigte.\\
„Mit dem Schwimmen läuft es zur Zeit echt gut, Anthony.“\\
„Ja?“, antwortete der Freund.\\
„Ja. Ich bin in letzter Zeit fast jeden Tag schwimmen und habe angefangen meine Leistungen zu notieren. Habe mich auch bisschen informiert und mache mittlerweile regelmäßig auch Intervalltraining, das hat echt viel gebracht.“\\
„Intervalltraining? Das ist aber auch nicht mehr gesund, wenn mans zu sehr übertreibt, oder, sag?“\\
„Ich weiß auch nicht. Ich bin irgendwie voll drin. Ich kann mich jedes Training voll reinhauen. Es fühlt sich einfach gut an. Ich notiere ja auch die Zeiten und sehe, dass ich besser werde, aber das merke ich beim Schwimmen fast noch deutlicher. Ich liege ganz anders im Wasser, es strömt ganz anders an mir vorbei, meine Muskeln bewegen sich ganz anders.", Leno lachte, "Wenn ich nicht aufpasse, wachsen mir bald noch Schwimmhäute!“\\
„Ja, hört sich doch gut an!“, Anthony hielt kurz inne und musterte Leno sorgfältig, „Bist auch echt breit geworden, muss man sagen. So gut trainiert warst du noch nie. Kommt bei den Ladies bestimmt auch gut an.“. Die Augen der Freunde trafen sich und Leno drehte sich errötend zur Seite. Er hätte lügen müssen, um zu behaupten, ihm selbst sei das nicht aufgefallen, und er wusste nicht recht zu antworten.\\
„Ich fang jetzt auch an, mehr Sport zu treiben.“, warf Anthony ein: „Ich steh einfach jeden Tag um 5 Uhr auf und geh noch vor der Arbeit ins Fitnessstudio. Dann mach ich Klimmzüge, Curls und Bankdrücken. Gute Übungen, oder, sag? Und ich kaufe mir jedes Pulver, das in einem Sportfachgeschäft dieser Stadt verkauft wird.“ Er lachte. „Nein, Quatsch, wenn, dann such ich mir schon ein richtiges Programm und kauf mir auch so ein Notizbuch wie du. Dann ess ich auch vernünftig." Anthony schob den Ärmel seines T-Shirts nach oben und spannte die Muskeln seines dünnen Arms an. „Wie lange meinst du braucht es, bis der so dick ist wie der von Schwarzenegger, sag?“ Anthony war sichtlich belustigt von der Vorstellung. Es machte ihm Spaß darüber nachzudenken, dass die Möglichkeit durchaus bestand, seinen Worten Taten folgen zu lassen. Leno begegnete den Aussagen seines Freundes eher mit Skepsis. So sehr er Anthony schätzte, er glaubte nicht an sie. Anthony war niemand, der den schweren Weg gehen konnte. Er könnte dann ins Fitnessstudio gehen, wenn er ausgeruht war, satt und entspannt. Er könnte ins Fitnessstudio gehen, wenn er einem Mädchen imponieren wollte, wenn er am Tag zuvor ein Footballspiel angesehen hatte, oder er einen neuen Kollegen mit gut austrainierten Armen bekommen würde. Wenn er aber ins Fitnessstudio gehen müsste aus keinem anderen Grund als dem, dass er sich gegenüber sich selbst dazu verpflichtet hatte, dann würde er einen besseren Grund finden, um zuhause zu bleiben. So dachte Leno und beobachtete Anthony dabei, wie er die Zigarettenschachtel aus seiner Hosentasche zog. Leno streckte ihm seine geöffnete Hand entgegen. Anthony legte eine Zigarette hinein und nahm sich selbst eine, wollte sie sich anzünden, aber stoppte. „Ich bin jetzt ja Sportler, fast vergessen!“, lachte er Leno an und fragte ihn, ob er seine trotzdem rauchen wollte. Dieser nahm sich das Feuer, zündete die Zigarette an und nahm einen Zug. „Eine ist ok.“, meinte er lachend zu Anthony.\\
Die beiden unterhielten sich noch eine Weile, dann zahlten sie. Leno stand auf und warf sich seinen Rucksack über die Schulter. Er begleitete Anthony bis zur Bushaltestelle und setzte seinen Weg in Richtung des Schwimmbads fort. Er dachte an seinen heutigen Trainingsplan, daran was er heute Abend essen würde und wann er ins Bett gehen sollte, um genug Schlaf zu bekommen. Zur gleichen Zeit überkam den auf den Bus wartenden Anthony die Lust auf eine Zigarette, die er ohne weiter nachzudenken aus der Schachtel nahm und anzündete.
\chapter{Anmeldung}
Leno war gerade im Schwimmbad angekommen, da befand er sich in Gedanken schon im Wasser und zog seine Bahnen, las die Uhr, schrieb Ziffern in sein Notizbuch. Sein Rucksack hing ihm tief seitlich, zum Abwurf bereit auf der linken Schulter und die letzten Schritte zum Schwimmbecken verwoben sich mit den ersten Armzügen im Wasser. Da rissen ihn ein paar in seine Richtung gerufenen Worte abrupt aus seinem Rhythmus und veranlassten ihn dazu irritiert stehen zu bleiben.\\
„Hey Leno, die Anmeldung für die Wettkämpfe vom Stadt-schwimmfest sind offen!“, schallte es aus dem Kassiererhaus. Leno blieb stehen. Langsam schob sich der Satz durch seine Ohren bis in sein Gehirn, wo die Neuronen begannen wild hin und her zu feuern. Es brauchte einige Sekunden bis er sich gesammelt hatte, so fest war er in seiner Routine verankert gewesen. Endlich nahm er die Hände aus den Trägern des Rucksacks. Dann drehte er sich, versuchte so cool wie möglich dabei auszusehen, und ging in Richtung der Stimme. Es war Nacar gewesen, der ihn gerufen hatte. Der Kassier blickte durch das gekippte Fenster seines kleinen Kabuffs zu ihm und deutete ihm durch energisches Winken an, um das Häuschen herum zum Eingang zu kommen. Leno fiel auf, dass er Nacar noch nie mehr als zwei Worte am Stück und niemals so laut hatte sprechen hören. Er schritt vorsichtig um das Mauerwerk und als er vor der Tür angekommen war, traute er sich nicht richtig einzutreten und lugte vorerst zögerlich mit dem Kopf um die Ecke. Nacar schien sein Zögern zu bemerken. "Komm ruhig herein.", sagte er, und Leno sah, dass er ihn schon nicht mehr beobachtete, sondern geschäftig an seinem Schreibtisch saß und einen Stempel auf einige Blätter Papier drückte. Also trat er ein. Das Häuschen war so klein, dass er sich kaum anders drehen konnte, als zur Wand links vom Eingang. An dieser hingen mehrere Zettel, und auf jedem davon war eine Liste vorgedruckt, die je eine Distanz, ein Geschlecht und einen Schwimmstil als Überschrift hatte. Unter den Überschriften gab es viele leere Spalten, in die vereinzelt schon Namen gekritzelt worden waren. „Ein Stift liegt auf dem Medizinschrank.“, brummte Nacar, ohne von seiner Arbeit aufzusehen. Leno nahm den Stift und zeichnete über alle Distanzen für die männlichen Wettbewerbe im Freistil seinen Namen in die oberste freie Zeile der vorgedruckten Tabelle. Dann machte er einen Schritt zurück, kontrollierte noch einmal ob er keinen Zettel vergessen hatte und legte den Stift zurück. Beim Verlassen des Häuschens rief Nacar ihm nach:\\
„Du warst jetzt eine lange Zeit sehr fleißig beim Training, Leno, du schneidest bestimmt gut ab. Viel Spaß heute!“\\
Leno drehte sich mit einem Lächeln zu ihm um und bedankte sich. Er sah ins Gesicht des Kassiers, das nur seines musterte anstatt selbst etwas auszudrücken, dann fädelte er sich wieder in seinen gewohnten Ablauf ein. Er lief ein paar schnelle Schritte zur ersten Sitzbank auf der Längsseite des Beckens, ließ seinen Rucksack von der Schulter gleiten und warf ihn im Vorbeigehen neben sie. Dann duschte er sich ab und schwamm. Erneut stoßend, erneut schlagend, erneut ringend und erneut beißend. Sein Körper glitt durchs Wasser, sein Geist aber war dem Wasser fern. War dem Wasser fern, dem Schwimmbecken, dem Boden darunter und allem was eine greifbare Gestalt annehmen konnte. Anstatt im Wasser zu sein und seinem Fluss zuzuhören durchfühlten Lenos Gedanken seine Erwartungen an den Wettkampf. Er malte sich einen Moment aus, in dem er sich nach dem Anschlag umdreht, am Beckenrand festhält und alle Kontrahent noch schwimmen sieht. Er malte sich einen Moment aus, in dem er auf der Spitze eines Podests steht und seinen Namen durch einen Lautsprecher schallen hört. Er malte sich einen Moment aus in dem ihm ein Mädchen zulächelt nach seinem Sieg.
\vspace{0.5em} \\
Er machte ein paar Armzüge, ein paar Beinschläge, kam am Beckenrand an und hielt sich an ihm fest. Erschöpfung ergriff ihn. Die Tage des harten Trainings steckten in seinen Muskeln und Knochen, sein Körper fühlte sich ausgezehrt an und seine rechte Schulter hörte nicht damit auf einen dumpfen Schmerz auszusenden. Der Wettkampf stand direkt bevor, ein Tag, an dem nicht nur in seinem Notizbuch abgerechnet werden würde. Leno hatte für diesen Tag trainiert und musste an ihm an der absoluten Höhe seiner Leistungsfähigkeit sein. Er musste den Muskeln erlauben sich zu regenerieren. Er ließ den Beckenrand los, atmete ein, sank ins Wasser und stieß sich mit den Füßen ab. Er begann wieder zu schwimmen, hielt sich aber zurück, schwamm langsam, bewegte seine Arme mit weniger Kraft. Bewegte seinen Körper auf unnatürliche, gezwungene Weise, bis er sich erlaubte, das Training als ausreichend zu empfinden. Dann stieg er aus dem Wasser, trocknete sich ab, schwang sich seinen Rucksack auf den Rücken und verabschiedete sich von Nacar. Auf dem Weg nach Hause machte er einen Umweg über eine Apotheke und kaufte sich ein Schmerzmittel.
\chapter{Turnier}

Lenos Augen bewegten sich sprunghaft unter seinen Lidern hin und her während er sich unruhig im Bett herum wälzte. Das Zimmer war dunkler als sonst. Er hatte gestern vor dem Schlafengehen den Vorhang nicht nur zugezogen, sondern zusätzlich ein Stück Pappe besorgt und es passend zugeschnitten, um es vor dem Fenster anzubringen und die Sonne noch effektiver zu verdecken. Dadurch wollte er sicherstellen, dass er für den Wettkampf so gut erholt war wie möglich. Aber sein Körper spielte nicht mit, drehte sich die ganze Nacht im Schlaf, spannte sich an, gab leise Laute von sich, zuckte mit den Gesichtsmuskeln. Als arbeitete er sich mühsam durch einen allzu bewegten Traum. Er wühlte noch unruhig herum, als die Sonne es langsam schaffte durch die Lücken zwischen Pappe und Fenster ein bisschen Licht ins Zimmer zu kämpfen. Einen kleinen Streifen Helligkeit, der sich auf die Decke, den Boden und den schlafenden Leno warf, während der sein linkes Bein noch einmal anzog, sich drehte, das Rechte streckte. Dann, endlich, begann sein Mobiltelefon wie von alleine schwach zu leuchten und melodisch zu klingeln. Leno stieß einen undefinierbaren Laut aus. Das Klingeln wurde zunehmend lauter, fordernder, und irgendwann schoben sich die Lider von seinen Augen und er war wach. Und weil er einen Moment brauchte um in der Wirklichkeit anzukommen, blickten sie kurz geradeaus wie ins Leere, bevor er nach dem Mobiltelefon griff, sich senkrecht im Bett aufsetzte und den Wecker ausmachte.
\vspace{0.5em} \\
Er sah seine Benachrichtigungen durch, dann ging er in die Küche. Er kochte Wasser auf, stellte eine Schüssel und eine Tasse auf den Küchentisch. Holte das Kaffeepulver, Milch und Haferflocken, gab sie in Tasse und Schüssel. Er aß auf, kontrollierte nochmal den Inhalt des Rucksacks, den er am Abend zuvor schon vorbereitet hatte. Handtuch. Badehose. Wasserflasche. Schwimmbrille. Er öffnete die Packung mit dem Schmerzmittel, das er gestern gekauft hatte. Er schmierte sich eine gute Menge davon auf die Schulter. Dann warf er die Tube mit in den Rucksack, zog ihn sich auf den Rücken und ging los. Durch das Treppenhaus und die Haustür hindurch, erst Richtung Stadt, bis er die Abzweigung zum Schwimmbad erreichte. Dort bog er ab. Mike fuhr mit Fahrrad an ihm vorbei und grüßte ihn. Schon von einiger Entfernung sah man eine große Ansammlung Menschen vor dem Schwimmbad stehen. Trotzdem machte Leno nichts anders als die Tage zuvor und gab sein bestes, die Andersartigkeiten zu ignorieren. Er wich ein paar Menschengruppen aus und kam am vergitterten Tor vor dem Gelände an, das sonst so penibel von Nacar bewacht wurde. Heute stand es weit geöffnet, und niemand interessierte sich für Leno, als er das Gelände betrat. Er beeilte sich zu seiner Bank zu kommen. Zum Glück war sie noch frei, und Leno konnte sich hinsetzen und ausruhen. Er atmete ein paar mal durch. Er sah sich etwas um. An vielen Stellen im Schwimmbad verteilt standen und saßen die Leute in Gruppen herum und unterhielten sich. Mike, Michael und Mark waren an der Rückseite des Kassiererhauses versammelt. Vor der Umkleidekabine, an deren Außenwand ein großer Spiegel befestigt war, stand ebenfalls eine größere Menge in einem Halbkreis angeordnet. Die Leute, die Leno dort sah, schienen sich nicht zu unterhalten, sondern schauten nur konzentriert auf den Spiegel, was ihm etwas komisch vorkam. Er drehte seinen Kopf wieder nach vorne und blickte aufs Wasser. Dann öffnete er seinen Rucksack und nahm das Schmerzmittel heraus. Er schmierte erneut großzügig seine Schulter ein. Dann nahm er sich nochmal die Zeit ein paar mal durchzuatmen, bevor er aufstand um zur Umkleidekabine zu gehen. Als er ankam, suchte er eine Lücke im Halbkreis, um selbst einen Blick auf den Spiegel zu werfen zu können, der so interessant zu sein schien. Die Teilnehmer und Startzeiten der Wettbewerbe waren ausgedruckt auf Zettel auf den Spiegel geklebt worden. Während er sie sich einprägte, sah er die Reflexion der Ränder seines Gesichts und Haaren und Oberkörper in den Lücken zwischen den aufgeklebten Zetteln. Er merkte sich seine erste Startzeit, und anstatt sich dann umziehen zu gehen, ging er nochmal aus dem Schwimmbad heraus und machte eine Runde um den Block, um die restlichen Minuten bis zu seinem Start zu überbrücken. Als er wieder zurückkam war seine Bank immer noch frei, er stelle seinen Rucksack ab. Dann ging er zur anderen Seite des Beckens, stellte sich in einer Schlange an, sagte einer Person mit einer Pfeife und einer Liste seinen Namen, und stieg auf den ihm zugewiesenen Startblock.
\vspace{0.5em} \\
Leno stand auf dem Startblock und wartete. Er hatte die Knie leicht gebeugt, sein Oberkörper lehnte weit nach vorn. Die Muskeln seiner Beine befanden sich in Antizipation des jederzeit möglichen Pfiffs mal unter Vorspannung, mal in einer lockeren Streckung. Seine Arme hingen entspannt nach unten, wie er es in unzähligen Videos auf YouTube gesehen hatte. Er war nervös gewesen, als er auf den Block gestiegen war. Jetzt war er nur noch voll konzentriert. Auch dieser Moment war einer derer gewesen, die er sich in den Tagen vor dem Wettkampf vorgestellt hatte. Die aufgeladene Ruhe. Die perfekt glatte Oberfläche des Wassers. Die Einsamkeit auf dem Startblock. Und manche Dinge konnten Wirklichkeit werden, ungeachtet dessen, wie oft man sie sich vorstellte. Er lehnte seinen Oberkörper noch weiter nach vorne, bis er Gefahr lief, von der Schwerkraft vom Startblock ins Becken gekippt zu werden. An den seitlichen Rändern seines Sichtfelds konnte er die Köpfe seiner Kontrahenten erkennen. Er fragte sich, für was sie wohl hier auf dem Startblock standen. Was und wie viel sie geopfert hatten. Ob sie sich auch eine ihrer Schultern mit Schmerzsalbe einschmierten. Und er erinnerte sich an seinen Gang zum Schwimmbad, der heute ein anderer gewesen war. Der sich nicht mehr angefühlt hatte wie ein Teil eines längeren, größeren Weges. Auf dem es sich nicht mehr so angefühlt hatte, als würde Leno einem goldenen Faden folgen.\\
Dann hörte er den Pfiff, sprang, schwamm, und erreichte die gegenüberliegenden Beckenwand. Es war vorbei. Er hielt sich fest, drehte sich um. Die eben noch flache Oberfläche des Wassers war von Wellen und Wirbeln aufgewühlt. Aber immer noch überdeckt von der gleichen Stille wie der, als er auf dem Startblock stand.\\
Und Leno trieb im Wasser und wusste nicht, was jetzt der nächste Schritt war. Er blickte nach links und rechts, sah andere Menschen im Wasser, manche unterhielten sich, manche trieben wie er, sich umschauend und am Beckenrand festhaltend. Irgendwann verkündete die Wettkampfleitung Platzierungen und Zeiten. Dann begannen die Wettkämpfer das Schwimmbecken zu verlassen, begleitet vom Applaus der Zuschauer. Leno tauchte unter den Absperrbändern hindurch hin zur Leiter am seitlichen Beckenrand, stieg aus dem Wasser und trocknete sich ab. Er warf einen kurzen Blick auf den ausgedruckten Wettkampfplan, um den Überblick über seinen nächsten Start zu behalten und reihte sich dann zusammen mit Michael und Mark in die Gruppe der Vereinsschwimmer, die neben dem Kassiererhaus einen Kreis gebildet hatten, ein. Sie klatschten die ihnen entgegengestreckten Hände ab, es wurde ihnen gratuliert, besonders Michael, der das Rennen gewonnen hatte. Die Stimmung war locker und vergnügt und Leno stand einfach ein bisschen mit in der Gruppe herum, bis jemand das Wort an ihn richtete:\\
„Leno wie war es eigentlich für dich? Das war dein erster Wettkampf, oder?“\\
Ja, wie war es für Leno gewesen? Es war die Frage, die er sich selbst auch stellte, seit er am Beckenrand angeschlagen hatte. Er wollte es sich nicht leicht machen und behaupten, es war zu schnell vorbei gewesen, um sich ein Bild und eine Meinung davon gemacht zu haben. Das wäre dem nicht gerecht geworden. Es war passiert. Er war gesprungen, er war getaucht, hatte gekämpft, gekämpft um jeden Zentimeter, um jede Handvoll Wasser und gegen jeden Schwimmer auf den anderen Bahnen. Er hatte die Präsenz seiner Kontrahenten gespürt. Er hatte den Wettkampf mit ihnen gefühlt. Er hatte sich mit ihnen gemessen.\\
Und wie oft er auch in Gedanken das Rennen geschwommen war, so unvorbereitet auf dieses Erlebnis war er gewesen. Niemals hatte er im von den anderen Schwimmern aufgewühlten Wasser gewackelt. Hatte er noch so oft in Gedanken Luft geholt, niemals hatte er dabei den Schwimmer der benachbarten Bahn auf gleicher Höhe gesehen. So viele kräftige Armzüge hatte er gezogen, so viele Beinschläge geschlagen, niemals war er auf einem Startblock gestanden und auf einen Pfiff hin so stark abgesprungen wie er konnte.\\
Er sah in die Runde, machte eine Geste, begegnete der Frage, wie es für ihn gewesen sei mit einer Ausführung seines Trainingsprogramms, der Qualität seines Schlafs und der Art seines Frühstücks. Auf die Nachfrage, warum er denn beim Start so schlecht weggekommen sei, antwortete er wahrheitsgemäß, dass er den Start vom Block nie trainiert hatte und erntete die Erheiterung der Gruppe. Und er wäre selbst heiter gewesen, wenn er nicht sein Vertrauen in sich so in Frage stellen müsste. Jeden Tag hatte er seinen Rucksack gepackt, war zum Schwimmbad gelaufen und hatte trainiert. Und hatte gedacht, diese Tage würden ein Gewicht haben für den heutigen. Er hatte Wissen und Kraft angesammelt und war dabei einer roten Linie gefolgt. Und wie er jetzt feststellen musste, hatte diese Linie nur ins Nichts geführt. Hatte ins Nichts geführt und war selbst nicht einmal real gewesen. Sie war, wie er feststellen musste, von ihm nur in so satter Farbe gemalt worden, dass er sich sicher war, sie wäre schon immer da gewesen.\\
Leno hörte den Späßen der anderen noch eine Weile zu und verschwand dann Richtung Ausschank. Er bewegte sich durch verschiedene Menschengruppen hindurch, lehnte sich an den behelfsmäßig aufgebauten Tresen und bestellte sich ein Getränk. Noch am Tresen nahm er einen großen Schluck und versöhnte sich mit dem milden, sonnigen Tag. Dann steckte er Wechselgeld und Pfandmarke ein und machte sich auf den Weg zurück zum Schwimmbecken, um sich auf sein nächstes Rennen über die längere Distanz vorzubereiten. Als er an einem großen, schönen Mädchen vorbeiging, drehte sie sich um und sprach Leno an.\\
„Hey, bist du nicht beim ersten Start der Männer mit geschwommen? Hast du denn jetzt schon Feierabend, oder denkst du das Bier hilft?“, sie lachte ihn an „Wir Zuschauer erwarten schon Höchstleistungen!“\\
Leno sah sie an. Er wurde sich seiner Körpersprache bewusst, richtete sich auf, schob seine Brust nach vorne und ließ seine Schultern sinken. Er sagte, dass am heutigen Tag vermutlich nichts mehr helfe, er aber für alle Distanzen eingetragen sei. Außerdem sei das Bier alkoholfrei. Das Mädchen strich sich eine Haarsträhne aus dem Gesicht und lachte:\\
„Wenn das so ist, dann trink mal schnell aus und schwimm dich wieder warm! Vielleicht sieht man sich ja später nochmal!“\\
Leno antwortete, dass man sich ja tatsächlich eventuell noch einmal sehen könnte und ging weiter, den Kopf erhoben, zielstrebigen Schrittes, in jede Bewegung seines Körpers so viel Selbstbewusstsein legend, wie es ihm möglich war. Er warf einen Blick auf seinen üblichen Platz auf der Sitzbank und stellte erfreut fest, dass er frei war. Er erreichte die Bank, ließ sich nieder und musterte die Wasseroberfläche des Schwimmbeckens. Ruhig lag sie da, gefühllos, urteilsfrei. Ahnungslos war das Wasser von Pfiffen und Schwimmern und Zahlen, mit denen man Distanz und Zeit bemaß, die dann einem Menschen zugeordnet wurden. Ahnungslos war es genauso von den Zahlen, die Leno monatelang in seinem Notizbuch vermerkt hatte. Gleichgültig war ihm das Treiben um es herum, die Musik, die Geschäftigkeit. Es existierte in tiefem Einverständnis. So saß Leno grübelnd auf der Sitzbank, sich ärgernd, dass das Leben dem Wasser so unähnlich sein musste. Es ärgerte Leno, dass das Leben urteilte, dass es hektisch war, dass es abrechnete. Und er fand es nur fair, so auch über das Leben zu urteilen, zum einen über das Leben an sich, zum anderen über sein eigenes, dessen Flamme in ihm brannte, begehrend, fordernd, unzufrieden.\\
Dann, als es wieder Zeit wurde, nahm er sein Bier mit unter die Dusche, trank es leer, ging zurück zum Schwimmbecken und stieg erneut auf den Startblock. Er beugte seine Knie leicht, lehnte seinen Oberkörper nach vorne und wartete. Und als der Pfiff ertönte sprang er ab, wartete auf den Aufprall und tauchte, war zufrieden mit seinem Tauchen und schwamm. Er wurde von den Wellen der Kontrahenten umhergeschleudert und versuchte dabei, seine Arme so rhythmisch und kraftvoll wie möglich durchs Wasser zu ziehen. Er schlug vor dem Beckenrand ein letztes Mal mit den Beinen, zog den Kopf zur Rollwende nach unten, ließ seinen Körper im Wasser rotieren, und drückte sich anschließend wieder explosiv von der Wand ab. Die Zähne aufeinanderbeißend tauchte er auf und trieb seinen Körper unnachgiebig an. Er fühlte den Schmerz in den Muskeln, er fühlte das Wasser an ihm vorbeiströmen, schnell, unruhig. Und dann fühlte er seine Hand an der Beckenwand anstoßen, ließ alle Spannung aus seinem Körper entweichen und ihn im Wasser treiben. Seine Hand griff nach dem Beckenrand und sein Atem ging schnell und stoßend. Sein Körper schaukelte wie ein Stück Treibholz im Wasser und er sah sich interessiert um, sah sich die Gesichter der anderen Schwimmer an, sah durch die Reihen der Zuschauer, und sah der Wettkampfleitung dabei zu, wie sie die Ergebnisse zusammentrug. Nach kurzer Zeit des geschäftigen Notierens nahm jemand das Mikrofon in die Hand und verkündete die Platzierungen. Leno war diesmal einer der letzten gewesen, aber er machte sich nichts mehr daraus. Er hatte abgeschlossen mit seiner Vorstellung des Wettkampfs, von der Demonstration seiner Arbeit, seiner Kraft, seiner Hingabe. Er ließ den Tag nach belieben verfahren, ließ sich von hier zu dort treiben, fand sich noch einige weitere Male auf dem Startblock stehend wieder, den Oberkörper vorbeugend und auf einen Pfiff wartend. Dann nahm Leno den Pfiff wie er eben kam und nahm das Wettkämpfen wie es eben kam und warf sich mit all seiner Entschlossenheit, all seinem Ehrgeiz und all seinem Eifer in die Waagschale. Und als Zeiten und Platzierungen des letzten Rennens verkündet wurden, war es Leno, der unter den Schwimmern am lautesten Jubelte, weil er gewonnen hatte. Er freute sich ehrlich und aufrichtig, aber die letztendliche Bedeutung, den Charakter eines Ziels, hatte der Sieg für ihn verloren.
\chapter{Julia}
Dann war es Abend geworden, die Sonne stand tief und Leno saß mit Mike und Malte auf ein paar Plastikstühlen vor dem bereits geschlossenem Ausschank. Schon über längere Zeit schwenkte sein Blick immer wieder zu dem Mädchen, mit dem er nach dem ersten Rennen ein paar Worte gewechselt hatte. Sie saß in einer Dreiergruppe am Rand des Schwimmbeckens. Dort ließen ihre Freunde — ein großes Mädchen mit breiten Schultern und ein zierlicher Junge — und sie ihre Füße ins Wasser hängen. Und nachdem Mike und Malte sich von Leno verabschiedet hatten und den Heimweg antraten, wartete er noch eine kurze Weile, überlegte, und ging dann zu der Dreiergruppe hinüber. Die am Beckenrand Sitzenden bemerkten ihn erst, als er fast schon neben ihnen stand. Während sie sich noch überrascht zu ihm umdrehten, stieß er hastig die Frage hervor, ob er sich vielleicht zu ihnen gesellen durfte, und sie bejahten zwar etwas verwundert, aber durchaus freundlich sein Gesuch. Und Leno ging neben ihnen in die Hocke, ließ seine Füße ebenfalls ins Wasser hinab und suchte dann nervös nach einer bequemen Sitzposition. Irgendwann war er zumindest einigermaßen zufrieden, und er stellte die in sorgfältiger Planung zurechtgelegte Frage in Richtung des Mädchens, das er schon kennen gelernt hatte:\\
„Und? Warst du mit den Höchstleistungen der Spitzenathleten zufrieden?“\\
Das Mädchen lachte. Sie entgegnete:\\
„Wenn ich ehrlich bin habe ich von den weiteren Rennen gar nicht so viel mitbekommen. Wir waren eher mit uns selbst beschäftigt. Wie lief es denn noch für dich?“\\
Leno lachte ebenfalls: „Aha! Und ich habe mich extra angestrengt!“, und dann erzählte er von seinem großartigen Triumph im letzten Wettkampf, wusste aber nicht genau, wo er den Ton seiner Geschichte zwischen Ironie und Ernsthaftigkeit ansiedeln sollte. Aber zum Glück waren die Drei wirklich nett, machten es ihm leicht, und gingen mit seiner Geschichte nicht allzu streng ins Gericht. Es schien so, dass alle vier Beteiligten von einer guten Laune beseelt waren und so konnte sich mühelos ein kurzweiliges Gespräch entwickeln. Und während sie redeten begann die Sonne sich langsam hinter den Horizont zu schieben und der Tag sich dem Ende zuzuneigen. Leno fühlte sich wirklich wohl in der Gruppe. Das Mädchen hieß Julia, schwarze Haare wehten über ihr Gesicht, zweimal berührte sie Leno während des Gesprächs am Oberschenkel und einmal an seiner Schulter. Und als ihm ein Scherz besonders gut gelang lehnte sie beim Lachen den Kopf in den Nacken, den Körper zurück und bot einen Anblick von Jugend und Schönheit, der jeden Dichter der Romantik dazu veranlasst hätte, seine gesammelten Werke auf einem Scheiterhaufen zu verbrennen, aus Scham dieser niemals gerecht werden zu können. Und so gern Leno den Rest seines Lebens neben ihr am Beckenrand sitzend verbracht hätte, so unvermeidlich war es doch, dass sie irgendwann ihre Füße aus dem Wasser nahm und ihren Abschied verkündete. Sie wandte sich an ihre Begleiter, die sich als Tobi und Sophie vorgestellt hatten, und als klar wurde, dass die beiden noch etwas bleiben wollten, verabschiedete Julia sich, ging und war weg.
\vspace{0.5em} \\
Leno blieb, um zumindest ein klein wenig zu verschleiern, dass er nur wegen ihr zu der Gruppe gestoßen war, noch eine Zeit lang bei Tobi und Sophie sitzen und führte das Gespräch fort. Dann, kurz bevor auch er sich verabschieden wollte, richtete Sophie, welche zuvor sehr zurückhaltend und ruhig gewesen war, das Wort an ihn:\\
„Du, Leno, Glückwunsch nochmal zu deinem Sieg im letzten Rennen. Ich hab dich zufällig auf dem Startblock beobachtet. Du bist noch nicht so oft im Sprung gestartet, oder?“\\
Ertappt sah Leno sie an. Zustimmend bewegte er den Kopf. Er war sehr gespannt, was jetzt kommen würde. Sophie fuhr fort:\\
„Ich will wirklich kein Klugscheißer sein, aber das Wichtigste vor dem Start ist einfach, entspannt zu bleiben. Man sieht dir an, dass du beim Versuch alles richtig zu machen verkrampfst. Dadurch atmest du nicht komplett durch und die Luft fehlt dir auf den ersten Bahnen. Bleib erstmal einfach entspannt auf dem Block. Um deinen Sprung zu perfektionieren bleibt dir noch genug Zeit.“
\vspace{0.5em} \\
Leno saugte ihre Worte ein, ging in sich, und versuchte sich zurückzuerinnern an die von ihm heute ausgeführten Startsprünge. Was ihm in den Sinn kam war, wie er hochkonzentriert die Platzierung seiner Füße justieren und jeden Winkel seiner Gelenke erfühlen wollte, während er mit spitzen Ohren den Ansagen des Wettkampfleiters gelauscht hatte. Er konnte auch jetzt, zurückblickend, nicht feststellen, dass er besonders angespannt oder verkrampft gewesen wäre. Dann ging er weiter durch sein Gedächtnis und spulte erst das Eintauchen ins Wasser, dann die ersten Male des Luftholens vor seinem inneren Auge ab. Und tatsächlich konnte er sich daran erinnern, dass er einen großen Teil der ersten Bahnen vor Allem mit dem Atmen beschäftigt gewesen war. Gierig hatte er sich abmühen müssen, seinen Mund weit aus dem Wasser zu heben für ausholende, arhythmische, tiefe Atemzüge, anstatt sich darauf zu fokussieren, seinen Körper nach vorne zu ziehen. Und mehr zu sich selbst als um Sophie zu antworten hörte er sich sagen:
\vspace{0.5em} \\
„...du hast Recht.“
\vspace{0.5em} \\
„Versteh das bitte nicht als Kritik!“, relativierte Sophie ihren Hinweis schnell, unsicher, ob sie mit ihrem unaufgeforderten Ratschlag eine Grenze überschritten hatte. Hastig entgegnete Leno:
\vspace{0.5em} \\
„Nein... Nein! Vielen Dank! Das war mir gar nicht bewusst, aber jetzt wo du's sagst, fällt es mir richtig auf.“\\
Dann, nachdem er einen Moment gebraucht hatte um seine Überraschung über Sophies Kurzintervention in sein Schwimmen abzuschütteln, ergriff Leno ein großes Interesse an ihren Beweggründen, und er frage sie gründlich aus: Darüber ob und wie viel sie Schwamm, wieso, und was sie darüber gelernt hatte. Und Sophie wog die Situation kurz ab, überwand etwas in sich, und begann von ihrer Zeit als Sportsoldatin bei der Armee zu erzählen. Sie brauchte ein bisschen, bis ihre Sätze flüssig wurden, und dann teilte sie ihre Geschichte, sachlich, ohne Stolz, sogar gänzlich ohne Urteil, erst oberflächlich, dann detaillierter, als sie merkte wie gebannt Leno zuhörte. Und auch Tobi, der anscheinend selbst noch nicht so viel über diesen Teil von Sophie gewusst hatte, war nun zusehends aufmerksamer. Sie erzählte von strikten Trainingsplänen, von frühen Morgen, von großer Enthaltsamkeit. Sie erzählte von langen, harten Trainingseinheiten im Wasser und erbarmungslosem Athletiktraining, von stundenlangem Ausrollen und Dehnen nach dem Training. Von Tagen mit drei Einheiten im Wasser und ganzen Wochen, in denen sie kein Schwimmbecken gesehen und ausschließlich mit Lang-, und Kurzhanteln trainiert hatte. Sie erzählte von großem Enthusiasmus, der mit der Zeit durch eiserne Disziplin ersetzt worden war, und schließlich erzählte sie von Grenzen, an die sie immer wieder gebracht worden war und die eines Tages überschritten wurden. \\
Und sobald sie mit ihren Ausführungen fertig war, sprudelte aus Leno die Frage heraus, warum sie nicht selbst am Wettkampf teilgenommen hatte. Sie lachte. Kurz sah sie ihn intensiv an, so als ob sie ihn nur allzu gut verstand. Dann drehte sie ihren Kopf nach vorne und ihr Blick verfing sich einen Moment lang in der Wasseroberfläche zu ihren Füßen, so als musterte sie flüchtig ihr Spiegelbild darin. Es war mittlerweile fast dunkel, also konnte nicht mehr viel darin zu erkennen sein. Schließlich stütze Sophie ihre beiden Hände hinter sich in den Boden, lehnte sich zurück, atmete ein und antwortete Leno:\\ „Der Schwimmverein organisiert dieses Fest wirklich schön. Das Essen ist gut, ich kann mich einfach ein bisschen ungezwungen unterhalten und immer mal wieder wackle ich zum Ausschank. Einfach die Seele ein bisschen baumeln lassen, weißt du? Ein paar der Leute die hierher kommen kenne ich schon lange und immer mal wieder sieht man ein neues Gesicht. Ich hab keine Lust den Tag damit zu verbringen mich über meine Zeiten zu ärgern.“
\vspace{0.5em} \\
Sie lachte nochmal und beugte sich nach vorne, um mit ihrer Hand ins Wasser zu greifen. Dann schnippte sie Tobi mit einer schnellen Bewegung ein paar Tropfen Wasser auf die Hose. Sie fuhr fort:\\
„Auch wenn ich mich jetzt trotzdem über die Zeit ärgern muss, nämlich in dem Sinn, dass es schon so spät ist. Komm Tobi lass uns losmachen. Sorry, Leno, dass wir dich jetzt hier so sitzen lassen, aber wir sind schon viel länger hier als wir eigentlich vorhatten.“
\vspace{0.5em} \\
Tobi brauchte man nicht lange zu überreden und Leno, der natürlich nicht alleine zurückbleiben wollte, stand mit ihnen auf und gemeinsam gingen sie noch den Weg bis zum Tor. Als sie am Kassiererhaus vorbeikamen, lugte Leno durch das Fenster, um Nacar zum Abschied zuzuwinken, aber er war nicht da. Bald wurde klar, dass Tobi und Sophie in eine andere Richtung laufen mussten als Leno, und sie verabschiedeten sich, und Leno begab sich auf den Heimweg.
\chapter{Lesen}
Hätte Leno eine bessere Alternative gekannt, wäre er am nächsten Tag sicher nicht wieder im Schwimmbad gestanden. Nun stand er hier, den Rucksack in den Händen und musste selbst daraus schließen, dass er die nicht kannte. Und hinge nicht die Bedeutungslosigkeit seines Schwimmens über seinen Gedanken wie ein dichter Nebel, schon bald hätte man ihn seine Bahnen ziehen sehen. Aber die hing über ihm. Und Leno stellte nicht seinen Rucksack neben der Bank ab und duschte sich nicht, ging nicht in das Wasser sondern zur Liegewiese, breitete dort sein Handtuch aus und legte sich hin. Er hatte ein Buch mitgebracht, das er sich vor einiger Zeit gekauft, in ein Regal gestellt und nie wieder angefasst hatte. Heute hatte er es aus eben jenem Regal genommen, kurz abgestaubt und zu seinen Schwimmsachen in den Rucksack gepackt. Und jetzt lag es aufgeschlagen vor ihm und Leno schob den Fokus seines Augenlichts von Links nach Rechts über seine Seiten und las die Worte, die sich der Autor einst ausgedacht hatte.\\
Schon als Leno heute Morgen aufgestanden war, stand es nicht in Frage, ob er auch an diesem Tag Schwimmen würde. Es war ihm ein beträchtlich unsympathischer Gedanke wieder Bahnen im Wasser zu ziehen. Denn das Wasser hatte ihn getäuscht. Es hatte Leno so von sich überzeugt, dass er jede Aufmerksamkeit für das größere Bild verloren hatte. Und es hatte sich gut angefühlt. Das Wasser auf seiner Haut, die Kontraktion seiner Muskeln. Der Lauf des Lebens reduziert auf in ein Notizbuch geschriebene Zahlen. Zahlen, die für immer schnellere Zeiten standen, für immer längeres Training, für ein beständiges, eindeutig zu bezifferndes Vorankommen. Und Hand in Hand damit war der Gedanke gegangen, dass das Schwimmen das exakt Richtige für Leno war. Die trügerische Sicherheit, dass es genau das war, was er wollte. Als wäre das Herausragen als Freizeitschwimmer einer Provinzstadt ihm auf eine bestimmt Weise vom Universum als Ziel vorbestimmt worden. Und diese Sicherheit war es, die gestern von Leno abgefallen war wie ein fauler Apfel. Wie als hätte eine Schiene von der er sich hatte führen lassen plötzlich aufgehört und ihn in den freien Raum entlassen, in dem er jetzt schwebte, schwerelos, ohne Möglichkeit sich aus eigener Kraft wieder in eine von ihm geforderte Richtung zu bewegen. Oder vielmehr richtungs-, als schwerelos, denn eine Schwere hatte Leno sehr wohl. Was ihm seit gestern fehlte war ein Gefühl von Richtung, ein goldener Faden, ein weisender Finger. Wohin, das wäre Leno egal. Weiter, vorwärts, zum Ziel, nur nicht treiben, nicht antriebs-, und orientierungslos treiben.\\
Und heute war Leno erfüllt von einer gewissen Ehrfurcht vor der Leichtigkeit, mit der er sich seiner Sache sicher geworden war. Es war einfach passiert. Das Schwimmen hatte sich selbst zum Richtigen gemacht. Ohne sein Zutun. Ohne bewusste Entscheidung. Leno hatte Spaß gehabt, war schon immer ehrgeizig gewesen und hatte dann eine Hingebung zu seinem Sport entwickelt, die ihn sich mit der Zeit gänzlich über das Schwimmen definieren ließ. Er hatte einen Punkt erreicht, an dem ihm ein Gefühl der Unvollkommenheit überkam, sollte er einen Tag nicht schwimmen. Und sogar heute, an dem Tag, an dem ihn jede Sicherheit verlassen hatte, hatte er einen Teil dieser Identität nicht aufgeben können und war zum Lesen ins Schwimmbad gelaufen. Er blickte auf, sah seinen Rucksack an, in dem Schwimmbrille, Badehose und Handtuch bereit lagen. So weit weg deren Benutzung von seinem Tagesplan waren, er hatte sie dennoch mitgebracht. Einfach nur, weil er sie immer in diesen Rucksack packte. Und dann schwenkte er seinen Blick zum Schwimmbecken und sann über die Möglichkeit nach, dass er sich bei einem anderen Verlauf des Turniers vielleicht tatsächlich heute schon wieder darin wiedergefunden hätte. Mit der selben Sicherheit wie am Tag davor. Mit der selben Zielstrebigkeit. Seinen trügerischen Faden fest in der Hand. Vielleicht an seinem Start arbeitend, vielleicht auch nicht. In jedem Fall hätte er nach dem Training etwas in dem Notizbuch notiert und sich seinem Ziel etwas näher gefühlt.\\
Er atmete tief ein. Er legte ein Lesezeichen in sein Buch, seine Arme neben seinem Körper ab und ließ den Kopf langsam zwischen die Seiten des noch immer aufgeschlagen daliegenden Buchs sinken. Ein leichter Windstoß fuhr ihm den Rücken entlang und ließ ihn etwas frösteln. Er entspannte seine Muskulatur, rollte seine Schultern nach hinten und ließ sich von der Schwerkraft noch ein klein bisschen tiefer in sein warmes Liegetuch ziehen. Das Buch, in dem er sein Gesicht vergraben hatte, roch nach Tinte und Zellulose. Das Tuch war zu kurz für seinen Körper, sodass seine Füße und ein Teil seiner Unterbeine auf dem ungemähten Rasen lagen und das kühle Gras eindrückten. Dicke, dichte Wolken hingen am Himmel, verdeckten die Sonne und gaben dem Tag eine verschlafene Stimmung. Leno wägte ab, ob es ihm zu kalt war, um liegen zu bleiben. Er fühlte in seinen Körper auf dessen ganzer Länge hinein und begann ihn zu strecken und dabei die Augen zusammenzukneifen. Die Muskeln seiner Beine spannten sich, brachten sich in eine gerade Position und kamen in eine angenehme Dehnung. Er wand sich ein Stück nach Links, schob seine Arme an seinem Körper vorbei über den Kopf, was sein Gesicht am Buch entlang nach unten rutschen ließ. Seine Hände strichen über den Rand des Liegetuchs hinweg auf den Rasen. Er öffnete sie, um die Grashalme zwischen seine Finger gleiten zu fühlen. Sie zogen kühle Linien auf seinen Handflächen. Und dann erreichten seine Arme ihre maximale Streckung und er drängte die Fingerspitzen noch ein Stück weiter nach vorne und die Zehenspitzen noch ein Stück weiter nach hinten und brachte die Spannung in Schulter, Oberarm, Unterbein und Oberschenkel auf ihr Maximum. Er drückte den Rücken durch und hielt sich einen Moment in dieser Position. Dann drehte er seinen Körper gänzlich nach Links, machte eine halbe Umdrehung, rollte von seinem Liegetuch auf seinen Rücken ins Gras. Da entließ er die Luft aus seinen Lungen und die Spannung aus seinem Körper. Der Rasen unter ihm fing an, sich unter seinem Gewicht zu beugen und er sank langsam in ein Bett aus Grashalmen hinab. Er öffnete seine Augen und blickte geradeaus nach Oben. Er sah das kräftige Blau im Zenit des Himmels, das sich dem Horizont nähernd in ein helleres, trüberes Blau überging. Er sah die dunkelgrauen Wolken verteilt auf der Himmelsfläche von hellweißen Streifen umrandet. Und an den Stellen, an denen die Sonne durch sie hindurchstrahlte, strömte aus ihnen ein gleißendes Leuchten. Und Leno tat eine Weile nichts als zu liegen und nach Oben zu starren und zu Atmen. Er atmete ein und atmete aus und die Wolken schoben sich über den Himmel. Und dann, nachdem er eine Weile so verbracht hatte, entschied er, dass es Zeit war nach Hause zu gehen, und stand auf. Er packte Buch und Handtuch in seine Tasche, warf sie sich über den Rücken und ging einige Schritte in Richtung des Tors. Dort blieb er nochmal stehen, drehte sich um und betrachtete den Ort, an dem er gelegen war. Das Gras war an drei Stellen eingedrückt. Genau an dem Platz, an dem sich die Grashalme unter einer Last gebeugt hatten. Eine kleine Delle an der Stelle, an der sein Rucksack gelegen war und zwei längliche Einbuchtungen dahinter, nebeneinander. Eine dort, wo das Liegetuch ausgebreitet worden war, und eine zweite daneben, die er hinterlassen hatte, als er seitlich auf den Rücken gerollt war. Für einen unbeteiligten Beobachter würde es später so aussehen, als wären hier zwei Menschen dicht nebeneinander gelegen. Dann riss Leno sich von dem Anblick los und machte sich auf den Weg zu seiner Wohnung.
\chapter{Mehr Lesen}
Am nächsten Tag blieb Leno zu Hause und las das Buch, das er gestern im Schwimmbad dabei hatte. Schon kurz, nachdem er aufgestanden war, setzte er sich an seinen Küchentisch, brühte sich eine Tasse Kaffee auf und klappte es in seinem Schoß auf. Er saß still, ließ seine Augen über die Zeilen wandern und ergründete das Buch Satz für Satz, nur unterbrochen von den kurzen Pausen, in denen er die Seiten umblätterte oder einen Schluck aus seiner Tasse nahm. Es fiel ihm nicht schwer sich voll und ganz auf das Buch zu konzentrieren. Es war ein gutes Buch, und als er sich dieses glücklichen Umstands bewusst wurde, schlug er es um, betrachtete den Buchrücken und merkte sich dessen Autor, „Dale Carnegie“. Dann fuhr er mit dem Lesen fort. Draußen vor dem Fenster schob sich währenddessen gemächlich die Sonne über den Horizont und tauchte Lenos Wohnung in erst mattes, dann immer heller werdendes Licht. Und nachdem er die letzte Seite gelesen hatte, blätterte Leno auch die Rückseite des Einbands um, schloss damit das Buch und hielt es in seiner linken Hand. Hielt es fest in seiner linken Hand und fühlte sich ein Stück mehr als am Morgen.\\
Noch am selben Tag durchsuchte er seine Wohnung nach weiteren Büchern. Er kramte in alten Kartons, tastete in die hinteren Ecken seiner Schubladen. Er hatte immer gerne gelesen, nur hatte er in den letzten Jahren das gebundene Buch mehr und mehr durch Artikel und Beiträge aus dem Internet ersetzt. In seinem alten Zimmer im Haus seiner Eltern, da hätte eine Inventur seiner Sachbücher und Romane länger gedauert. In der kleinen Wohnung, die er seit ein paar Jahren bewohnte, waren bald alle möglichen Verstecke abgesucht. Am Ende fand er einen englischen Kriminalroman, die Novelle „Der alte Mann und das Meer“ von Ernest Hemingway, und zwei Gesundheitsratgeber, die er für sich selbst gekauft und nie gelesen hatte. Alle gefundenen Bücher wurden auf dem Küchentisch gesammelt. Und als er sich sicher war, keines übersehen zu haben, nahm Leno ein Tuch, staubte die Bücher eines nach dem anderen ab und packte sie auf sein Bett. Danach wischte er den Tisch sauber. Und dann setzte sich Leno neben den Buchhaufen, verschaffte sich einen Eindruck von jedem Buch und legte eine Reihenfolge fest, in der er sie lesen wollte. Als er zufrieden mit seiner Ordnung war, stapelte er die Bücher in die hintere linke Ecke der Tischplatte. In die gegenüberliegende Ecke platzierte er das heute fertig gelesene Buch. Und dann blieb Leno einen Augenblick lang vor seinem Küchentisch mit den beiden Bücherstapeln stehen und stellte sich vor, wie er im Lauf der Zeit ein Buch nach dem anderen vom linken Stapel nehmen, sich durch die Seiten arbeiten, den Buchrücken umklappen, und es anschließend auf dem rechten Stapel ablegen würde. Bis an der Stelle, an der sich gerade die ungelesenen Bücher befanden, nur noch die Platte des Küchentischs zu sehen sein würde und der Stapel der gelesenen Bücher ihm bis zur Brust reichen würde. Dann würden die geschriebenen Worte nicht mehr allein in ihren Büchern geschrieben sein, sondern auch ihm gehören.\\
Leno setzte sich auf den Stuhl des Küchentischs, nahm das oberste Buch vom Stapel und begann zu lesen. Es war einer der beiden Gesundheitsratgeber. Ein grob gezeichneter, grüner, großer Apfel war auf dem Einband abgebildet. Anders als beim letzten Buch flossen die Worte nur zäh. Leno ertappte sich oftmals abgelenkt. Er ertappte sich dabei, wie er nebenbei auf seinem Mobiltelefon recherchierte oder gedankenlos an die Wand glotzte und sein Blick wegwanderte von den Seiten des Buchs und gedankenlos im Raum verharrte. Seine Augen schweiften durch sein Zimmer, zu seinem Bett, aus seinem Fenster, oder blieben an den restlichen Büchern hängen. Sein Stuhl war fast perfekt in der Mitte zwischen den Stapeln platziert. Drei der Bücher lagen links von Leno, eines rechts von ihm. Ganz unten im linken Stapel lag der zweite der beiden Gesundheitsratgeber. Er hatte bewusst den einen nach unten geordnet und den anderen ganz nach oben, damit es zwischen den beiden recht ähnlichen Sachbüchern ein wenig Abwechslung gab. Vielleicht hätte er die beiden Bücher in umgekehrter Reihenfolge ordnen sollen. Zufrieden fühlte sich Leno mit den beiden Stapeln jedenfalls nicht. Und dann senkte er den Blick wieder, las mühsam einen Paragraphen, und als er ein weiteres mal aufsah erkannte er, was ihn an den Büchern auf dem Küchentisch störte. Es waren zu wenige. Der Stapel war zu niedrig. Wenn er das Buch in seiner Hand fertig gelesen hatte und auf das Buch von Dale Carnegie legte, dann wären beide Stapel schon fast gleich groß. Dann wäre das Ende schon abzusehen. Dann wäre er wieder ziellos.\\
Leno wollte das Ende nicht absehen können. Er wollte die Bücher von seinem Küchentisch bis unter seine Decke stapeln. Er hatte keine Zeit, sich an etwas heranzutasten. Er wollte sich vollkommen der Sache verpflichten. Und dafür war der Stapel zu niedrig. Er schob also ein Lesezeichen in das Buch mit dem Apfel und klappte ihn auf dem Kopfkissen zu. Dann setzte er sich im Bett auf und fuhr sich mit den Händen ein paar mal über Gesicht und Haare. Er stand auf, streckte sich, ging ein paar Schritte und stellte sich vor den Spiegel. Etwas müde sah er aus und verbraucht. Gerade noch gut genug, um aus dem Haus gehen zu können, dachte er, und zog sich seine Schuhe an. Und wo er schon einmal beim Denken war, dachte er auch daran, dass er eine Tasche brauchte, um die Bücher zu transportieren. Also nahm er sein Handtuch und seine Schwimmsachen aus dem Rucksack, legte sie aufs Bett, warf ihn sich über die Schultern und verließ die Wohnung, schloss die Tür hinter sich und trabte den Hausflur hinunter. Der leere Rucksack auf seinem Rücken sprang dabei munter auf und ab, was sich ungewohnt und komisch anfühlte.\\
Das Treppensteigen tat Leno gut, der seine Beine den ganzen Vormittag noch nicht bewegt hatte. Und als er unten im Treppenhaus ankam und die Haustür aufschob, war sein Geist schon beschwingter und er fühlte sich schon lebendiger. Schwungvoll schritt er ins Freie. Auch wenn es ihm so vorkam, als hätte den ganzen Tag die Sonne nicht geschienen, war es draußen nicht unangenehm. Sie war versteckt hinter einem grauen Schleier, der das Licht dumpf und schal werden ließ und die Welt schwermütig. Es wirkte, als hätte es gerade eben erst geregnet. Vielleicht hatte es das auch. Auch wenn die Luft den schweren Geruch nach Regen vermissen ließ.\\
Er lief in Richtung Stadtmitte los. Noch wusste er nicht, wo er die Bücher besorgen wollte, und er wusste auch nicht welche Art von Büchern er besorgen wollte. Aber er war sich sicher, dass sich diese Dinge auf dem Weg klären lassen würden. Und während er sich schon Schritt für Schritt fortbewegte, nahm er zunächst seine Kopfhörer aus dem Rucksack, entknotete sie behutsam und steckte sie sich in die Ohren. Nahm sein Mobiltelefon aus der Hosentasche, schloss das AUX Kabel an, entschied sich für ein Album und ließ es spielen. Dann suchte er rasch auf Google Maps nach Buchläden und Bibliotheken in der Stadt, wählte willkürlich ein Ziel aus, startete eine Navigation und steckte das Mobiltelefon wieder zurück in die Hosentasche. Er würde es erst in der Innenstadt wieder brauchen.
\vspace{0.5em} \\
Der Weg in die Stadt war zu einem Teil der Strecke der selbe wie der ins Schwimmbad, und wäre der Rucksack auf Lenos Rücken nicht leer und völlig gewichtslos gewesen, dann hätte es eine exakte Wiederholung all der Tage sein können, an denen er eben dahin gelaufen war. Und dieses Gefühl durchtrieb Leno. Es fühlte sich an, als würde ihm jeden Moment einfallen, dass er doch auf dem Weg ins Training war. Dass er unterwegs war zum Schwimmbad, und sein Rucksack leer war und er keine Badehose, keine Schwimmbrille und kein Handtuch dabei hatte. Und dass sein Weg somit ohne Sinn war, weil er im Schwimmbad angekommen feststellen musste, dass er ohne diese Sachen nicht trainieren konnte. Mehr noch. Er fühlte die Sinnlosigkeit aller seiner Wege ins Schwimmbad, als sein Rucksack noch voller Schwimmsachen war und sein Geist voller Enthusiasmus, als er mit aller Kraft und Ehrgeiz geschwommen war. Und er konnte das Gefühl nicht wegdenken. Es half ihm nicht, dass ihm bewusst war, dass sein Ziel nicht das Schwimmbad war, und dass der leere Rucksack sehr wohl seinen Sinn hatte. Das Gefühl blieb. Monatelang hatte er sich dem Schwimmen verpflichtet. Hatte täglich trainiert, war mit Gedanken an Zeiten und Rollwenden eingeschlafen und aufgestanden und hatte sein Leben um das Schwimmen herum geplant. Und alles was heute davon übrig war stand im Notizbuch geschrieben und würde keinen weiteren Zweck mehr erfüllen als allmählich zu Staub zu zerfallen. Leno konnte nicht anders, als in das Gefühl einzutauchen und sich zu fragen, ob es sein Schicksal war, sich immer neuen sinnlosen Tätigkeiten hinzugeben. Immer neuen Götzen nachzujagen, während er versuchte, seinen Platz in der Welt zu finden. Und einerseits tat es ihm gut diese Gedanken zuzulassen und gleichzeitig bereute er es. Er wusste, dass er sich keine Rechenschaft schuldig war, solange er das tat das, was er ehrlich sinnvoll fand und sich richtig anfühlte. Mehr konnte er von sich nicht verlangen. Und er wusste auch, dass Wissen und Fühlen zwei sehr gegensätzliche Konzepte waren, deren Gemeinsamkeit es war, dass man ihnen beiden nicht trauen konnte. Er konnte nur die Entscheidungen treffen, die ihm richtig erschienen. Er konnte nur nach den Plänen handeln die er für gut befand. Aber trotz seines neu gefassten Plans und seiner neu gefundenen Entschlossenheit blieb ihm die enorme Unsicherheit. Wie sollte er wissen, ob er am Ende seines Vorhabens nicht wieder enttäuscht werden würde? War er erneut in Gefahr, sich zu verlieren? War er in Gefahr die Worte hunderter Bücher zu konsumieren und darüber ein paar wichtige, leise, gesprochene Sätze zu überhören? War er in Gefahr eine Chance zu verpassen, weil er abgelenkt sein würde? Eine sich öffnende Tür zu übersehen, weil sein Blick nach unten auf ein Buch gerichtet sein würde?\\
Während der letzten Gedanken war Lenos Gang langsamer geworden und er hatte die Kopfhörer aus seinen Ohren genommen, damit er sich besser konzentrieren konnte. So weit er die Straße hinunterblickte, war sie menschenleer. Es waren hier nie viele Leute zu Fuß unterwegs. Heute war er auf der ganzen Strecke noch niemandem begegnet. Vielleicht lag es tatsächlich am kühlen Tag. Die Sonne versteckte sich nach wie vor schüchtern hinter trüben Wolken, die stets so aussahen, als würde jeden Moment ein ansehnlicher Regen aus ihnen platzen. Der Tag selbst allerdings brauchte den Regen nicht, um grau zu sein. Leno bemerkte, dass er fast zum Stehen gekommen war, und er fing an, seine Schritte aufs neue zu beschleunigen. Er setzte die Kopfhörer wieder auf, zog mit einer eleganten Bewegung sein Mobiltelefon aus der Hosentasche und wählte ein neues Lied aus. In ein paar Schritten passierte er die Weggabelung, an der es rechts einen kleinen Pfad runter zum Schwimmbad ging. Als er an ihm vorbeikam, drehte er den Kopf ein wenig und warf einen Blick den Weg entlang, dann sah er wieder geradeaus. Der leere Rucksack saß sanft auf seinem Rücken, übte wenig Druck aus, wackelte bei jedem Schritt hin und her, auf und ab. Jetzt war es nicht mehr weit in die Stadt. Noch die Straße runter, dann Links und später wieder Rechts. Dann würde er irgendwann in einem Laden oder einer Bibliothek stehen, würde sich mehrere Bücher aussuchen, bezahlen, sich auf den Rückweg machen, und dann würde er irgendwann wieder daheim sein. Unabhängig davon wie viel er noch nachdachte. Unabhängig davon was falsch oder richtig war, gut oder schlecht. Unabhängig von jeder Unsicherheit.
\vspace{0.5em} \\
Aber würde er irgendwann ein neues Notizbuch kaufen und anfangen, täglich die Anzahl seiner gelesenen Seiten zu notieren? Die Möglichkeiten lagen auf der Hand. Er könnte die Anzahl der gelesenen Seiten in einer Stunde notieren und versuchen, mit der Zeit schneller zu werden. Er könnte ein System entwickeln, mit dem er kontrollieren konnte wie viel ihm von den Büchern in Erinnerung blieb. Dann könnte er versuchen, mittels verschiedener Methoden mehr im Gedächtnis zu behalten. Und obwohl er momentan nichts davon in Betracht zog, wusste Leno, dass ihm Dinge dieser Art nahe lagen. Dass es nicht unwahrscheinlich war, dass er irgendwann über die Anschaffung eines neuen Notizbuchs nachdenken würde. Dass er anfangen würde, Karteikarten mit dem Inhalt der Bücher zu beschriften. Und obwohl die Parallelen zu seinen Anstrengungen als Schwimmer auf der Hand lagen, weigerte sich Leno, sie als absolut und endgültig schlecht abzuurteilen. Folgte unweigerlich ein Desaster, wenn man sich dazu entschied, ein Notizbuch zu führen? Verlor man zwangsläufig den Blick fürs Ganze, wenn man schneller und effektiver lesen wollte? Wenn man mehr im Gedächtnis behalten wollte? Der Gedanke, etwas könnte ihm gut und richtig erscheinen und gleichzeitig vollkommen falsch sein, setzte Leno schlimm zu. Möglichkeiten, die ihm sinnvoll und vielversprechend erschienen, fühlten sich an wie Bedrohungen. Konnte er sich nicht mehr Vertrauen? Er war immer davon ausgegangen, die Fähigkeit zu besitzen, sein Leben durch seine Entscheidungen zu seinem Besten zu gestalten. Wenn ihm das nicht möglich war, könnte er jetzt auch in ein Kino gehen anstatt in eine Bücherei. Dann hätte er auch zu Hause bleiben können, antriebslos treibend im Fluss des Lebens. Sich irgendwann irgendwo anspülen lassen. Wer könnte sagen, ob es ihm dann nicht besser gehen würde, als wenn er ständig versuchen würde gegen jegliche Strömung im Fluss anzuschwimmen. Niemand könnte das. Niemand konnte wissen, was der nächste Morgen brachte. Niemand hatte alle Variablen unter seiner Kontrolle. Aber Leno wusste sehr genau, dass er sich niemals damit zufrieden geben würde, einer Art Schicksal ausgeliefert zu sein.\\
Er war derjenige, der sein Leben gestaltete. Diese Gewissheit war wichtiger als alles andere. Er allein hatte die Verantwortung zu übernehmen für seine Zukunft. Und Verantwortung beinhaltete das Richtige und Notwendige zu tun. Und wenn er das Ergebnis seiner Taten nicht wissen konnte, dann war es eben seine Pflicht nach seiner besten Überzeugung zu handeln. Vielleicht war er nicht der Schlauste oder Weiseste. Vielleicht hatte er noch nicht genug vom Leben gelernt. Vielleicht war er auch genetisch einfach nicht dazu in der Lage, etwas wirklich Gutes für sich zustande zu bringen. Egal wie die Ausgangslage war. Egal wie die Aussichten waren. Er würde in jedem Fall mit dem arbeiten müssen, was er hatte. Sich selbst. Und dann würde er mit dem zufrieden sein müssen, was am Ende dabei heraus kam. Das war er sich schuldig. Nicht mehr und nicht weniger.\\
Er konnte Bücher über alles lesen. Bücher über Psychologie, Bücher über Geschichte, Bücher über den Aufbau des menschlichen Körpers. Bücher über das Basteln von Papierflugzeugen. Er konnte fantastische Geschichten lesen, welche die Fesseln der Realität ablegten und sich nur durch die Vorstellungskraft ihres Verfassers Grenzen setzen ließen. Und er konnte wahre Geschichten lesen, die oft noch fantastischer anmuteten als die ausgedachten. Er konnte Biographien der bedeutendsten Frauen und Männer lesen und so ein klein bisschen an ihrem Geist teilhaben. Oder Bücher lesen über ihre Errungenschaften und Geisteshaltungen in Ethik, Philosophie und Physik. Oder er konnte Abends ein Buch aufschlagen zu keinem anderen Zweck als dem, sich unterhalten zu lassen. Nein, von einem realistischen Standpunkt betrachtet, konnte man mit einem Buch keinen Fehler begehen. Nicht wenn man es las und nicht, wenn man es auf seinen Küchentisch stapelte.
\vspace{0.5em} \\
So dachte Leno, zog sein Mobiltelefon aus der Hosentasche, übersprang die nächsten beiden Lieder seiner Playlist, und folge dem blauen Navigationspfeil auf dem Display durch die Stadt.
\chapter{Die Buchhandlung}
Eine Computerstimme unterbrach seine Musik, als Leno an der Buchhandlung vorbeilief, den er beim Loslaufen in Google Maps ausgewählt hatte:
\vspace{0.5em} \\
„Sie haben ihr Ziel erreicht.“
\vspace{0.5em} \\
Irgendwann war er mit seinen Überlegungen zu einem Ende gekommen; dann hatte ein gutes Lied es geschafft seine Aufmerksamkeit von den um ihn herum schwirrenden Gedanken loszureißen, und die letzten Meter der Strecke waren wie im Flug unter ihm hinweggeglitten. Er blieb stehen, zog sein Mobiltelefon aus der Hosentasche, beendete die Navigation und spähte durch das Schaufenster des Ladens. Aus der großen Scheibe blickte ihm neugierig die Reflexion seines eigenen Gesichts entgegen. Es spiegelte sich verschwommen im Glas. Er richtete seinen Fokus auf den weiter hinten liegenden Bereich. Im unteren Drittel befand sich eine Auslage in der einige Bücher, Pappaufsteller und Dekorationsartikel drapiert waren. Durch die oberen zwei Drittel konnte man weit in den Laden schauen, denn die Bücherregale waren senkrecht zum Schaufenster ausgerichtet und so blickte man von der Straße aus also die schmalen Gänge zwischen den Regalen hinunter. Neben den vielen Büchern standen wahllos auch ein paar Stühle im Raum, auf denen vereinzelt Menschen saßen und in Zeitschriften blätterten. Es sah einladend aus. Und nachdem er sich so ein erstes Bild von dem Laden gemacht hatte, ging Leno durch dessen offene Tür hinein. Dingeling. Der Verkäufer, der hinter der Kasse saß, sah von einem Buch auf, bedachte ihn mit einem dünnen Lächeln und Leno nickte ihm seinerseits kurz zu. Schnell entzog Leno sich der Situation, indem er inmitten der Regale verschwand. Er ging geradeaus nach hinten, tiefer in den Laden, und betrachtete die über den verschiedenen Bereichen angebrachten Beschriftungen. Es roch nach Papier, Tinte, Klebstoff und Kunstleder. Durch die Regale konnte er — zwischen den Büchern und den nächsthöheren Regalbrettern gab es jeweils eine hohe Lücke — in den gegenüberliegenden Gang schauen. Es war wenig los und Leno flanierte einfach einmal durch die Reihen. Irgendwann blieb er vor einem willkürlichen Regal stehen und kippte den Kopf etwas zur Seite, um die Titel der Bücher besser lesen zu können. Er begutachtete die Buchrücken. Die Einbände waren in satten Farben gestaltet. Er hob seine Hand über den Kopf, legte den Zeigefinger auf ein Buch, dessen Titel ihn neugierig gemacht hatte und zog es aus der Reihe. Mit beiden Händen hielt er es ins Licht. Sorgfältig untersuchte er das Cover. Dann drehte er es um und las den Klappentext aufmerksam. Als er fertig war, schob er das Buch zurück in die Lücke und suchte ein neues aus. So arbeitete er sich die Regale entlang, und immer wieder, wenn er von einem Buch besonders überzeugt war, schob er es nicht wieder zurück ins Regal, sondern legte es in seinen Rucksack. Bei seinem Suchprozess achtete er gezielt darauf, Bücher aus verschiedenen Themengebieten auszuwählen. Und von den beiden Romanen, die er einpackte, bezeichnete einer sich selbst als „Klassiker“ und einer als „Moderne Literatur“. Er wollte unbedingt für eine gewisse Ausgewogenheit sorgen, in der Mischung die er mit nach Hause nehmen würde.
\vspace{0.5em} \\
Als er schließlich bis zum Zerbersten gefüllt war, griff Leno den Rucksack mit beiden Händen von oben an den Trägern und hievte ihn zwischen den Regalen hervor in Richtung Kasse. Der Verkäufer bemerkte ihn aus den Augenwinkeln, und er sah von seinem Buch auf, klemmte ein Lesezeichen in die Seiten, und legte es unter den Tresen.
\vspace{0.5em} \\
„—Hallo“
\vspace{0.5em} \\
murmelte er in Lenos Richtung. Man konnte sofort hören, dass er eine raumfüllende, sonore Stimme besaß. Leno grüßte freundlich zurück, stellte den prall gefüllten Rucksack vor der Kasse ab und hob einzeln die Bücher auf den Tresen. Währenddessen stand der Verkäufer auf und begann damit, die Ware abzukassieren. Nachdem Leno das letzte Buch aus dem Rucksack genommen und vor den Verkäufer gestapelt hatte, richtete er sich auf und wartete. Der Angestellte, der die Bücher über den Scanner schob, war groß und dünn — bestimmt zwei Köpfe größer als Leno. Seine langen Finger glitten gewissenhaft über die Bände, während er deren Oberfläche nach dem richtigen Barcode absuchte. Buch um Buch zog er den Stapel so über den Scanner. Als er ganz unten bei „Der Hobbit“ von J.R.R. Tolkien angekommen war, wendete er das Buch zweimal mit skeptischem Blick, sah dann zu Leno und sagte zu ihm:
\vspace{0.5em} \\
„Ich glaube dieses Exemplar hatten wir damals als Ausstellungsstück, es sieht etwas ramponiert aus. Ich gebe dir 50\% Rabatt darauf.“
\vspace{0.5em} \\
Und als Leno seine Stimme jetzt deutlicher hörte, konnte er sich an den großen Kerl hinter der Kasse erinnern. Er war vor einiger Zeit mit Anthony und ihm zusammen in der Warteschlange vor dem Nachtclub gestanden. Er war der Typ, der damals laut die Geschichte erzählt hatte, der die Raucher gemeinsam zugehört hatten. Piep. Der Mann hatte den Hobbit über den Scanner gezogen und klapperte jetzt mit seinen langen Fingern auf der Tastatur.
\vspace{0.5em} \\
„Und ich geb dir auf die gesamte Rechnung nochmal 15\%. Weil du über 5 Bücher einkaufst.“
\vspace{0.5em} \\
Leno bedankte sich, bezahlte, packte die Bücher eilig wieder in den Rucksack. Als er fertig war, schwang er sich diesen auf den Rücken und verabschiedete sich vom großen Mann mit der lauten Stimme. Dingeling. Beim Verlassen machte die Tür das selbe Geräusch das sie auch gemacht hatte, als er in den Laden hineingegangen war. Draußen zögerte Leno kurz, blieb stehen und warf von der Straße aus noch einmal einen Blick durch das Schaufenster zurück in die Buchhandlung, und wieder war das Erste, das ihm in die Augen fiel, die Reflexion seines eigenen Gesichts in der Scheibe. Es sah ein bisschen abgekämpft und verdutzt aus.
Und wie er auf der Straße stand und seine Reflexion musterte, nahm er an sich selbst eine Eigenschaft wahr. Es war ein stilles Wahrnehmen. Kein bewusstes Ausformulieren von Gedanken. Es war ein tieferes, unterbewusstes Erkennen einer Tatsache, die schon lange Zeit bestand hatte. Und er stand da und musterte sich, und es gefielen ihm die Haare nicht, die ihm wild über der Stirn nach oben standen. Und es gefiel ihm nicht, wie er seinen Kopf nach vorne fallen ließ und damit seiner Haltung einen gebückten Eindruck gab. Seine gesamte Erscheinung war unrund, grenzte sich von den Reflexionen der anderen Gegenstände seltsam ab und wirkte auf Leno etwas verloren. Und Leno war nicht der selbstüberzeugteste Mensch und verband mit jenem Bild im Spiegel neben seinen körperlichen Makeln auch eine ganze Reihe von persönlichen, denen er in diesem Moment in die Augen sah und die ihn aus bekannten Augen zurück ansahen. Die Behauptung aufzustellen, Leno würde sich selbst immer gefallen, wäre eine Lüge gewesen. Er gefiel sich nicht immer, mochte sich nicht immer und war nicht immer immer mit sich zufrieden. Aber wenn Leno in den Spiegel sah, dann blickte ein Mensch daraus zurück, den er liebte. Er selbst. Er liebte sich ehrlich, zeitlos, bedingungslos und mit ganzem Herzen. Und gerade jetzt, als er vor dem Schaufenster der Buchhandlung stand, da erkannte er es. Nicht so, dass er es in sein Tagebuch geschrieben hätte, würde er eines führen, und auch nicht so, dass er es in einem Fragebogen angekreuzt hätte. Es war keine plötzliche Einsicht. Es war keine aus Vernunft gefällte Entscheidung. Es war ein Gefühl. So real wie der Boden unter Lenos Füßen, wie der Druck der Rucksackträger auf seinen Schultern und die in seinen Lungen beim Atmen zirkulierende Luft. Und er erlaubte es sich, diese Liebe, die er in sich trug, einen Augenblick lang zu fühlen. Als würde er eine Nadel auf eine Schallplatte legen, die sich schon sein ganzen Leben unbeachtet gedreht hatte. Einen Moment lang verstand er, dass er wirklich von ganzem Herzen versuchte zum Besten für sich zu handeln. Und dass er sich deshalb nicht schuldig fühlen musste.\\
Dann zog Leno sein Mobiltelefon aus der Hosentasche, steckte sich die Kopfhörer in die Ohren und machte sich auf den Heimweg. Zuhause angekommen stapelte er die neuen Bücher zu den alten auf die linke Seite des Küchentischs. Dann nahm er sich den Gesundheitsratgeber mit dem großen, grünen Apfel auf dem Einband, und las darin bis er müde war.
\chapter{Wieder Anthony}
Zwei ereignislose Tage waren vergangen, seit Leno die Bücher gekauft hatte, und er saß nun den Großteil seiner wachen Stunden mit einem aufgeschlagenen Buch im Schoß am Küchentisch. So saß er auch, als plötzlich sein Mobiltelefon klingelte. Es war mitten am Vormittag, er hatte die Wohnung heute noch nicht verlassen und sich vielleicht insgesamt 30 Meter zwischen den Zimmern hin und her bewegt. Klingeling, rief das Telefon. Er sah kurz auf, überlegte, ob er das Buch weglegen und den Anruf annehmen sollte, ließ es aber doch einfach unbeachtet schellen. Sein Blick senkte sich wieder aufs Papier, und er las weiter. Zu tief war er im Inhalt des Buches versunken. Aber das Mobiltelefon gab nicht auf und klingelte kurze Zeit später wieder. Und da stand er schließlich auf, ging mit dem Buch in der Hand zum Bett und warf einen Blick auf die Displayanzeige. Dann nahm er den Anruf an. Es war Anthony.
\vspace{0.5em} \\
„Na, du alter Spitzensportler?“, begrüßte ihn Anthony.
\vspace{0.5em} \\
„Hi Anthony.“, grüßte Leno zurück.
\vspace{0.5em} \\
„Leno, ich bin gerade zufällig vorbeigekommen und stehe draußen vor der Tür. Lässt du mich rein? Sag?“
\vspace{0.5em} \\
Er war überrumpelt, aber wirklich überrascht war er nicht. Wenn Anthony anrief, dann war damit zu rechnen, dass er bereits vor der Tür stand. Also legte er sein Mobiltelefon zur Seite, ging durch die Wohnung ins Treppenhaus und lief die Stufen hinab, um seinen Freund abzuholen. Unten angekommen zog Leno mit einem beherzten Ruck an der Haustür, und schon durch den ersten kleinen Spalt strömte Anthonys Stimme ins Innere. Und als die Tür schließlich ganz offen war, trafen Lenos Augen auf die seines Freundes, der, anstatt auch nur irgendwelche Anstalten zu machen ins Haus zu kommen, von der Straße aus in seine Richtung redete. Weil Anthony schon viel zu früh losgesprochen hatte, bevor die Tür auch nur ansatzweise offen war, hatte Leno seine ersten Worte nicht ganz verstehen können. Und jetzt, als er seinen Freund sah, blickte dieser in Lenos Augen und sprach einfach weiter. Redete in sein an den Mund gehaltenes Telefon. Und da realisierte Leno, dass die Worte, die er nicht gehört hatte, auch nicht für ihn bestimmt gewesen waren. Er lehnte sich in den Türrahmen, während ihn Anthony weiterhin entschuldigend ansah und seine Sprachnotiz zu Ende formulierte. Irgendwann war er fertig. Dann schob er sein Mobiltelefon zurück in die Hosentasche, befreite die Welt aus ihrem pausierten Zwischenzustand, und schlug mit Leno ein. Und in diesem Moment bemerkte Leno, wie sehr er sich über den Besuch freute. Vor einigen Sekunden, als er den Anruf angenommen und mit Anthony telefoniert hatte war er sich noch unsicher gewesen. Aber als ihre Hände ineinander klatschten, er seinen Freund begrüßte und nach oben in die Wohnung führte, als Anthony seine Schuhe abstreifte und sich erschöpft aufs Bett fallen ließ, da wusste Leno, dass er sich sehr freute.
\vspace{0.5em} \\
„Ich bin so fertig!" seufzte Anthony vom Bett her, "Ich hab gestern auf eBay Kleinanzeigen einen Schallplattenspieler gekauft und hab den gerade mit John abgeholt.“
\vspace{0.5em} \\
Leno setzte sich neben Anthony auf die Bettkante. „Und wo ist der Plattenspieler jetzt? Und wo ist John?“, fragte er interessiert nach.
\vspace{0.5em} \\
„John fährt ihn erstmal zu sich nach Hause, ich hol ihn später ab, ich muss gleich nochmal los.“, antwortete Anthony und blickte zu Leno hoch, „Leno hast du vielleicht irgendwie Kaffee da oder so? hm? sag?.“
\vspace{0.5em} \\
Leno hatte Kaffee, und natürlich war er ein guter Gastgeber, erhob sich, und ging in die Küche um Anthonys Wunsch zu erfüllen. Das Kaffeepulver stand auf der schmalen Küchenzeile bereit, die Tasse, die er immer benutzte, stand daneben — eine weitere holte er aus dem Schrank. Leno agierte in der Küche, erhitzte Trinkwasser im Wasserkocher und füllte einen Teelöffel Kaffeepulver in jede der Tassen. Danach noch eine Löffelspitze. Dabei blieb Anthony auf dem Bett liegen, war anscheinend wirklich erschöpft und hatte es dennoch geschafft eine gewisse Geschäftigkeit in die Wohnung zu bringen. Mit ihm war auf eine Weise auch wieder etwas Normalität eingetreten. Er hatte ein Loch in Lenos Ballon gestochen, aus dem die Luft langsam entwichen war und sich jetzt wieder frei verteilen konnte. Die Zeit war wieder messbar geworden in Sekunden und Stunden. Der Wasserkocher gab ein anschwellendes Summen von sich.
\vspace{0.5em} \\
„Hey, Leno, hast du vielleicht irgendwo eine Steckdose, an der ich mein Handy laden kann? sag?“, rief Anthony vom Bett aus in die Küche.
\vspace{0.5em} \\
Da ließ Leno nochmal vom Kaffee ab. Er drehte sich um und sah, dass Anthony sich aufgesetzt hatte und Handy mitsamt Ladekabel schon in der Hand hielt. Beides nahm er ihm ab und schloss das Mobiltelefon an der Steckdose neben dem Bett an. Dann ging er zurück in die Küche, goss das kochende Wasser auf das Kaffeepulver. In seinem Rücken konnte er hören, wie sich Anthony durch einen tief seufzenden Atemzug wieder Energie einverleibte, sich aufrappelte, zwei Schritte ging und sich auf den Stuhl in der Küche setzte. Und als Leno sich umdrehte, um Anthony seinen Kaffee auf den Tisch zu stellen, hatte dieser bereits eines der Bücher vom Küchentisch genommen. Er sah ihn fragend an:
\vspace{0.5em} \\
„Hast du's angeschlossen? Hm? Das hat nur noch 3\%.“
\vspace{0.5em} \\
„Ja, lädt.“
\vspace{0.5em} \\
Der einzige Sitzplatz in der Küche war nun besetzt, und Leno lehnte sich so bequem wie möglich zurück gegen die Kante der Küchenzeile. Seine Tasse hatte er gleich in der Hand behalten, hob sie zum Mund und fing an, den Kaffee durch Pusten auf Trinktemperatur zu kühlen. Vor ihm saß Anthony tief im Stuhl vor dem Tisch, auf dem sich seit einigen Tagen die Bücher stapelten. Sauber in zwei Stapeln, der hohe ungelesene Stapel links und der kleine Stapel rechts, der nur noch aus einem Buch bestand, dem von Dale Carnegie, weil das andere sich in Anthonys Hand befand. Er hatte es scheinbar wahllos an irgendeiner Seite aufgeschlagen, in der er jetzt konzentriert las. Leno hatte gerade also keine Aufgabe, war nur Zuschauer des lesenden Anthony und hatte so noch etwas Zeit, in die neue Situation hineinzugleiten. Er führte seine Tasse zum Mund, spürte die Hitze des Kaffees an seiner Oberlippe und pustete sanft. Zum Trinken war er zwar noch viel zu heiß. Etwas anderes zu tun gab es im Moment aber nicht.\\
„Hast du das hier gelesen?“, fragte Anthony nach einer Weile und hielt dabei das Buch mit dem grünen Apfel auf dem Cover einige Zentimeter nach oben. Und Leno, der auf diese Frage gewartet hatte, antwortete nicht ohne Stolz, er habe es vor drei Tagen begonnen und sei heute Morgen damit fertig geworden. Da schaute ihn Anthony genau an, kniff dabei scherzhaft skeptisch die Augen zusammen und widmete sich dann mit neuer Bestätigung wieder dem Buch. Die anderen Fragen blieben ungestellt. Und weil es die beste Alternative war, stellte auch Leno seinen Kaffee ab. Suchte nach seinem aktuellen Buch. Fand es schließlich am Fensterbrett über dem Bett. Ging zurück in die Küche, lehnte sich wieder gegen den Herd, griff in das Lesezeichen, klappte es auf und las weiter. Und so saßen die beiden eine Zeit lang schweigend in der Küche und lasen. Draußen zwitscherten die Vögel. Ab und zu raschelte eine Seite beim Umblättern. Nach einer Weile stieß Anthony energisch seinen Zeigefinger in das Buch:
\vspace{0.5em} \\
„Der schreibt hier, dass die ständige Verfügbarkeit von rotem Fleisch Gift für die Gesundheit der modernen Gesellschaft ist. Und hier später, dass sich durch den Konsum von rotem Fleisch das Krebsrisiko um das Vierfache erhöht!“
\vspace{0.5em} \\
Mit dem Finger noch immer auf dem Papier sah er Leno an, so als ob er für diese Aussage eine Rechtfertigung von ihm erwarten würde.
\vspace{0.5em} \\
„Ja, in eigentlich allem was man liest steht, dass man gar kein oder zumindest viel weniger Fleisch essen sollte. Meistens geht's dabei aber nicht um Krebs sondern um Herzerkrankungen.“
\vspace{0.5em} \\
Ehrlich entrüstet klappte Anthony das Buch zu, legte es auf den Tisch, hatte keinen Finger mehr in der Seite und kein Lesezeichen. Er hatte genug.
\vspace{0.5em} \\
„Warum muss alles was gut schmeckt immer ungesund sein? Ich les' am besten gar nicht weiter, wer weiß was ich sonst noch alles Falsches esse!“
\vspace{0.5em} \\
Da entgegnete ihm Leno ebenso ehrlich: „Es ist wirklich besser, wenn du nicht weiter liest. Es ist mit Sicherheit kein gutes Buch. Ich habe auch wirklich keine Ahnung mehr, warum ich es gekauft habe.“, und dann ergänzte er, weil er es so meinte, „Lies lieber das Buch rechts. Ich weiß, der Titel ist etwas seltsam, aber es ist interessant und gut geschrieben. Jedes Kapitel ist eine kleine Erzählung.“
\vspace{0.5em} \\
„Und du meinst das lohnt sich, ja? Oder?“, Anthony sah ihn zweifelnd an, „Weißt du was, ich muss jetzt sowieso wieder los. Wenn ich darf, dann leih ich mir das Buch aus. Geht das klar?“
\vspace{0.5em} \\
Damit stand Anthony vom Stuhl auf und Leno schob sich an ihm vorbei, griff sich das Buch, von dem er geredet hatte, vom Tisch und drückte es ihm in die Hand. Mehr, weil der Platz in der Küche für die beiden zu eng war, als weil er wirklich los musste, trat Anthony in den Flur hinaus vor die Wohnungstür. Und bevor Leno ihm folgte, blieb er noch einen Augenblick lang vor dem Küchentisch stehen und schob das Buch mit dem grünen Apfel, in dem Anthony gerade gelesen hatte aus der Mitte der Tischplatte auf die rechte Seite. Das Buch, das er Anthony ausleihen würde, war die letzten Tage ganz unten im rechten Stapel gelegen, und jetzt, wo Anthony es in der Hand hielt, lag dort wieder nur der eine, einsame Gesundheitsratgeber. Schließlich verließ auch Leno die Küche, begleitete seinen Freund nach unten und verabschiedete ihn. Wieder oben angekommen ging er zurück in die Küche, öffnete das Fenster, stemmte die Ellenbogen auf die Fensterbank und streckte den Kopf nach vorne. Es schepperte laut. Vor dem Haus spielten die Nachbarskinder Fußball auf ein Garagentor. Weiter hinten konnte Leno Anthony beobachten, wie er sich mit schnellem Schritt entfernte.

\chapter{Laufen statt Schwimmen}
Immer wieder schepperte es laut, wenn der Ball hinter dem Torwart in das Blechtor einschlug. Leno schaute aus dem Fenster. Seit Anthony weg war, musste er wieder an das Schwimmturnier zurückdenken. Mittlerweile war seitdem eine ganze Woche vergangen und Leno hatte bisher keine Anstalten gemacht das Schwimmbad ein weiteres Mal zu betreten. Vor einigen Tagen hatte er das Haus verlassen, um die Bücher aus der Buchhandlung zu holen. Seitdem saß er den Großteil seiner Zeit auf dem Stuhl in der Küche und las. Und allmählich musste er sich eingestehen, dass ihm das Fehlen einer körperlichen Ertüchtigung ernsthaft zu Schaffen machte. Er war unruhig, schlief schlecht, hatte keinen Appetit. Als er noch täglich trainiert hatte, war er jeden Abend erschöpft gewesen und wie ein Stein in sein Bett gefallen. Immer hatte er sich vom letzten Training für das nächste Training erholt. Die vergangenen Nächte dagegen hatte er sich lange umher gewälzt anstatt zu schlafen, und die anstrengendste Tätigkeit, die er diese Woche von seinem Körper verlangt hatte, war das Hochsteigen der Treppen zu seiner Wohnung gewesen. Er fühlte sich wie eine bis zum Anschlag gespannte Feder, die sein Geist mit Hilfe schier unendlicher Willenskraft in der Spannung halten musste. Und nicht immer gelang ihm das gänzlich. Ihm fehlte das Beißen, ihm fehlte der körperliche Kampf, ihm fehlten die Herausforderung und der Puls, der ihm das Blut durch den Körper trieb. Und ihm fehlten die Endorphine, die durch eben jene Betätigung in sein System gespült worden waren und deren Ausbleiben jetzt auf seine Stimmung drückte. Er wusste, dass er den Sport brauchte. Ganz auf diesen zu verzichten kam nicht in Frage. Genauso wenig wie es in Frage kam, ein Handtuch und eine Badehose in seinen Rucksack zu packen und einfach wieder zum Schwimmbad zu laufen. Er brauchte eine Alternative.
\vspace{0.5em} \\
Schließlich, als er sich nicht mehr anders zu helfen wusste, öffnete Leno die am wenigsten benutzten Schubladen seiner kleinen Wohnung. Sie protestierten knirschend, konnten sich jedoch nicht wehren. Er kniete sich vor ihnen auf den Boden, zog sie bis zum Anschlag auf und wühlte in ihnen mit vorsichtigen Fingern. Und tatsächlich brauchte er nicht lange zu suchen, bis ein Paar abgetragene Turnschuhe in einer der Schubladen auftauchte. Er griff sich einen der beiden aus dem Gerümpel, stand auf und probierte ihn. Er passte. Dann zog er sich ein altes T-Shirt über und schlüpfte in den zweiten Schuh. Um sich möglichst wenig Zeit zu geben, seine Entscheidung nochmal zu überdenken, band er die Schnürsenkel mit einem Doppelknoten fest, trank noch ein paar Schlücke Wasser und verließ ohne großes Nachdenken die Wohnung. Er sprang die Treppen hinunter. Unten stieß er die Haustür kraftvoll auf und schritt endlich ins Freie.\\
Draußen trat ihm der Tag hell entgegen. Die Sonne schien ihm ins Gesicht und hieß ihn willkommen in der Welt derer, die an schönen Tagen das Haus verließen. Von der Wucht der Begrüßung überrascht war Leno gezwungen stehen zu bleiben und einige Male schnell zu blinzeln. In seinem Rücken fiel krachend die Tür ins Schloss. Ein Luftzug streifte seinen Nacken, was ihm eine leichte Gänsehaut verursachte und die kleinen dünnen Haare, die seinen Hals bedeckten, stellten sich etwas auf. Er nahm sein Mobiltelefon aus der Hosentasche und steckte sich die Kopfhörer in die Ohren. Sein Daumen fuhr ein paar mal gekonnt über das Display und wählte eine automatisch vom Musikdienst zusammengestellte Playlist aus. Dann begann er leicht zu traben. Drückte sich mit jedem neu gesetzten Fuß erneut ab. Nach einigen langsamen Schritten steigerte er sein Tempo. Das war es jetzt also. Seine Beine schlugen immer wieder nacheinander am Boden auf. Der Wind strich ihm über Gesicht und Haare. Seine Arme begannen sich im Takt mitzubewegen. Seine Hände waren zu lockeren Fäusten geballt. Er erinnerte sich an einen Trick, den ihm ein Triathlet einmal erklärt hatte, und nahm die Daumen in die Hände. In seinen Ohren klang laut die Musik aus den Kopfhörern. Dann beschleunigte er noch etwas mehr, wich ein paar Spaziergängern in der Kurve aus und zog an ihnen vorbei. An dieses Gefühl würde er sich gewöhnen können, dachte er. Wieder erhöhte er sein Tempo. Seine Daumen fühlten sich in den Fäusten unnatürlich an und er nahm sie wieder nach außen. Er lief eine Weile einfach nach vorn, ohne eine besondere Route im Sinn. Er kannte sich gut aus und würde jederzeit eine Schleife nach Hause finden. Mit jedem Schritt wurden seine Füße leichter und sein Atem rhythmischer. Ihm wurde warm und er begann zu Schwitzen. Und er konnte spüren, wie es ihm gut tat. Also weiter. Er bemerkte, dass seine Füße bei jedem Schritt zuerst mit der Ferse aufsetzten und bei jedem Aufprall ein kleiner Ruck durch seinen Körper ging. Um dem entgegen zu wirken versuchte er sich besser abzufedern, mehr Kontrolle über seinen Laufstil zu gewinnen und kippte fortan den Fuß beim Absenken etwas weiter nach vorne, sodass zuerst der Fußballen auf den Boden traf. Es gelang ihm gut, machte das Laufen aber anstrengender. Nicht nur körperlich, sondern vor allem psychisch. Denn nun musste er sich auf jeden Schritt im Einzelnen konzentrieren. Jeden Fuß bewusst setzen. Und obwohl er wusste, dass das so besser und gesünder war, fühlte es sich dennoch unrund an. Aber das war normal, war doch eine lange Zeit vergangen, seit er das letzte Mal gelaufen war. Mit der Zeit würde er sich an das Setzen der Füße mit den Fußballen gewöhnen und dann wieder ganz von selbst diesen Laufstil bevorzugen.\\
Mittlerweile hatte der Untergrund zu Kies gewechselt. Der Weg war schmal geworden und ruhig. Er hatte sich inzwischen ein gutes Stück von den letzten Wohnhäusern entfernt. Die Strahlen der Sonne hatten an Kraft verloren und die Luft hatte abgekühlt. Während draußen nur Lenos Schritte und sein schwerer Atem zu hören waren klang in seinen Ohren immer noch irgendein automatisch ausgewählter, stimmungsvoller Upbeat-Popsong. Ausdauernd schwang er seine Beine und ließ Meter um Meter des Wegs hinter sich. Büsche, Bäume und Sträucher zogen an ihm vorbei. Er selbst nahm davon aber kaum noch etwas wahr. Seine Aufmerksamkeit galt ganz seinen Beinen. Seit er sich so konzentrieren musste, war sich nicht mehr sicher, ob die Länge seiner Schritte in Ordnung war. Es schien ihm, als wären seine Schritte mit der Zeit immer weiter geworden. Er musste so aussehen, als würde er versuchen dem Gallop einer Gazelle nachzueifern. Also zwang er sich, seine Schritte kürzer zu setzen.\\
Langsam machte sich die Anstrengung bemerkbar und ließ Lenos Atem schnell und flach werden. Jetzt galt es, sich von ihr nicht aus dem Rhythmus bringen zu lassen. Seine Beschäftigung mit sich selbst drängte die Szenerie um ihn herum fern in den Hintergrund. Jedes Körperteil verlangte einzelne Zuwendung. Setzen des Fußes. Stoßen des Beins. Bewegung des Armes und der Hand. Fuß. Bein. Fuß. Arm. Bein. Arm. Lunge. Sein Atem war die letzten Minuten unrhythmisch und flach geworden. Allmählich konnte er sich vor der Realität nicht mehr verschließen. Er konnte das Tempo so nicht länger halten. Er musste sich dazu zwingen abzubremsen und seinen Atem zu verlangsamen. Gleichzeitig bemerkte er, wie seine Fersen wieder hart auf dem Boden aufschlugen, und er musste es geschehen lassen. Es ging jetzt darum, nicht stehen zu bleiben. Wenn er diese Phase überstanden hätte, würde es leichter werden. Nach einiger Zeit stabilisierte sich seine Atmung wieder und er lief weiter mit der neuen, gemäßigten Geschwindigkeit. Kein Wind strich ihm über das Gesicht. Kein Gefühl breitete sich in ihm aus. Die Fäuste waren geballt, die Füße wurden gesetzt. Nach einigen weiteren Metern spürte er ein Ziehen in der linken Wade, das er fürs Erste ignorierte. Sein Körper musste sich an die neue Belastung gewöhnen. Das war normal. Ein kurzes Stück musste er noch geradeaus, dann könnte er rechts abbiegen und wieder in die Richtung seiner Wohnung zurücklaufen. Also biss er auf die Zähne und lief weiter. Er hatte sich nicht seine alten Turnschuhe angezogen um vielleicht ein bisschen zu Laufen und dann vielleicht den Rest spazieren zu gehen. Er hatte das Haus laufend verlassen und würde auch laufend zurück kommen. Als er schließlich an der Rechtsabbiegung angelangt war, wurde die Straße von zu vielen Autos befahren, um sie direkt zu überqueren. Leno war gezwungen zu warten. An einer nahen Straßenlaterne konnte er sich abstützen, während Autos und Motorräder an ihm vorbei rasten. Er drückte den Fuß gegen die Laterne, presste seine Zehen gegen das Metall, seine Ferse auf den Boden und lehnte sich nach vorn. Erst leicht, dann mit Kraft, als er sich sicher war, dass die Dehnung das Ziehen in der Wade nicht noch verstärkte. Der Druck auf den Muskel tat gut. Sein Körper war warm und seine Haut kühl vom Schweiß. Dann, als sich eine Lücke in den Massen der Fahrzeuge auftat, löste er sich von der Straßenlaterne, fing wieder an zu Laufen und überquerte die Straße. Er hatte jetzt etwas mehr als die Hälfte geschafft und befand sich auf den Heimweg. Seine Muskeln und Sehnen waren von der Pause etwas fest geworden. Die Füße hatten angefangen unangenehm in den alten Schuhen zu krampfen. Vielleicht hätte er mit einer kürzeren Strecke beginnen sollen, aber jetzt war er eben unterwegs und daran war jetzt nichts mehr zu ändern. Im Laufen nahm er die Kopfhörer aus den Ohren und schob sie in die Hosentasche. Dann zog er sein Mobiltelefon heraus und stoppte die Musik.
\chapter{Zwischenziel}
Leno schaffte noch ungefähr ein Drittel der übrig gebliebenen Strecke. Seine Füße hatten sich zwar wieder erholt, dafür begannen jetzt seine beiden Knie beunruhigend zu schmerzen. Kurz unter der Haut, außen gleich neben den Kniescheiben, pulsierte ein immer stärker werdendes, stechendes Brennen im Rhythmus seiner Schritte. Er konnte sich nicht mehr vor dem Offensichtlichem verschließen. Er musste akzeptieren, dass es ihm nicht möglich sein würde, den kompletten Rückweg ohne Pause zu Laufen. Sein Körper war zu müde und geschunden. Jeden Moment — so kam es ihm vor — würde an irgendeiner Stelle etwas wichtiges abreißen, herausspringen oder durchbrechen und damit einen wirklichen Schaden anrichten. Und da sah Leno endlich ein, dass irgendwo eine Grenze lag, über die hinaus man sich nicht mit bloßem Willen treiben konnte. Und er machte sich gedanklich auf die Suche nach einem Zwischenziel, an dem er eine kurze Pause machen und etwas ausruhen können würde.
Er entschied sich schnell für einen geeigneten Ort. In ein paar hundert Metern gab es einen kleinen Weiher mit ein paar Bänken, auf denen sonntags manchmal ein paar älteren Leute saßen, um Enten zu füttern. Bis zur nächsten Kreuzung war es nicht weit. Dort angekommen bog er ab. Er würde sich ein bisschen auf einer der Bänke ausruhen und später über den Kiesweg am Weiher vorbei zu seinem Wohnblock zurückkommen. Ein steiniger Feldweg führte in einem langgezogenen Bogen zu dem Weiher hinunter, und weil es auf ihm ziemlich steil nach unten ging, musste er in einen langsamen Trott verfallen, bei dem er seinen Körper mit jedem Schritt etwas abbremste, um nicht zu schnell zu werden. Trotzdem zwang er sich eisern, nie zu gehen, sondern weiter zu laufen. Als er nach einer gefühlten Ewigkeit ein paar der größeren Sträucher umkurvt hatte, tauchten endlich die ersten Bänke in seinem Sichtfeld auf. Gleich dahinter lag das Ufer des kleinen Weihers. Er trabte den Rest des Weges hinunter, bis dieser nicht mehr bergab führte und in eine Fläche aus brachem, trockenem Erdboden überging. Unvermittelt ließ ein Reflex ihn nach Rechts wegkippen. Ein Stechen schoss ihm durch den Fuß. Zu allem Überfluss hatte er jetzt auch noch einen Stein im Schuh. Er biss sich auf die Zähne. Um nicht bei jedem zweiten Schritt kräftig auf den Stein zu treten, musste er für die letzten Meter den wesentlichen Teil seines Gewicht auf die linke Seite verlagern. Jedes Abdrücken mit dem linken Bein feuerte dadurch einen hellen Schmerz aus seinem Knie in seinen Kopf. In einem Akt größter Disziplin schaffte er es, auch die letzten Schritte bis zur Bank zu joggen, und so konnte er zumindest einen Teil der Niederlage noch abwenden.
\vspace{0.5em} \\
An der Bank angekommen ließ er augenblicklich alle Spannung aus seinen Muskeln fahren und fiel wie ein luftig gefüllter Kartoffelsack in die Bretter. Sank tief in sie hinein und hielt sich mit den Händen an der Rückenlehne fest. Sein Atem ging stoßweise durch seinen offenen Mund und sein Bauch hob und senkte sich im selben Rhythmus.
Erst jetzt fiel ihm auf, dass auf der Bank noch eine weitere Person saß und er warf einen kurzen Blick zu der neben ihm sitzenden Gestalt. Dabei traf sich sein Blick mit dem des Anderen. Er hatte ihn wohl ebenfalls gemustert. Das war Leno unangenehm und er drehte den Kopf wieder weg. Er erkannte den Mann, der neben ihm saß, sofort. Es war der Verkäufer aus der Buchhandlung; der große, hagere Geschichtenerzähler mit der lauten Stimme.
Mit der Situation überfordert sah Leno wieder nach vorne zum Wasser, atmete, und fühlte in seinen Körper hinein. Der Stein in seinem rechten Schuh zwickte immer noch unangenehm, aber er hatte noch nicht die Kraft etwas dagegen zu unternehmen. Das nassgeschwitzte T-Shirt klebte an seiner Brust und seinem Bauch und kühlte die oberste Hautschicht. Darunter pumpte sein Puls heißes Blut durch seinen Körper. Es war relativ windstill und ruhig hier am Weiher — auch sein Sitznachbar machte kein Geräusch. Leno bildete sich ein, die einzelnen Schläge seines Herzens hören zu können.
Mittlerweile gingen Lenos Puls und Atmen ein klein wenig langsamer und es war genug Kraft in ihn zurückgekehrt, um den störenden Stein aus seinem Schuh zu entfernen. Er beugte sich nach vorn, öffnete die Schleife, griff an die Ferse und streifte ihn mit einer notdürftigen Bewegung vom Fuß ab. Als die Öffnung des Schuhs nach unten zeigte fiel in kleiner, weißer Kieselstein vor Leno auf die Erde. Er griff nach ihm, wuchtete seinen Körper nach oben. Nahm mit zwei entschiedenen Schritten Schwung auf und feuerte den Kiesel so weit er konnte aufs Wasser. Dann setzte er sich wieder hin. Aus den Augenwinkel hatte er gesehen, wie der Andere ihn dabei beobachtete.
\vspace{0.5em} \\
„Puh, du siehst ja ganz schön fertig aus. Ganz schön hart trainiert heute, was?“, sprach er Leno von Links an.
\vspace{0.5em} \\
„Sag mal, bist du nicht der Verkäufer aus der Buchhandlung?“, entgegnete Leno ohne Umschweife, und er war selbst ganz überrascht von seiner Direktheit. Aber der Geschichtenerzähler nahm die Frage sehr gelassen und antwortete:
\vspace{0.5em} \\
„Ja, da hast du Recht, ich kann mich an dich auch erinnern, du hast neulich buchstäblich einen ganzen Rucksack voller Bücher bei uns gekauft.“
\vspace{0.5em} \\
Und weil Leno noch sehr erschöpft war und auch nicht wusste, wie das Gespräch genau weiter gehen sollte, schwiegen die beiden wieder eine Zeit lang. Bis der Andere die Stille füllte:
\vspace{0.5em} \\
„Wie heißt du?“
\vspace{0.5em} \\
„Leno“, sagte Leno.
\vspace{0.5em} \\
„Mein Name ist Ingo“, sagte der Andere, und fuhr fort:
\vspace{0.5em} \\
„Weißt du, ich mache diesen Job in dem Laden nur übergangsweise. Eigentlich ziehe ich mit ein paar Kumpels gerade ein kleines Softwareprojekt hoch. Wir haben da eine kleine Nische gefunden. Aber aller Anfang ist schwer, was? Ich sage dir unser Business-Modell ist bis ins Detail durchgerechnet. Bis das aber natürlich anläuft muss ich mich bisschen über Wasser halten, da ist der Job in der Buchhandlung nicht schlecht.“
\vspace{0.5em} \\
Während er redete, kramte Ingo eine Schachtel Zigaretten aus seiner Hosentasche hervor, nahm sich eine und hielt sie Leno hin. Dieser griff ebenfalls zu. Sie zündeten nacheinander die Zigaretten an und Leno, der immer noch schwach und ausgelaugt war, musste nach den ersten Zügen ein wenig husten. Ingo ignorierte das. Er schien selbst nachdenklich zu sein. Und nach einer Zeit drehte er den Kopf zu Leno und sprach mit ihm in einem eindringlichen Tonfall, der zusammen :
\vspace{0.5em} \\
„Weißt du Leno, du kannst nicht frei sein, so wie du es gern hättest. Du musst es dir so vorstellen: Du wirst mit einer Kette um den Fuß geboren. Und an der anderen Seite der Kette ist eine Kugel aus Eisen befestigt. Du wirst älter, lernst zu krabbeln, lernst zu gehen und die Kugel wächst dabei mit und du ziehst sie hinter dir her und störst dich nicht dabei. Du kennst dich selbst nur mit Kette und Kugel und siehst die gleiche Kette auch an dem Fußgelenk deiner Mutter und deines Vaters. Und dann wirst du älter und alle die du kennenlernst ziehen eine Kugel hinter sich her und sie gehört nicht weniger zu ihnen als ihr linker Arm, sowie deine Kugel nicht weniger zu dir gehört als eines deiner Körperteile.\\
Du musst verstehen, dass in dieser Metapher Freiheit nicht ist ohne Kugel zu Leben, Leno, in dieser Metapher ist Freiheit, zu merken, dass die Kugel dich behindert und dann immer wieder das Sandpapier in die Hand zu nehmen und ein bisschen etwas von der Eisenkugel abzuschmirgeln.“
\chapter{Leben}
Und als Ingo fertig war, da spürte Leno auf einmal eine unerklärliche Anziehungskraft vom Wasser ausgehen. Wie als würde ihn das Universum durch einen sanften Druck in Richtung des Weihers schieben. Noch nie hatte Leno einen so starken Ruf des Schicksals, der überirdischen Vorbestimmung gefühlt. Mit einem lautem Ächzen zog er sich das triefnasse T-Shirt über den Kopf und warf es mit einer energischen Bewegung von sich. Stieg aus dem verbliebenen Schuh und streifte sich die Socken von den Füßen. Die kalte Luft auf der nackten Haut fühlte sich gut an. Er löste den Knoten am Gummibund seiner Sporthose, stand auf und schob sie zusammen mit der Unterhose runter zu seinen Knöcheln, stieg aus den am Boden zusammengefallenen Hosenbeinen, erst mit dem rechten, dann mit dem linken Fuß und ging nackt in Richtung des Wassers. Er ging langsam und bestimmt. Immer einen Fuß vor den anderen setzend. Und als er den ersten Schritt ins Wasser machte, sein Fuß beim Eintauchen die glatte Oberfläche durchbrach und sie zum Zittern brachte, da fühlte es sich so an als ob er durch die Oberfläche des Wassers in eine andere Welt übergehen würde. Der Weiher wurde schnell tiefer und mit jedem Schritt verschluckte er ein weiteres Stück von Lenos Körper. Nach zwei Schritten waren schon beide Knie im Wasser, nach zwei weiteren auch Schwanz und Sack, und kurz nachdem sein Bauchnabel auch eingetaucht war, warf er sich nach vorne, versenkte sich vollständig im Wasser und begann zu schwimmen. Kraftvoll zog er den rechten Arm unter sich durch, drückte sich gegen den Widerstand nach vorne, schlug wild mit den Beinen. Streckte den linken Arm noch etwas gerader aus, als er den rechten über seinem Körper wieder zum Kopf führte und dort erneut ins Wasser stieß. Er schloss die Augen. Jetzt schob er den linken Arm nach hinten, drehte den Kopf, nahm den Mund aus dem Wasser und atmete tief ein. Er musste weder in einer markierten Bahn bleiben, noch aufpassen, nicht am gegenüberliegenden Beckenrand anzustoßen, und so entschied er, die Augen noch eine Weile geschlossen zu halten, und ganz im Gefühl des an ihm vorbeigleitenden Wassers aufzugehen. Er achtete nicht auf Technik und Form, paddelte nur irgendwie mit den Beinen und schwang irgendwie die Arme durchs Wasser. Zwanglos. Ohne etwas dabei von sich zu verlangen. Er genoss es einfach zu schwimmen. Er mochte die Bewegungen, die man immer wieder ausführte und er mochte es, wie sich der Strom des Wassers anfühlte, wenn es versuchte, sich einen Weg an seinem Körper und seinen Händen vorbei zu bahnen. Er mochte die Schwerelosigkeit und das Gefühl auf der Haut an den Stellen, an denen sie sich immer wieder kurz in die Luft hob, nur um im nächsten Augenblick wieder vom Wasser verschluckt und umschwemmt zu werden. Und er bewegte sich eine ganze Weile so durchs Wasser und entfremdete sich an den Empfindungen, die er am Schwimmen so liebte.
\vspace{0.5em} \\
Schließlich drosselte Leno seine Geschwindigkeit, schob den Kopf aus dem Wasser, öffnete die Augen und sah sich um. Der Weiher war viel größer, als es von der Bank aus gewirkt hatte. Und noch mehr überraschte Leno, dass er nicht der einzige zu sein schien, der durch das Wasser schwamm. Zuerst erkannte er Magnus, wie er rhythmisch seine Arme einsetzte. Dann, nicht weit von ihm entfernt, Mike, der krampfhaft versuchte in seinem Fahrtwasser mit ihm mitzuhalten. Dicht hinter den beiden folgten Nacar, Michael, Malte und Mark, und Leno streckte einen Arm aus dem Wasser und winkte ihnen zu, aber sie bemerkten ihn nicht. Er drehte seinen Kopf, und aus einer anderen Richtung kommend sah er Tobi, Sophie und Julia. Man konnte sofort erkennen, wie elegant Sophie sich durchs Wasser bewegte. Aber sie schwamm nicht übermäßig schnell und die anderen beiden waren ungefähr auf einer Höhe mit ihr. Plötzlich platschte es laut neben Leno, er drehte sich zum Geräusch hin, und beobachtete verblüfft, wie Anthony und John an ihm vorbeizogen. Und dann erkannte Leno, dass sie alle, so wie er, auf nichts zuschwammen. Dass es kein Ziel gab für keinen von ihnen, an dem sie irgendwann ankommen und zu einem Ende finden könnten. Ihr Schicksal war es wie seines, immer zu stoßen, zu schlagen, sich vorwärts zu schieben, weiter zu machen. Immer auf dem Weg. Ständig irgendwo hinschwimmend. Das Ferne immer in einem besseren, einem schönerem Licht sehend als das Momentane. Sie alle wussten nicht, dass das Wasser überall das selbe war und sie egal wie weit sie kamen immer weiter schwimmen mussten. Leno erkannte in diesem Moment das Missverständnis. Sie alle waren vereint in dem Glauben, dass ihnen etwas fehlte. Dass es ein Defizit gab, das ausgeglichen werden musste. Eine angeborene Fehlfunktion jedes Menschen. Und unser Schicksal ist es unser Leben lang gegen diese Einschätzungen zu kämpfen. Weil du immer nur schwimmst und nie ausruhst. Weil du nirgends ankommen kannst. Weil es ohne Ziel kein Vorankommen gibt. Es gibt nur dich, das Wasser, und das Schwimmen.\\


\chapter{Epilog}
Leno saß auf der Bank neben dem Schwimmbecken und spürte sein Herz schnell in seiner Brust schlagen. Sein Blick war auf die an der gegenüber liegenden Wand angebrachte Uhr gerichtet und er konzentrierte sich darauf, tief und gleichmäßig zu atmen. Der Zeiger der Uhr machte eine volle Umrundung, danach etwas weniger als eine halbe. Dann stand Leno auf und stieg ins Wasser. Seine Arme arbeiteten beim Hinablassen seines Körpers gegen die Schwerkraft und die Kälte des Wassers wanderte langsam an seiner Haut entlang bis zu seinem Hals. Dann war er ganz eingetaucht. Hielt sich mit einer Hand am Beckenrand fest. Dabei fixierte er stets aufmerksam den Zeiger der Uhr bei seiner kreisenden Bewegung. Schließlich, als der Zeiger eine unsichtbare Markierung erreicht, holt Leno nochmals tief Luft, taucht ins Wasser ein, stößt sich kraftvoll mit den Beinen vom Beckenrand ab, gleitet etwas, und beginnt dann, seine Arme zyklisch durchs Wasser zu ziehen. Es umströmt Leno völlig, als wäre Leno ein Fixpunkt, an dem ein Fluss vorbeizieht.
% begin back matter
\end{document}
% END THE DOCUMENT