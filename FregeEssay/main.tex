%%%%%%%%%%%%%%%%%%%%%%%%%%%%%%%%%%%%%%%%%
% Diaz Essay
% LaTeX Template
% Version 2.0 (13/1/19)
%
% This template originates from:
% http://www.LaTeXTemplates.com
%
% Authors:
% Vel (vel@LaTeXTemplates.com)
% Nicolas Diaz (nsdiaz@uc.cl)
%
% License:
% CC BY-NC-SA 3.0 (http://creativecommons.org/licenses/by-nc-sa/3.0/)
%
%%%%%%%%%%%%%%%%%%%%%%%%%%%%%%%%%%%%%%%%%

%----------------------------------------------------------------------------------------
%	PACKAGES AND OTHER DOCUMENT CONFIGURATIONS
%----------------------------------------------------------------------------------------
\PassOptionsToPackage{ngerman}{babel}
\documentclass[ngerman,12pt]{diazessay} % Font size (can be 10pt, 11pt or 12pt)
\setcounter{tocdepth}{4}
\usepackage{setspace}
  \usepackage[
backend=biber,
style=trad-abbrv,
]{biblatex}
\addbibresource{sample.bib}

%----------------------------------------------------------------------------------------
%	TITLE SECTION
%----------------------------------------------------------------------------------------

\title{\textbf{Eine Diskussion über Freges Unterscheidung zwischen Sinn und Bedeutung von singulären Termen} \\
\vspace{3em}
{\Large\itshape \vspace{4em}}} % Title and subtitle

\author{
FAU Erlangen \\
\textit{Philosophische Fakultät, Grundkurs Theoretische Philosophie} \\
Dozent Prof. Dr. Erasmus Mayr, Tutor Kevin Braun \\
\textbf{Simon Dorr, Matrikel-Nr. 23132534} \\
simon.dorr@fau.de \\
2. Fachsemester Philosophie/Politikwissenschaft \\
SoSe23
} % Author and institution
 % Date, use \date{} for no date

%----------------------------------------------------------------------------------------

\begin{document}
\begin{doublespace}
\selectlanguage{ngerman}
\maketitle % Print the title section
\newpage
%----------------------------------------------------------------------------------------
%	ABSTRACT AND KEYWORDS
%----------------------------------------------------------------------------------------

%\renewcommand{\abstractname}{Summary} % Uncomment to change the name of the abstract to something else

\tableofcontents
\newpage
%----------------------------------------------------------------------------------------
%	ESSAY BODY
%----------------------------------------------------------------------------------------

\section{Einleitung}

Gottlob Freges Aufsatz "Über Sinn und Bedeutung"
Frege nähert sich in seinem Aufsatz .
\par\bigskip   
Der schottische Philosoph David Hume stellt im Abschnitt: \textit{Von den Motiven des Willens} seines \textit{Traktats über die menschliche Natur} zu seinem Bedauern fest, dass der Affekt als Motiv unserer Handlungen einen üblen Ruf besitzt. Seiner Meinung nach besteht sogar eine Forderung der Allgemeinheit, nach der jedes vernünftige Wesen seine Handlungen ausschließlich nach seiner Vernunft einrichten soll. Hume selbst hält diese Forderung für unsinnig. Und er macht es sich zum Ziel, diese Sinnlosigkeit zu beweisen. Also stellt er die These auf, dass die Vernunft allein niemals einen Willen erzeugen oder unterdrücken kann \cite[siehe S.484]{Hume.2013}. Die Erkenntnisse aus der Behandlung dieser These verwendet Hume in einem späteren Abschnitt seines Buches dazu, seinen Gedanken vom allgemeinen Handeln auf das sittliche Handeln zu übertragen. Seine daraus entstehende zweite These besagt, dass: „[D]as sittlich Gute und das sittlich Böse allein durch die Vernunft zu unterscheiden [nicht möglich ist]“ \cite[S.433]{Hume.2013}. In diesem Essay ist es mein Ziel, in schlüssiger und verständlicher Form die Argumentation für die zweite These darzulegen, weswegen natürlich eine ebensolche Argumentation für seine erste These notwendig ist.
\par\bigskip   
Des Weiteren werde ich die Frage behandeln, in welchem Verhältnis Hume die Vernunft und die Affekte zueinander stehen sieht.

%------------------------------------------------

\section{Erklärung des Begriffs „singulärer Term“ und Einführung des verwendeten Beispiels}
Ein singulärer Term ist ein Ausdruck oder eine sprachliche Einheit, die auf eine individuelle Entität oder einen bestimmten Gegenstand verweist. Singuläre Terme beziehen sich auf konkrete Dinge oder Personen in der Welt und dienen dazu, diese zu identifizieren oder über sie zu sprechen. Der Begriff „singulärer Term“ wurde nicht von Frege geprägt, doch das Konzept geht auf seine Verwendung der Begriffe „Name“ und „Zeichen“ zurück \cite[siehe S.26]{Frege}. Dabei muss ein singulärer Term keinesfalls aus nur einem Wort bestehen. Auch: „Der, gegen den ich gestern 1v1 Basketball gespielt habe“ bezieht sich auf eine konkrete Person und fällt deshalb unter die Kategorie „singulärer Term“.

Freges Konzeption von singulären Termen steht im Gegensatz zu allgemeinen Begriffen oder Ausdrücken, die auf eine Klasse von Entitäten oder Gegenständen verweisen. Während allgemeine Begriffe eine Gruppe von Objekten zusammenfassen, richten sich singuläre Terme auf einzelne Elemente innerhalb dieser Gruppe.

Die beiden singulären Terme, anhand derer ich Freges Argument für die Unterscheidung von Sinn und Bedeutung rekonstruieren werden, sind die Namen „Karl Malone“ und „Mailman“. Beide Namen beziehen sich auf eine spezifische Person, nämlich den ehemaligen amerikanischen Basketball-Profi Karl Malone. Den Namen Karl Malone verlieh ihm seine Mutter, den Namen Mailman verlieh ihm die Basketball Gemeinschaft während seiner Profi-Karriere. Zur Entstehung dieses Namens werde ich im Kapitel

\section{Zweite Beweisführung: Der Affekt als originales Etwas}

Nachdem Hume in der dargestellten Weise einige Absätze lang philosophiert, räumt er selbst ein, dass die Vorstellung, die Vernunft sei den Affekten vollkommen ausgeliefert, etwas sonderbar ist. Aus diesem Grund möchte er sie durch eine zweite Beweisführung untermauern. Er beginnt den zweiten Beweis mit der Untersuchung des Wesens jedes Affekts. Hume stellt dabei fest, dass ein subjektiv von mir erfahrener Affekt keine Repräsentation von etwas ist, das außerhalb meiner Erfahrung liegt. Der Affekt ist daher kein Abbild eines sich in der Welt befindlichen Gegenstands, sondern ein „originales Etwas“, das tatsächlich in mir ist \cite[siehe S.486]{Hume.2013}.
\par\bigskip   
Es gibt also nichts, was der Affekt repräsentiert. Deshalb gibt es auch nichts, was mit dem Affekt verglichen werden kann, um eine Übereinstimmung bzw. Nicht-Übereinstimmung zu erkennen. In dieser Erkenntnis aber liegt das Fundament von Wahrheit und Irrtum \cite[siehe S.534]{Hume.2013}. So zeigt Hume, dass ein Affekt keinerlei Beziehung zur Wahrheit haben und damit auch nicht der Vernunft wider-, oder entsprechen kann.

\subsection{Was es bedeutet, dass der Affekt nicht der Vernunft wider-, oder entsprechen kann}

Weiterhin rundet Hume in den letzten Zeilen des Abschnitts: \textit{Von den Motiven des Willens} seine Beweisführung ab und stellt endgültig fest, dass jede Handlung immer auf einen Affekt zurückzuführen ist \cite[siehe S.489]{Hume.2013}.\footnote{Diese Aussage ist im Sinne der Beweisführung vereinfacht. In der Wirklichkeit ist es immer eine Vielzahl von Affekten, die um ihren Einfluss auf unseren Willen konkurrieren. Hume sagt außerdem, dass sich die Gewichtung der Affekte gegeneinander mit der Stimmung des Menschen ändert.} Und da für einen Affekt gilt, dass er nicht der Vernunft wider-, oder entsprechen kann, muss also dasselbe für jede Handlung gelten. Es gilt also, dass das sittlich Gute einer Handlung nicht in ihrer Übereinstimmung mit der Vernunft liegen kann, da es so eine Beziehung mit der Vernunft gar nicht gibt \cite[siehe S.534]{Hume.2013}.
\par\bigskip   
Aus dem selben Beweis folgt auch, dass die Vernunft nicht der Ursprung der Sittlichkeit sein kann: Denn die Erkenntnis des sittlich Guten oder sittlich Bösen hat die Macht, Einfluss auf unsere Handlungen auszuüben, wie ich im Kapitel 2.2 erkläre. Da die Vernunft aber nie einer Handlung wider-, oder entsprechen kann, kann sie auch nicht diese geforderte Macht besitzen \cite[siehe S.534]{Hume.2013}.

\section{Das Verhältnis von Vernunft und Affekt}

Nachdem ich mich bisher ausschließlich mit Beweisführungen dahingehend befasst habe, worin das Verhältnis von Vernunft und Affekt \textbf{nicht} besteht, wende ich mich jetzt dem zu, worin Hume das Verhältnis von Vernunft und Affekt stattdessen sieht. Dabei ist dieses Verhältnis unbedingt nur als ein indirektes zu verstehen. Der fälschliche Eindruck des direkten Verhältnisses entsteht dadurch, dass die Änderung eines Affekts unter einem bestimmten Umstand sofort einer Tätigkeit der Vernunft folgen kann \cite[siehe S.488]{Hume.2013}. Dieser bestimmte, notwendige Umstand ist das Bestehen eines Irrtums. Und das Verhältnis von Vernunft und Affekt besteht in der Fähigkeit der Vernunft, mich über diesen Irrtum aufzuklären \cite[siehe S.536]{Hume.2013}.
\par\bigskip   
Zwei Arten von Irrtümern sind möglich: Der erste Irrtum kann über die Existenz eines Gegenstands bestehen. In diesem Fall gibt es eine Differenz zwischen der Wirklichkeit des Gegenstands und meiner Vorstellung von ihm. Will ich beispielsweise unbedingt einen Urlaub auf den Malediven machen, kann mein Wunsch auf einem Irrtum über die Eigenschaften einer Insel beruhen. Die zweite Art des Irrtums kann mir bei meinem Urteil über einen Ursache/Wirkung Zusammenhang unterlaufen. Mache ich beispielsweise täglich Sit-Ups, so kann mein Wille dazu vom Urteil geleitet sein, diese Handlung würde meinen Bauch bis zum Sommer in ein Sixpack verwandeln. Auch dieses Urteil kann vielleicht falsch sein \cite[siehe S.536]{Hume.2013}.
\par\bigskip   
Konsequent folgert Hume aus seinen bisherigen Schlüssen folgendes: Sobald wir uns der Wahrheit über einen solchen Irrtum bewusst werden, ändern sich gleichzeitig mit der Auflösung des Irrtums auch unsere Affekte \cite[siehe S.488]{Hume.2013}.

%------------------------------------------------

\section{Schluss}

Dieses nun zuletzt besprochene Verhältnis von Vernunft und Affekt ist natürlich kein unbedeutendes, hat aber keine Bedeutung für die spezielle Frage nach dem eigentlichen Ursprung unserer Affekte. Denn haben wir ein klares, korrektes Bild von einer Sachlage, dann können Vernunft und Affekt in keinerlei Verhältnis zueinander stehen. Und dementsprechend kann dann auch niemals unvernünftig genannt werden, was immer unser Wille im Kontext dieser Sachlage ist.
Diese Schlussfolgerung spitzt Hume treffend zu:
\begin{quote}
„Es läuft der Vernunft nicht zuwider,
wenn ich lieber die Zerstörung der ganzen Welt will als einen
Ritz an meinem Finger.“ \cite[siehe S.487]{Hume.2013}
\end{quote}
Ich hoffe dieser Essay konnte nachvollziehbar darlegen, warum Hume mit dieser Aussage recht hat. Keine Handlung und kein Wille können je durch die Vernunft entstehen. Und weil mich das sittlich Gute doch zum Handeln drängt, so muss jedes sittliche Urteil am Ende von nichts anderem als einem mir eigenem Gefühl geleitet sein.
%----------------------------------------------------------------------------------------
%	BIBLIOGRAPHY
%----------------------------------------------------------------------------------------
\newpage
\printbibliography


%----------------------------------------------------------------------------------------
\end{doublespace}
\end{document}
